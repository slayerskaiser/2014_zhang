\chapter{Conclusions and Future Work}

%%%%%%%%%%%%%%%%%%%%%%%%%%%%%%%%%%%%%%%%%%%%%%%%%%%%%%%%%%%%%%%

\section{Summary}

This thesis confirmed the results of previous investigations. In particular, the 22nd Dynasty Abydos faience beads of this study have compositions and microstructures similar to the beads in the 2007 study by Tite \emph{et al.}\ with the same distinctions (time period and production site). Similar conclusions on the raw materials and glazing method were reached by both investigations.

This thesis also improved upon previous studies by employing new techniques. Compositional mappings were made that showed the concentration of each element over an area of the faience. This technique allows the investigator to easily notice how the concentration of certain elements change in two dimensions and not just one. In addition, an attempt was made to analyze the unglazed inner surface found in faience beads. While some differences between it and the body were noted, the cause of these differences was not determined.

Lastly, this thesis explored potential modern applications for faience technology. It was determined that there are areas that could benefit from the technology of faience. The economic and environmental considerations of faience usage were also determined.

%%%%%%%%%%%%%%%%%%%%%%%%%%%%%%%%%%%%%%%%%%%%%%%%%%%%%%%%%%%%%%%

\section{Future Work}

Future studies of faience will need to consider three factors: the time period, the production site, and the specific type of object. This study and previous ones~\cite{vandiver83,vandiver98,tite07} have noted how these factors can greatly impact the composition and microstructures of faience. However, it would be difficult for one study to have diversity in each of these areas. Therefore, another recommendation is that future researchers need to collaborate in order to map the entire spectrum of faience.

A different area of study is the replication of faience, which still needs a lot of research. Future replication studies should focus on hybrid glazing methods to better understand how ancient faience was glazed. Furthermore, replication studies need to consider the wide range of colors found in ancient faience.

Additionally, modern applications of faience technology should be further explored. Conversely, current nanotechnology could be used to improve faience by increasing the quality of the raw materials to allow for better glazing.

%This thesis set the framework for future work by demonstrating that there is more to learn about faience. Future work should be more of a quantitative nature and should apply the techniques of this study to a wider range of faience objects. In particular, the impurities in faience can be used to locate the exact source of the raw materials of faience.

%Future work should also investigate reproducing faience. Little is known about the mechanisms behind glaze formation, particularly in the cementation method. In-depth knowledge of the glazing process gained from reproducing faience could potentially be used to improve modern ceramic technology. Additionally, faience could have direct uses as a water-resistant, antimicrobial material.

%Future work may include varying the particle size and distribution of the powders in recreating Egyptian faience to understand the exact processing conditions. Recent studies by Mimi Leveque suggest that adding gum-Arabic will aid in removing the faience from molds (Nicholson in Friedman's book 51).
%
%The next step in analyzing Ancient faience may be to separate the glaze from the body and perform XRD to determine the individual crystal structures.
%
%Compile a map of provenance for the different sand compositions of the Egyptian faience, building on Kaczmarcsyk and Hedges exploration of different regional techniques (Kaczmarcsyk and Hedges 5).
%
%Differences that arise from size of the faience, the difference between beads and shabti figurines, has also largely gone unstudied (Kaczmarcsyk and Hedges 9).