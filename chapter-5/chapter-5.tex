\chapter{Modern Applications}

%%%%%%%%%%%%%%%%%%%%%%%%%%%%%%%%%%%%%%%%%%%%%%%%%%%%%%%%%%%%%%%

Even though faience is an ancient material, it has certain properties not commonly found in modern materials. Its notable properties include self-glazing, low firing temperature, and antimicrobial attributes. The applications of these properties in modern industry are explored in Sections 5.1 through 5.3. The economic and environmental considerations associated with faience technology are described in Section 5.4.

%%%%%%%%%%%%%%%%%%%%%%%%%%%%%%%%%%%%%%%%%%%%%%%%%%%%%%%%%%%%%%%

\section{Applications of Self-Glazing}

Modern ceramics are glazed using a method similar to the application method of glazing associated with faience production. As in faience, this applied glaze can bear drying or firing marks and is prone to runs or drips. These marks can be prevented with the self-glazing methods of faience, of which cementation is better suited as it does not require a granular body unlike efflorescence. In addition, the production process would be simplified as glazing would be done in conjunction with firing.

A method similar to cementation is salt glazing (according to Dr.\ Williamson~\cite{vandiver83}), which was widely used before environmental clean air restrictions led to its demise~\cite{dodd94}. Glazing by cementation may be used in place of salt glazing to produce similar results but without the associated air pollution and with a simpler firing process.

Another use of self-glazing could be to simulate enamels and glazes made using frit, which is similar in composition to faience glaze. There are many modern uses for frit. For example, frits of primarily silica, diboron trioxide, and soda have been used as enamels on steel pipes. However, creating frit is a complicated process that involves fusing the components, quenching them to form glass, and then granulating the glass. The simpler cementation glazing method could be used to produce similar enamels and glazes as with frit.

%%%%%%%%%%%%%%%%%%%%%%%%%%%%%%%%%%%%%%%%%%%%%%%%%%%%%%%%%%%%%%%

\section{Applications of Low Firing Temperature}

Faience has a lower firing temperature ($<$1000$^\circ$~C) than most modern ceramics. Thus, faience could be used in place of modern ceramics when materials sensitive to heat are used.

Current packagings for semiconductor chips are made from ceramics or glass-ceramics. The high heat of the production process necessitates that the semiconductor chips be attached after the ceramic has been made~\cite{tummala91,ivf.se}. Presently, the chips are attached using solder. Solder is harmful to the environment (often lead-based) and less conductive than copper wiring, which restricts operating speeds. Using faience as the package allows the semiconductor chip to be placed prior to firing, eliminating the use of solder.

Additionally, advances have been made in the field of organic semiconductors. Organic semiconductors are of interest due to the low cost of their manufacture---they could theoretically be manufactured using simple inkjet printer techniques~\cite{chiang77}. However, they are very sensitive to heat, which makes them difficult to package using modern ceramics or glass-ceramics. The low heat of faience manufacturing could potentially be used with organic semiconductors.

%%%%%%%%%%%%%%%%%%%%%%%%%%%%%%%%%%%%%%%%%%%%%%%%%%%%%%%%%%%%%%%

\section{Applications of Antimicrobial Properties}

There are many antimicrobial surfaces approved by the EPA. These surfaces use copper or silver alloys, which have certain antimicrobial properties~\cite{epa,coppertouch}. However, no ceramic material has been approved. For aesthetic and practical purposes (e.g.\ tiles), it would be beneficial to have an antimicrobial ceramic material. The obvious choice is faience, which has some antimicrobial properties due to the copper oxide colorant in the glaze. These antimicrobial properties can be improved with cementation or application glazing, which result in thick glazes high in copper oxide.

Faience can be made into tiles for use in such places as hospitals and kitchens. Faience tiles would not only prevent the spread of disease, but also serve a decorative function. Also, the faience glaze may be applied to metals to further improve their antimicrobial properties.

%%%%%%%%%%%%%%%%%%%%%%%%%%%%%%%%%%%%%%%%%%%%%%%%%%%%%%%%%%%%%%%

\section{Economic and Environmental Considerations}

%The coating process of faience is governed by the process of efflorescence, from French meaning ``to flower out''. Efflorescence is a natural occurrence when soluble salts in a ceramic dry to the surface. It is most commonly seen as a white frost on bricks or concrete. As the soluble salt dries, it evaporates out of the ceramic body, leaving a salt powder on the surface. Egyptian faience makes uses of this property of copper salts evaporating as a self-glazing technique.

The use of faience in modern industry would be beneficial both economically and environmentally. Faience production requires fewer steps than traditional ceramics. Reducing steps in processing ceramics leads to an increase in efficiency. This reduction may cause a reduction in energy used creating a lower impact on the environment. Additionally, the low firing temperature of faience also lends to a reduction in energy use.
%Some of the methods used in efflorescence could be applied to modern ceramic technology. Fluxes could be used create gradients to drive alkalis through the ceramic body possibly for composite technology.

Environmentally, faience production uses little energy, which would have been necessary in Egypt, a country devoid of ample fuel. Faience can be used in place of more environmentally damaging processes, such as salt glazing. Additionally, faience is nontoxic---ancient Egyptians would even ingest faience as medicine~\cite{bittle11}.
 
%Fabricating faience coatings may also be a viable alternative to some high tech
%wear or corrosion resistant coatings. It may be similar to the CMAS techniques
%of covering columnar structures with a hard top layer that fills in-between the
%columns. The replicated faience has a copper oxide content of about 3 to 10 weight
%percent. By increasing the copper oxide, to increase efflorescence, improved antimicrobial protection may be achieved.

%If the ceramic has similar anti-microbial protection, while using
%less copper, the cost would come down. Faience is comprised of mostly quartz
%which is easily acquired at low cost. The benefit of Tite's method of recreating faience, is that it can be done by hand, but the properties and thickness of the coating leave room for improvements. The copper oxides do not effloresce to the surface as much as the authentic Egyptian faience. Lost to time are ancient processing techniques that may be useful for current industry.

%Certain metals and alloys, like copper, silver and bronze have been approved as
%alloys for antimicrobial purposes in public health products (EPA 1). No ceramic
%based material has been approved. Copper salts may bind to fungal enzymes and
%causes K+ ions to release [26]. This technique may not be better than pure copper alloys for their antimicrobial properties, but the ceramic may have other sought-after characteristics. For instance, the ceramic could be made into tile flooring, whereas metallic flooring is relatively unheard of since it is slippery, expensive and not aesthetically appealing. The goal of creating an anti-microbial ceramic tile is within reach.

%The reason for creating faience is unclear. The materials used to create faience were readily available and may have happened by mistake. Faience is relatively easy to recycle and recreate new lower quality faience. It may also be possible to harvest the copper through leaching in aqueous processing. Since Egyptian faience doesn't require high quality materials, recycled copper and copper alloys maybe applicable for use.

%In the New Kingdom, faience may have been used to simulate lapis lazuli. Faience was lightweight enough to wear around the extremities whereas turquoise, lapis lazuli, malachite and azurite may have been harder to carve and heavier on the skin.