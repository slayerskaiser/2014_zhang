% Battery State of Charge estimation is one of the key procedures in battery management systems.
% Typically, the measured voltage and current of a battery are used to estimate the state of charge.
% In battery management systems, the state of charge (SOC), obtained from voltage and current measurements, is a figure of merit.
In battery management systems, the main figure of merit is the battery's SOC, typically obtained from voltage and current measurements.
Present estimation methods use simplified battery models that do not fully capture the electrical characteristics of the battery, which are useful for system design.
This thesis studied SOC estimation for a lithium-ion battery using a nonlinear, electrical-circuit battery model that better describes the electrical characteristics of the battery.
% However, these measurements are prone to noise and the battery does not behave linearly.
% The estimation of State of Charge is challenging due to the nonlinear behavior of the battery, measurement noise, and the trade-off between accuracy and computational complexity in selecting the sample rate.
% Additionally, for the purposes of system design, an electrical-circuit battery model is useful, which presents additional filtering difficulties when compared to presently-used analytical models.
% This means that the prediction method has to be able to handle the effect of nonlinear noise on the system.
% This thesis investigated noise removal and state estimation for a nonlinear model of a battery cell.
% This thesis investigated the performance of various nonlinear filters for estimating the State of Charge using an electrical-circuit battery model.
The extended Kalman filter, unscented Kalman filter, third-order and fifth-order cubature Kalman filter, and the statistically linearized filter were tested on their ability to estimate the SOC through numerical simulation.
Their performances were compared based on their root-mean-square error over one hundred Monte Carlo runs as well as the time they took to complete those runs.
% The accuracy and speed of the estimation were investigated for various nonlinear filters through numerical simulation.
% The effect of the choice of filtering method and sampling rate was determined for various amounts of additive white Gaussian noise.
The results show that the extended Kalman filter is a good choice for estimating the SOC of a lithium-ion battery.

\vfill

\noindent Keywords: nonlinear filtering; battery management; state of charge; battery modeling
