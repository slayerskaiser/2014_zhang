\section{Concluding Remarks}
We have quantified the impact of network synchronization on both detection and estimation problems over sensor networks. We have also performed a cost benefit analysis of performing synchronization and clock stabilization on network nodes. Our results can be used in a twofold manner--to design the network to support these applications, and to design the applications around the constraints imposed by the network that supports them. 

\indent Our results show that network synchronization may not be necessary for certain estimation and tracking problems. In these problems, the randomness of the sensor locations combined with randomness of sense times can be used to obtain global stability and convergence of the estimation problem. The smoothing properties of estimators, including batch least squares and Kalman filters can be used to naturally reduce overall tracking error.

\indent For detection problems, our analysis shows that synchronization becomes important when the events detected are of short duration and frequent. It is also needed when the parameters of an event, such as the location and intensity of an explosion need to be estimated in real-time. Our analysis of the detection combined with the cost-benefit analysis of synchronization provides immediate solutions to setting detection thresholds and to the choice of synchronization scheme and frequency, along with clock accuracy needed. 