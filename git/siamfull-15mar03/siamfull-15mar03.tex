% Document Type: LaTeX
% Master File: siamfull-15mar03.tex

% siam.tex - CDS Panel report in SIAM book format
% RMM, 6 April 2002
%
% This is the report for the CDS Panel on Future Directions in Control and
% Dynamical Systems, typeset in SIAM book format.
%

% SIAM book style; standard header
\documentclass{newsiambook}
\usepackage{epsfig}
\usepackage{graphicx}
%\usepackage{makeidx,showidx}
\usepackage{index}		% use this version to allow short indexing
\usepackage{multicol}

% Additional packages specific to this document
\usepackage[dvips]{color}
\usepackage{framed}		% for producing vignettes
\usepackage{url}		% define command for formatting URLs

% Some local definitions and inclusions
\def\siam{true}			% set siam flag
%% \input{macros.tex}		% local macros
%%
%% Vignettes
%%
%% This macro defines the style used for vignettes throughout the report.
%% The current format is to use a shaded box (from the 'framed' style with a
%% font change to set it off from the text
%%

% Define the color to be used for shading
% \definecolor{shadecolor}{rgb}{0.8,0.8,1} % orig color
\definecolor{shadecolor}{rgb}{0.90,0.90,1}   % lighter

\newcommand{\vignette}[2]{
\bgroup\fontencoding{OT1}\fontfamily{cmss}\fontseries{m}\selectfont
\index{vignettes}%
\begin{shaded}
\noindent{\bf Vignette: #1}\par
\parskip 1ex
#2\par
\end{shaded}
\egroup}

%%
%% Short macro to reset figure spacing (not currently used) + length to
%% use for figures
%%
\def\tightfigs{
  \renewcommand{\textfraction}{0.0}
  \renewcommand{\floatpagefraction}{0.9}
}
\newlength\figwidth
\setlength\figwidth\textwidth

%%
%% Define the term that we will use through the report to refer to CDS
%%
%% These are here for historical reasons.  Can probably replace everything
%% with the defined text at some point.
%%
\def\CDS{Control}
\def\cds{control}

%% 
%% Define a labelled paragraph
%%
%% This is used to put a label at the beginning of a paragraph and set it off
%% from the text a bit.  Really a hack around the default paragraph styles,
%% which I don't like
%%
\newcommand{\partitle}[1]{\medskip\noindent{\it #1}.}

%%
%% Note commands
%%
%% These are various commands that were used to include notes in drafts of the
%% report.  In the final version, they should all be gone.
%%
\newcommand{\editor}[1]{}

\makeindex			% enable indexing
\shortindexingon		% allow short indexing

%%
%% Document start
%%

\begin{document}

\frontmatter
%% \input{titlepage.tex}		% Title page
% Master File: cdspanel.tex
\begin{titlepage}
\center
\LARGE 
Control in an Information Rich World

\vspace{5ex}
\Large
Report of the Panel on Future Directions in \\
Control, Dynamics, and Systems 
\vspace{5ex}
\normalsize

\vfill
{\large PREPRINT}
\vskip 1ex
{Not to be distributed without permission of the editor}
\vskip 0.5in
\centerline{\large \today} 
\vfill

\centerline
  {\copyright 2002, Richard M. Murray (editor)}
\centerline{All rights reserved}

\vfill
\small
This manuscript is for review purposes only and may not be reproduced,
in whole or in part, in any form, without written consent from the
editor.
\vfill

\clearpage
\thispagestyle{empty}
\end{titlepage}%% \begin{abstract}

New hardware and technologies enable low-power low-cost distributed
sensing systems. To realize certain applications such as real-time event
detection, target tracking, and system monitoring, time synchronization is
essential. However, time synchronization both consumes limited battery power on the network nodes and can clog the bandwidth of the network. The choice of a time synchronization mechanism will depend on the
application's requirement of timing accuracy as well as its energy budget. We derive the effect of time synchronization accuracy upon real time estimation and detection problems. We also quantify the costs and benefits of time synchronization algorithms. The intuitive assumption that using higher stability clocks will
automatically improve duty cycling performance, and thus decrease power
consumption, does not always hold true. In this article, we present the
link between clock stability, impact on duty cycling, and the possible bandwidth
savings that can be achieved by using temperature compensated clocks or clock drift
estimation techniques. This paper formalizes this relationship based on an analytical framework using representative applications, namely, event detection and estimation. The analysis shows the impact of timing errors for different event durations, target
speeds, number of sensors, and sampling frequencies. The analysis
framework can also be used to estimate the maximum synchronization error
each application can sustain while still achieving the desired Quality of Information (QoI).  
\end{abstract}
		% Abstract
\begin{titlepage}
\centerline{\large \begin{tabular}{c}
{\LARGE Control in an Information Rich World} \\[3ex]
Report of the Panel on Future Directions in \\
Control, Dynamics, and Systems \\[3ex]
{\normalsize 
%%  DRAFT: v2.8, 
  30 June 2002
}
\end{tabular}}

% Master File: cdspanel.tex
\section*{Abstract}

The field of \emph{\cds} provides the principles and methods used to design
engineering systems that maintain desirable performance by
automatically adapting to changes in the environment.  Over the last forty
years the field has seen huge advances, leveraging technology improvements in
sensing and computation with breakthroughs in the underlying principles and
mathematics.  Control systems now play critical roles in
many fields, including manufacturing, electronics, communications,
transportation, computers and networks, and many military systems.

\index{computation!ubiquitous computing}
As we begin the 21st Century, the opportunities to apply \cds{} principles and
methods are exploding.  Computation, communication and sensing are becoming
increasingly inexpensive and ubiquitous, with more and more devices including
embedded processors, sensors, and networking hardware.  This will make
possible the development of machines with a degree of intelligence and
reactivity that will influence nearly every aspect of life on this planet,
including not just the products available, but the very environment in which
we live.

New developments in this increasingly information rich world will require a
significant expansion of the basic tool sets of \cds{}.  The complexity of the
\cds{} ideas involved in the operation of the Internet, semi-autonomous
command and control systems, and enterprise-wide supply chain management, for
example, are on the 
boundary of what can be done with available methods.  Future applications in
aerospace and transportation, information and networks, robotics and
intelligent machines, biology and medicine, and materials and processing will
create systems that are well beyond our current levels of complexity, and new
research is required to enable such developments.

The purpose of this report is to spell out some of the prospects for \cds{} in
the current and future technological environment, to describe the role the
field will play in military, commercial, and scientific applications
over the next decade, and to recommend actions required to enable new
breakthroughs in engineering and technology through application of \cds{}
research.

\end{titlepage}

\ifx\siam\undefined{}\else\cleardoublepage\fi
\index{CDS Panel!membership}%
\def\pmember#1#2{\begin{tabular}{c} #1 \\ #2 \end{tabular}}

\begin{center}
  \section*{\ \hfill Panel Membership \hfill\ }
  \vspace{2ex}

\centerline{\begin{tabular}{cc}
    \multicolumn{2}{c}{
      \begin{tabular}{c}
        Richard M. Murray (chair) \\ 
  	California Institute of Technology
      \end{tabular}
    } \\[4ex]
    \pmember{Karl J. {\AA}str\"om}{Lund Institute of Technology} &
    \pmember{Pramod P. Khargonekar}{University of Florida} \\[4ex]
    \pmember{Stephen P. Boyd}{Stanford University} &
    \pmember{P. R. Kumar}{University of Illinois } \\[4ex]
    \pmember{Siva S. Banda}{Air Force Research Laboratory} &
    \pmember{P. S. Krishnaprasad}{University of Maryland } \\[4ex]
    \pmember{Roger W. Brockett}{Harvard University} & 
    \pmember{Greg J. McRae}
      {\makebox[0.49\textwidth]{Massachusetts Institute of 
       Technology}} \\[4ex] 
    \pmember{John A. Burns}{Virginia Tech} &
    \pmember{Jerrold E. Marsden}{California Institute of Technology} \\[4ex]
    \pmember{Munzer A. Dahleh}
      {\makebox[0.49\textwidth]{Massachusetts Institute of 
       Technology}} &
    \pmember{George Meyer}{NASA Ames Research Center} \\[4ex]
    \pmember{John C. Doyle}{California Institute of Technology} &
    \pmember{William F. Powers}{Ford Motor Company} \\[4ex]
    \pmember{John Guckenheimer}{Cornell University} &
    \pmember{Gunter Stein}{Honeywell International} \\[4ex] 
    \pmember{Charles J. Holland}{Department of Defense} &
    \pmember{Pravin Varaiya}{University of California, Berkeley}
  \end{tabular}}

  \vfill

  \subsection*{\ \hfill Additional Contributors \hfill\ } 
  Richard Albanese, Jim Batterson, Richard Braatz, Dennis Bernstein, Joel
  Burdick, Raffaello D'Andrea, Michael Dickinson, Frank Doyle, Martha
  Gallivan, Jonathan How, Marc Jacobs, Jared Leadbetter, Jesse Leitner, Steven
  Low, Hideo Mabuchi, Dianne Newman, Shankar Sastry, John Seinfeld, Eduardo
  Sontag, Anna Stefanopoulou, Allen Tannenbaum, Claire Tomlin, Kevin Wise

  \vfill\vfill

\end{center}
\tableofcontents		% Table of Contents
%% \input{preface.tex}		% Preface
% Master File: cdspanel.tex

\ifx\siam\undefined
\cleardoublepage\addcontentsline{toc}{section}{Preface}
\section*{Preface}
\else
\begin{thepreface}
\fi

\index{Air Force Office of Scientific Research (AFOSR)}
This report documents the findings and recommendations of the Panel on Future
Directions in Control, Dynamics, and Systems.  This committee was formed in
April 2000 under initial sponsorship of the Air Force Office of Scientific
Research (AFOSR) to provide a renewed vision of future challenges and
opportunities in the field, along with recommendations to government agencies,
universities, and research organizations to ensure continued progress
in areas of importance to the industrial and defense base.  The intent of this
report is to raise the overall visibility of research in \cds{}, highlight
its importance in applications of national interest, and indicate some of the
key trends that are important for continued vitality of the field.

\index{CDS Panel!formation}%
\index{CDS Panel!membership}
The Panel was chaired by Professor Richard Murray (Caltech) and was formed
with the help of an organizing committee consisting of Professor Roger
Brockett (Harvard), Professor John Burns (VPI), Professor John Doyle (Caltech)
and Dr.\ Gunter Stein (Honeywell).  The remaining Panel members are Karl
{\AA}str\"om (Lund Institute of Technology), Siva Banda (Air Force Research
Lab), Stephen Boyd (Stanford), Munzer Dahleh (MIT), John Guckenheimer
(Cornell), Charles Holland (DDR\&E), Pramod Khargonekar (University of
Florida), P. R. Kumar (University of Illinois), P. S. Krishnaprasad
(University of Maryland), Greg McRae (MIT), Jerrold Marsden (Caltech), George
Meyer (NASA), William Powers (Ford), and Pravin Varaiya (UC Berkeley).  A
writing subcommittee consisting of Karl {\AA}str\"om, Stephen Boyd, Roger
Brockett, John Doyle, Richard Murray and Gunter Stein helped coordinate the
generation of the report.

\index{Air Force Office of Scientific Research (AFOSR)}
\index{University of Maryland}
\index{Defense Advanced Research Projects Agency (DARPA)}%
The Panel held a meeting on 16-17 July 2000 at the University of Maryland,
College Park to discuss the state of the field and its future opportunities.
The meeting was attended by members of the Panel and invited participants from
the academia, industry, and government.  Additional meetings and discussions
were held over the next 15 months, including presentations at DARPA and AFOSR
sponsored workshops, meetings with government program managers, and writing
committee meetings.  The results of these meetings, combined with discussions
among Panel members and within the community at workshops and conferences,
form the main basis for the findings and recommendations of this Panel.

\index{CDS Panel!web site}%
A web site has been established to provide a central repository for 
materials generated by the Panel:
\begin{center}
  \verb|http://www.cds.caltech.edu/~murray/cdspanel/|
\end{center}
Copies of this report, links to other sources of information, and presentation
materials from the Panel workshop and other meetings can be found there.

\index{Fleming, W. H.}%
Several similar reports and papers highlighting future directions in \cds{}
came to the Panel's attention during the development of this report.  Many
members of the Panel and participants in the June 2000 workshop were involved
in the generation of the 1988 Fleming report~\cite{Fle88} and a 1987 {\em IEEE
Transactions on Automatic Control} article~\cite{Lev+87-tac}, both of which
provided a roadmap for many of the activities of the last decade and continue
to be relevant.  More recently, the European Commission sponsored a workshop
on future control systems~\cite{EC00} and several other more focused workshops
have been held over the last several
years~\cite{nsfcss99,CP00-workshop,nsf00-hsctrl,nsf00-dddas}.  Several recent
papers and 
reports highlighted successes_{control!successes of} of
control~\cite{NS99-4decades} 
and new vistas in
control~\cite{Bro00-mathunl,Kum01-csm}.  The Panel also made extensive use of a
recent NSF/CSS report on future directions in control engineering
education~\cite{nsfcss99}, which provided a partial basis for
Chapter~\ref{outreach.sec} of the present report.
\index{National Science Foundation (NSF)}

\index{terrorism, fight against}%
\index{homeland defense|see{terrorism}}
\index{biological and chemical weapons}%
The bulk of this report was written before the tragic events of September 11,
2001, but \cds{} will clearly play a major role in the world's effort to
combat terrorism.  From new methods for command and control of
unmanned vehicles, to robust networks linking businesses, transportation
systems, and energy infrastructure, to improved techniques for sensing
and detection of biological and chemical agents, the techniques and insights
from \cds{} will enable new methods for protecting human life and safeguarding
our society.

\index{Air Force Office of Scientific Research (AFOSR)}
The Panel would like to thank the \cds{} community for its support of this
report and the many contributions, comments, and discussions that help form
the context and content for the report.  We are particularly indebted to Dr.\
Marc Q. Jacobs for his initiative in the formation of the Panel and for his
support of the project through AFOSR.

\begin{flushleft}
Richard M. Murray \hfill Pasadena, June 2002
\end{flushleft}

\ifx\siam\undefined
\relax
\else
\end{thepreface}
\fi


\mainmatter
%% \include{summary}		% Executive Summary
\clearpage
% Master File: cdspanel.tex
\cleardoublepage
\chapter{Executive Summary}

Rapid advances in computing, communications, and sensing technology offer
unprecedented opportunities for the field of \cds{} to expand its
contributions to the economic and defense needs of the nation. This report
presents the findings and recommendations of a panel of experts chartered
to examine these opportunities. We present an overview of the field, review
its successes and impact, and describe the new challenges ahead. We do not
attempt to cover the entire field. Rather, we focus on those areas that are
undergoing the most rapid change and that require new approaches to
meet the challenges and opportunities that face the community.

\subsection*{Overview of \CDS{}}

\index{actuation}%
\index{control!system}
\index{industrial revolution}
\index{feedback}%
\index{Watt governor|see{centrifugal governor}}%
\index{centrifugal governor}%
\index{flyball governor|see{centrifugal governor}}%
\index{steam engine}%
Control as defined in this report refers to the use of algorithms and feedback
in engineered systems.
At its simplest, a control system is a device in which a sensed_{sensing}
quantity is used to modify the behavior of a system through computation and
^{actuation}.  Control systems engineering traces its roots to the industrial
revolution, to devices such as the centrifugal governor, shown in
Figure~\ref{watt.fig}.
\begin{figure}
  \centerline{
  \begin{tabular}{cc}
    \epsfig{figure=wattgov2.eps,height=2in} \qquad & \qquad
    \epsfig{figure=wattengine.eps,height=2in} \\
    (a) \qquad & \qquad (b)
  \end{tabular}
  }
  \caption{The centrifugal governor (a), developed in the 1780s, was an
  enabler of the successful Watt steam engine (b), which fueled the industrial
  revolution.
  Figures courtesy Richard Adamek (copyright 1999) and Cambridge University.
  }
  \label{watt.fig}
  \index{steam engine}
\end{figure}
This device used a flyball mechanism to sense the rotational speed of a steam
turbine and adjust the flow of steam into the machine using a series of
linkages. By thus regulating the turbine's speed, it provided the
safe_{safety}, 
reliable_{reliability}, consistent operation that was required to enable the rapid spread of
steam-powered factories.
\index{control!as enabling technology}

\index{disk drives}%
\index{control!as enabling technology}%
\index{electrical power}
\index{power grid, electrical}
\index{autopilot}%
\CDS{} played an essential part in the development of technologies such as
power, communications, transportation, and manufacturing.  Examples
include autopilots in military and commercial aircraft
(Figure~\ref{ctrlapps.fig}a), regulation and control of the electrical power
grid, and high 
accuracy positioning of read/write heads in disk drives
(Figure~\ref{ctrlapps.fig}b).
Feedback is an enabling technology 
in a variety of application areas and has been reinvented and patented many
times in 
different contexts.

\index{uncertainty management}%
\index{disturbances}
\index{dynamics}
A modern view of \cds{} sees feedback as a tool for uncertainty
management.  By measuring the operation of a system, comparing it to a
reference, and adjusting available control variables, we can cause the system
to respond properly even if its dynamic behavior is not exactly known or if
external disturbances tend to cause it to respond incorrectly.  This is an
{essential feature} in engineering systems since they must
operate reliably and efficiently under a variety of conditions.  
It is precisely
this aspect of \cds{} as a means of ensuring robustness 
to uncertainty that explains why feedback control systems are all around us in
the modern technological world. They are in our homes, cars and consumer
electronics, in our factories and communications systems, and in our
transportation, military and space systems.

\index{economic systems}%
\index{decision making!systems}
\index{uncertainty!component or parameter variation}
\index{biology}%
\index{medicine}%
The use of \cds{} is extremely broad and encompasses a number of different
applications.  These include control of electromechanical systems,
where computer-controlled actuators and sensors regulate the behavior of the
system; control of electronic systems, where feedback is used to compensate
for component or parameter variations and provide reliable, repeatable performance; and
control of information and decision systems, where limited resources are
dynamically allocated based on estimates of future needs.  \CDS{} principles
can also be found in areas such as biology, medicine, and economics, where
feedback mechanisms are ever present.  Increasingly, \cds{} is also a mission
critical function in engineering systems: the systems will fail if the control
system does not work.
\index{control!principles}%

\index{economics}
\index{operations research}
\index{mathematics}
\index{physics}
\index{biology}
Contributions to the field of \cds{} come from many disciplines,
including pure and ^{applied mathematics}; aerospace, chemical, mechanical, and
electrical engineering; operations research and economics; and the physical
and biological sciences.  The interaction with these different fields is an
important part of the history and strength of the field.

%\subsection*{Why Does It Matter?}
\subsection*{Successes and Impact}

\index{control!successes of|(}%
\index{control!as enabling technology}
Over the past 40 years, the advent of analog and digital electronics has
allowed \cds{} technology to spread far beyond its initial applications,
and has made it an enabling technology in many applications.  Visible
successes from past investment in \cds{} include:
\begin{itemize}
  \index{aerospace systems}%
  \index{aircraft}%
  \index{tracking}
  \index{guidance}%
  \index{satellites}%
  \item Guidance and control systems for aerospace vehicles, including
  commercial aircraft, guided ^{missiles}, advanced fighter aircraft, launch
  vehicles, and satellites.  These control systems provide stability and
  tracking in the presence of 
  large environmental and system uncertainties.

%% Moved this figure down here to force it onto separate page from governor
\begin{figure}
  \centerline{
  \begin{tabular}{ccc}
    \epsfig{figure=boeing777.eps,height=1.85in} &\hspace*{-2ex}&
    \epsfig{figure=barracuda36es2.eps,height=1.85in} \\
    (a) &\hspace*{-2ex}& (b)
  \end{tabular}
  }
  \caption{Applications of \cds{}: (a) the Boeing 777 fly-by-wire aircraft and
  (b) the Seagate Barracuda 36ES2 disk drive. 
  Photographs courtesy of the ^{Boeing Company} and ^{Seagate Technology}.
  }
  \index{fly-by-wire}
  \index{disk drives}
  \label{ctrlapps.fig}
\end{figure}

  \index{manufacturing}%
  \index{automobiles}%
  \item Control systems in the manufacturing industries, from
  automotive to integrated circuits.  Computer controlled machines provide
  the precise positioning and assembly required for high quality, high yield
  fabrication of components and products.

  \index{process control}%
  \item Industrial process control systems, particularly in the
  hydrocarbon and chemical processing industries.  These maintain high product
  quality by monitoring thousands of sensor signals and making corresponding
  adjustments to 
  hundreds of valves, heaters, pumps, and other actuators.

  \index{communications!systems}%
  \index{cellular phones}
  \index{Internet}%
  \index{telephone system}%
  \item Control of communications systems, including the telephone system,
  cellular phones, 
  and the Internet.  Control systems regulate the signal power levels in
  transmitters and repeaters, manage packet buffers in network routing
  equipment, and provide adaptive noise cancellation to respond to varying
  transmission line characteristics.
\end{itemize}
These applications have had an enormous impact on the productivity of modern
society.

\index{mathematics!contributions to}
\index{complex systems}
In addition to its impact on engineering applications, \cds{} has also
made significant intellectual contributions. Control theorists and engineers
have made rigorous use of and contributions to mathematics, motivated by the
need to develop provably correct techniques for design of feedback
systems. They have been consistent advocates of the ``systems perspective,''
\index{systems perspective}%
and have developed reliable techniques for modeling, analysis, design, and
testing that enable design and implementation of the wide variety of
very complex engineering systems in 
use today. Moreover, the \cds{} community has been a major source and
training ground for people who embrace this systems perspective and who wish
to master
the substantial set of knowledge and skills it entails.
\index{control!successes of|)}%

%\subsection*{Control Will Be Even More Important in the Future}
\subsection*{Future Opportunities and Challenges}

\index{software!systems}%
\index{computation!ubiquitous computing}
\index{data rich systems}
\index{distributed control}
\index{ubiquitous computing}
As we look forward, the opportunities for new applications that will build on
advances in \cds{} expand dramatically.  The advent of ubiquitous, distributed
computation, communication, and sensing systems has begun to create an
environment in which we have access to enormous amounts of data and the
ability to process and communicate that data in ways that were unimagined 20
years ago. This will have a profound effect on military, commercial and
scientific applications, especially as software
systems begin to interact with physical systems in more and more
integrated ways. Figure~\ref{networks.fig} illustrates two systems where
these trends are already evident.
\begin{figure}
  \centerline{
  \begin{tabular}{ccc}
    \epsfig{figure=capplants.eps,width=0.35\figwidth} &\qquad&
    \epsfig{figure=nsfnet.eps,width=0.575\figwidth} \\
    (a) && (b)
  \end{tabular}
  }
  \caption{Modern networked systems: (a) the California power grid and (b)
    the NSFNET Internet backbone.
  Figures courtesy of the state of California and the National Center for
  Supercomputer Applications (NCSA) and Robert Patterson.
  }
  \index{NSFNET}
  \index{Internet}
  \index{power grid, electrical}
  \label{networks.fig}
\end{figure}
\CDS{} will be an increasingly essential element of building such
interconnected systems, 
providing high performance, high confidence, and reconfigurable operation in
the presence of uncertainties.
\index{high confidence systems}

\index{reliability}
In all of these areas, a common feature is that system level requirements far
exceed the achievable reliability of individual components.  This is precisely
where \cds{} (in its most general sense) plays a central role, since it allows
the system to ensure that it is achieving its goal through correction of its
actions based on sensing its current state.  The challenge to the field is to
go from the traditional view of control systems as a single process with a
single controller, to recognizing control systems as a heterogeneous
collection of physical and information systems, with intricate
interconnections and interactions.

\index{information-based systems}%
\index{autonomous systems}%
\index{enterprise level systems}%
\index{decision making!higher level}
\index{unmanned vehicles}
\index{resource allocation}
In addition to inexpensive and pervasive computation, communication, and
sensing---and the corresponding increased role of information-based
systems---an important trend in \cds{} is the move from low-level control to
higher levels of decision making. This includes such advances as increased
autonomy in flight systems (all the way to complete unmanned operation),
and integration of local feedback loops into enterprise-wide scheduling and
resource allocation systems. Extending the benefits of \cds{} to these
non-traditional 
systems offers enormous opportunities in improved efficiency, productivity,
safety, and reliability.

%\subsection*{... But It Won't Be Easy}

\index{defense systems}
\index{terrorism, fight against}
\index{sensing!sensor webs}
\index{unmanned vehicles}
\index{decision making!systems}
\index{adversarial environments}
\index{autonomous systems}
\index{networks}
\index{command and control}
\index{control!as enabling technology}
\index{asymmetric threats}
\index{communications!systems}
Control is a critical technology in defense systems and is
increasingly 
important in the fight against terrorism and asymmetric threats.  Control
allows the operation of autonomous and semi-autonomous unmanned systems for
difficult and dangerous missions, as well as sophisticated command and control
systems that enable robust, reconfigurable_{reconfigurable systems} decision
making systems.  The use 
of control in microsystems and senosr webs will improve our ability to detect
threats before they cause damage.  And new uses of feedback in communications
systems will provide reliable, flexible, and secure networks for operation in
dynamic, uncertain, and adversarial environments.

\index{control!challenges}%
In order to realize the potential of \cds{} applied to these emerging
applications, new methods and approaches must be developed.  Among the
challenges currently facing the field, a few examples provide insight into the
difficulties ahead:
\begin{itemize}
  \index{symbolic dynamics}
  \index{dynamics}
  \index{decision making!logic}
  \index{reasoning, higher level}
  \item {\it Control of systems with both symbolic and continuous
  dynamics.}  Next generation systems will combine logical
  operations (such as symbolic reasoning and decision making) with continuous
  quantities (such as voltages, positions, and concentrations).  
  The current theory is not well-tuned for dealing with such systems,
  especially as we scale to very large systems.

  \index{distributed control}
  \index{asynchronous systems}
  \index{communications!role of, in control}
  \index{networks}
  \index{packet-based systems}
  \item {\it Control in distributed, asynchronous, networked
  environments.} Control distributed across multiple computational units,
  interconnected through packet-based communications, will require new
  formalisms for ensuring stability, performance and robustness.  
   This is especially true in applications where one cannot ignore
   computational and communications constraints in performing control
   operations.

  \index{C4ISR}%
  \index{airspace management}
  \index{enterprise level systems}
  \index{autonomous systems}
  \index{autonomy|see{autonomous systems}}
  \index{decision making!higher level}
  \index{decision making!systems}
  \item {\it High level coordination and autonomy.}  Increasingly, feedback is
  being designed into enterprise-wide decision systems, including supply chain
  management and ^{logistics}, airspace management and air traffic control, and
  C4ISR systems.  The advances of the last few decades in analysis and design
  of robust control systems must be extended to these higher level
  decision making systems if they are to perform reliably in realistic
  settings.

  \index{automatic synthesis}
  \index{verification and validation}
  \item {\it Automatic synthesis of control algorithms, with integrated
  verification and validation.} Future engineering systems will require the
  ability to rapidly design, redesign and implement control software.
   Researchers need to develop much more powerful design tools that automate
   the entire control design process from model development to
   hardware-in-the-loop ^{simulation}, including system-level software
   verification and validation.

  \index{reliability}
  \item {\it Building very reliable systems from unreliable parts.}  Most
  large engineering systems must continue to operate even when individual
  components fail. Increasingly, this requires designs that allow the system
  to automatically reconfigure itself so that its performance degrades
  gradually rather than abruptly.
\end{itemize}
Each of these challenges will require many years of effort by the research
community to make the results
rigorous, practical, and widely 
available.  They will also require investments by funding agencies to ensure
that 
current progress is continued and that forthcoming technologies are realized
to their fullest.

%\subsection*{What Needs to Be Done}
\subsection*{Recommendations}

\index{CDS Panel!recommendations}%
\index{recommendations|(}
To address these challenges and deliver on the promise of the \cds{} field,
the Panel recommends that the following actions be undertaken:
\begin{enumerate}

  \index{computer science!interaction of control and}
  \index{communications!integration of control and}
  \index{distributed control}
  \index{computation!integration of control and}
  \index{networking!integration of control and}
  \item Substantially increase research aimed at the {\em integration} of
  \cds{}, computer 
  science, communications, and networking.  This includes principles, methods
  and 
  tools for modeling and control of high level, networked, distributed
  systems, and rigorous 
  techniques for reliable, embedded, real-time software.
  \index{real-time systems}

  \index{enterprise level systems}%
  \index{decision making!higher level}
  \index{resource allocation}
  \item Substantially increase research in \cds{} at higher levels of
  decision making, moving 
  toward enterprise level systems.  This includes work in dynamic resource
  allocation in the presence of uncertainty, learning and adaptation, and
  artificial intelligence for dynamic systems.
  \index{adaptation}%
  \index{learning}%
  \index{artificial intelligence}
  \index{uncertainty!in resource allocation}

  \index{new domains, control in}
  \index{nanotechnology}
  \index{quantum systems}
  \index{electromagnetics}
  \index{environmental science}
  \index{biology}
  \index{dual investigator funding}
  \item Explore high-risk, long-range applications of \cds{} to new domains
  such as 
  nanotechnology, quantum mechanics, electromagnetics, biology, and
  environmental science.  Dual investigator, interdisciplinary funding might be
  a particularly 
  useful mechanism in this context.

  \index{theory!support for}
  \index{mathematics!interaction with}
  \item Maintain support for theory and interaction with mathematics, broadly
  interpreted.  The strength of the field relies on its close contact with
  rigorous mathematics, and this will be increasingly important in the future.

  \index{education}%
  \index{outreach}
  \index{accessibility, broadening}
  \index{control!principles}%
  \item Invest in new approaches to education and outreach for the
  dissemination of control concepts and tools to non-traditional audiences.
  The community must do a better job of educating a broader range of
  scientists and 
  engineers on the principles of feedback and the use of control to alter the
  dynamics of systems and manage uncertainty.

\end{enumerate}
\index{recommendations|)}

\index{control!impact and payoffs}
The impact of \cds{} is one which will come through many applications, in
aerospace and transportation, information and networking, robotics and
intelligent machines, materials and processing, and biology and medicine.  It
will enable us to build more complex systems and to ensure that the systems we
build are reliable, efficient, and robust.  The Panel's recommendations are
founded on the diverse heritage of rigorous work in \cds{} and are key actions
to realize the opportunities of \cds{} in an information rich world.
%% \include{overview}		% Overview of the Field
\clearpage
% Master File: cdspanel.tex
\cleardoublepage
\chapter{Overview of the Field}

\CDS{} is a field with broad relevance to a number of engineering
applications.  Its impact on modern society is both profound and often poorly
understood.  In this chapter, we provide an overview of the field, illustrated
with examples and vignettes, and describe the new environment for \cds{}.

\section{What is \CDS{}?}

\index{control!definition}%
\index{fly-by-wire}
\index{feedback}
\index{high confidence systems}
\index{amplifiers}
The term ``control'' has many meanings and often varies between communities.
In this report, we define \cds{} to be the use of algorithms and feedback in
{engineered} systems.  Thus, \cds{} includes such examples as feedback
loops in electronic amplifiers, set point controllers in chemical and
materials processing, ``fly-by-wire'' systems on aircraft, and
even router ^{protocols} that control traffic flow on the Internet.  
Emerging applications include high confidence software systems, autonomous
vehicles and robots, battlefield management systems, and biologically
engineered systems.
At its core,
\cds{} is an {\em information} science, and includes the use of information
in both analog and digital representations.
\index{battlefield management}
\index{biological engineering}

\index{control!system}%
\index{actuation}%
\index{sensing}
\index{closed loop control}
\index{feedback!loop}%
\index{stability}
\index{disturbance rejection}
\index{component failures, robustness to}
A modern controller senses the operation of a system, compares that
against the desired behavior, computes corrective actions based on a model of
the system's response to external inputs, and actuates the system to effect
the desired change.  This basic {\em feedback loop} of sensing, computation,
and actuation is the central concept in \cds{}.  The key issues in designing
control logic are ensuring that the dynamics of the closed loop system are
stable (bounded disturbances give bounded errors) and that dynamics have the
desired behavior (good disturbance rejection, fast responsiveness to changes
in operating point, etc).  
\index{dynamics}
These properties are established using a variety of
modeling and analysis techniques that capture the essential physics of the
system and permit the exploration of possible behaviors in the presence of
uncertainty, noise, and component failures.

\index{actuation}
\index{computation}
A typical example of a modern control system is shown in
Figure~\ref{control.fig}.
\begin{figure}
  \centerline{\input ctrlsys.pst}
  \caption{Components of a modern control system. 
  } 
  \label{control.fig}
\end{figure}  
The basic elements of of sensing, computation, and actuation are clearly seen.
In modern control systems, computation is typically implemented on a digital
computer, requiring the use of analog-to-digital (A/D) and digital-to-analog
(D/A) converters.  Uncertainty enters the system through noise in
sensing and actuation subsystems, external disturbances that affect the
underlying system physics, and uncertain dynamics in the physical system
(parameter errors, unmodeled effects, etc).
\index{disturbances}
\index{uncertainty!component or parameter variation}
\index{uncertainty!unmodeled dynamics}

The basic feedback_{feedback!loop} loop of \cds{} is often combined with {\em
feedforward}_{feedfoward control} control, where a commanded actuator input is
computed to achieve a desired action based on a model of the system.  While
feedback operates in a ^{closed loop}, with actions based on the deviation
between measured and desired performance, feedforward operates in ^{open
loop}, with actions taken based on plans.  It is often advantageous to use
feedback with feedforward to achieve both high performance and
robustness.

\index{feedback!versus control}%
\index{atmospheric systems}
\index{feedback!in biological systems}
\index{global climate dynamics}
\index{ecological systems}%
\index{economic systems}%
It is important to note that while feedback is a central element of \cds{},
feedback as a phenomenon is ubiquitous in science and nature.  Homeostasis in
biological systems maintains thermal, chemical, and biological conditions
through feedback.  Global climate dynamics depend on the feedback interactions
between the atmosphere, oceans, land, and the sun.  Ecologies are filled with
examples of feedback, resulting in complex interactions between animal and
plant life.  The dynamics of economies are based on the feedback between
individuals and corporations through markets and the exchange of goods and
services. 
\index{dynamics!economic systems}

While ideas and tools from \cds{} can be applied to these systems, we
focus our attention in this report on the application of feedback to 
engineering systems.  We also limit ourselves to a small subset of the many
aspects of \cds{}, choosing to focus on those that are undergoing the most
change and are most in need of new ideas and techniques.

\subsection*{Control Theory}

\index{control!theory}%
Control {\em theory} refers to the mathematical framework used to analyze and
synthesize control systems.
Over the last 50 years, there has been careful attention by control theorists
to the issues of completeness and correctness.  This includes substantial
efforts by mathematicians and engineers to develop a solid foundation for
proving stability and robustness of feedback controlled systems, and the
development of computational tools that provide guaranteed performance in
the presence of uncertainty.  This ^{rigor} in approach is a hallmark of modern
\cds{} and is largely responsible for the success it has enjoyed
across a variety of disciplines.
\index{control!successes of}

\index{history|see{control, history of}}
\index{control!history of}%
\index{control!history of!control theory}%
It is useful in this context to provide a brief history of the development of
modern control theory.  

\index{amplifiers}
\index{industrial revolution}
\index{steam engine}
\index{governors}
\index{centrifugal governor}
Automatic control traces its roots to the beginning of the industrial
revolution, when simple governors were used to automatically maintain steam
engine speed despite changes in loads, steam supply, and equipment.  In the
early 20th Century, the same principles were applied in the emerging field of
electronics, yielding feedback amplifiers that automatically maintained
constant performance despite large variations in vacuum tube devices.
\index{uncertainty!component or parameter variation}

\index{Bode plots}
\index{transfer functions}
\index{frequency domain control}
\index{feedback!loop}
\index{closed loop control}
\index{Nyquist plots}
The foundations of the theory of \cds{} are rooted in the 1940s, with the
development of methods for single-input, single-output feedback loops,
including transfer functions and Bode plots for modeling and analyzing
frequency response and ^{stability}, and Nyquist plots and
gain/^{phase margin}_{gain margin} for 
studying stability of feedback systems~\cite{Ben86}.  By designing feedback
loops to avoid 
positive reinforcement of disturbances around a closed loop system, one can
ensure that the system is stable and disturbances are attenuated.  
\index{disturbance rejection}
This first
generation of techniques is known collectively as ``classical control'' 
\index{classical control}%
and is still the standard introduction to controls for engineering students.

In the 1960s, the second generation of control theory, known as ``modern
control,'' 
\index{modern control}%
was developed to provide methods for multi-variable systems where
many strongly coupled loops must be designed simultaneously.  These tools made
use of state space representations of control systems and were coupled with
advances in numerical optimization and optimal control.  
\index{optimization}
These early state space_{state space control} methods 
made use of linear ordinary differential equations to study the response
of systems, and control was achieved by placing the eigenvalues of the closed
loop system to ensure stability.
\index{differential equations}

\index{trajectory generation}
\index{closed loop control}
At around this same time, optimal control theory 
\index{optimal control}%
also made great advances,
with the establishment of the maximum principle of Pontryagin
\index{Pontryagin, L.\ S.}%
and the dynamic
programming results of Bellman.  
\index{Bellman, R. E.}%
Optimal control theory gave precise
conditions under which a controller minimized a given cost function, either as
an open loop input (such as computing the thrust for optimal trajectory
generation) or as a closed loop feedback law.  Estimation theory also
benefited from results in optimal control, and the Kalman filter 
\index{Kalman filter}%
\index{estimation}
was developed and quickly
became a
standard tool used in many fields to estimate the internal states of a system
given a (small) set of measured signals.
\index{open loop control}

\index{multi-variable control}
\index{uncertainty}
\index{operator theory}
\index{mathematics!interaction with}
\index{mathematics!contributions to}
Finally, in the 1980s the third generation of control theory, known as
``robust multi-variable control,'' 
\index{robust control}%
\index{multi-variable control}
added powerful formal methods to guarantee
desired closed loop properties in the face of uncertainties.  In many ways,
robust control brought back some of the key ideas from the early theory of
control, where uncertainty was a dominant factor in the design methodology.
Techniques from operator theory were extremely useful here and there was
stronger interaction with mathematics, both in terms of using existing
techniques and developing new mathematics.

\index{adaptive control}%
\index{nonlinear control}%
\index{geometric control}%
\index{fuzzy control}%
\index{hybrid systems}%
\index{neural networks}%
Over the past two decades, many other branches of control have appeared,
including adaptive, nonlinear, geometric, hybrid, fuzzy, and neural control
frameworks.  All of these have built on the tradition of linking applications,
theory, and computation to develop practical techniques with rigorous
mathematics.  \CDS{} also built on other disciplines, especially ^{applied
mathematics}, physics, and operations research.
\index{physics}
\index{operations research}

\index{control!principles}%
Today, control theory provides a rich methodology and a supporting set of
mathematical principles and tools for analysis and design of feedback systems.
It links 
four important concepts that are central to both engineered and natural
systems: dynamics, modeling, interconnection, and uncertainty.

\index{complex systems}
\index{dynamics}%
The role of dynamics is central to all control systems and control theory has
developed a strong set of tools for analyzing stability and performance 
\index{stability}%
\index{dynamical systems}
\index{performance}%
of
dynamical systems.  Through feedback, 
\index{feedback}%
we can alter the behavior of a system to
meet the needs of an application: systems that are unstable can be stabilized,
systems that are sluggish can be made responsive, and systems that have
drifting operating points can be held constant.  Control theory provides a
rich collection of techniques to analyze the stability and dynamic response 
\index{stability}%
of
complex systems and to place bounds on the behavior of such systems by
analyzing the gains of linear and nonlinear operators that describe their
components.  These techniques are particularly useful in the presence of
disturbances, parametric uncertainty, and unmodeled dynamics---concepts that
are often not treated in detail in traditional dynamics and dynamical systems
courses.
\index{disturbances}
\index{uncertainty!component or parameter variation}
\index{uncertainty!unmodeled dynamics}
\index{unmodeled dynamics|see{uncertainty}}

\index{modeling}%
\index{identification|see{system identification}}%
\index{system identification}%
\index{input/output models}%
Control theory also provides new techniques for (control-oriented) system
modeling and identification. Since models play an essential role in analysis
and design of feedback systems, sophisticated tools have been developed to
build such models.  These include input/output representations of systems (how
disturbances propagate through the system) and data-driven system
identification techniques.  The use of ``forced response'' 
\index{disturbances}%
\index{forced response}%
experiments to
build models of systems is well developed in the control field and these tools
find application in many disciplines, independent of the use of feedback.  A
strong theory of modeling has also been developed, allowing rigorous
definitions of model fidelity and comparisons to experimental data.

\index{interconnection}%
\index{dynamics}
\index{input/output models}%
A third key concept in control theory is the role of interconnection
between subsystems.  Input/output representations of systems allow one to build
models of very complex systems by linking component behaviors.  The dynamics
of the resulting system is determined not only by the dynamics of the
components, but by the interconnection structure between these components.  The
tools of \cds{} provide a methodology for studying the characteristics of
these interconnections and when they lead to stability, robustness, and
desired performance.

\index{uncertainty}%
Finally, one of the powerful features of modern control theory is that it
provides an {\em explicit} framework for representing uncertainty.  Thus, we
can describe a ``set'' of systems that represent the possible instantiations
of a system or the possible descriptions of the system as it changes over
time.  While this framework is important for all of engineering, the \cds{}
community has developed one of the most powerful collection of tools for
dealing with uncertainty.  This was necessary because the use of feedback is
not entirely benign.  In fact, it can lead to catastrophic failure if the
uncertainty is not properly managed (through positive feedback, for example).
\index{feedback}
\index{uncertainty management}

\subsection*{Control Technology}

\index{control!technology}%
\index{actuation}%
\index{sensing}%
\index{computation}%
\index{automobiles}%
Control {\em technology} includes sensing, actuation and
computation, used together to produce a working system.  
Figure~\ref{trends.fig}a shows some of the trends in sensing, actuation, and
computation in automotive applications.  
\begin{figure}
  \centerline{
      \quad \input{engctrl-1.pst}\quad\quad\input{trends-1.pst}
  }
  \caption{Trends in control technology: (a) the number of sensors, actuators
    and control functions in engine controls~\cite{BP96-mech} and (b)
    illustration of cost/performance trends for component technologies.}
  \label{trends.fig}
\end{figure}
\index{adaptive control}%
\index{engine control}
\index{cruise control}
\index{brakes|see{antilock brakes}}
\index{antilock brakes}%
\index{microprocessors!use, in control}
As in many other application areas, the number of sensors, actuators, and
microprocessors is increasing dramatically, as new features such as antilock
brakes, adaptive cruise control, active restraint systems, and enhanced engine
controls are brought to market.  The cost/performance curves for these
technologies, as illustrated in Figure~\ref{trends.fig}b, is also insightful.
The costs of electronics technologies, such as sensing, computation, and
communications, is decreasing dramatically, enabling more information
processing.  Perhaps the most important is the role of communications, which is
now inexpensive enough to offer many new possibilities. 
\index{communications!role of, in control}

\index{software!embedded}
\index{control!as enabling technology}
\CDS{} is also closely related to the integration of software into physical
systems.  Virtually all modern control systems are implemented using digital
computers.  Often they are just a small part of much larger computing systems
performing various other system management tasks.  Because of this, control
software becomes an integral part of the system design and is an enabler for
many new features in products and processes.  Online reconfiguration is a
fundamental feature of computer controlled systems and this is, at its heart,
a control issue.
\index{reconfigurable systems}

\index{software!embedded}%
\index{computation!ubiquitous computing}
\index{ubiquitous computing}
\index{communications!systems}%
This trend toward increased use of software in systems is both an opportunity
and a challenge for \cds{}.  As embedded systems become ubiquitous and
communication between these systems becomes commonplace, it is possible to
design systems that are not only reconfigurable, but also aware of their
condition and environment, and interactive with owners, users, and
maintainers.  
\index{smart systems}
These ``smart systems'' provide improved performance, reduced
downtime, and new functionality that was unimaginable before the advent of
inexpensive computation, communications, and sensing.  However, they also
require 
increasingly sophisticated algorithms to guarantee performance in the face of
uncertainty and component failures, and require new paradigms for verifying
the software in a timely fashion.  
\index{verification and validation}
\index{component failures, robustness to}
Our everyday experience with commercial
word processors shows the difficulty involved in getting this right.
\index{uncertainty}

\index{software!embedded}
\index{computer science}
One of the emerging areas in control technology is the generation of such
real-time_{real-time systems} embedded software~\cite{nrc01-embedded}.  While often considered within
the domain of 
computer 
science, the role of dynamics, modeling, interconnection, and uncertainty is
increasingly making embedded systems synonymous with control systems.  Thus
\cds{} must embrace software as a key element of control technology and
integrate computer science principles and paradigms into the discipline.  This
has already started in many areas, such as hybrid systems and robotics, where
the continuous mathematics of dynamics and control are intersecting with the
discrete mathematics of logic and computer science.
\index{hybrid systems}
\index{robotics}
\index{discrete mathematics}

\subsection*{Comparison with Other Disciplines}

Control engineering relies on and shares tools from physics (dynamics and
modeling), computer science (information and software) and operations research
(optimization and game theory), but it is also different from these
subjects, in both insights and approach.

\index{natural science}%
\index{engineering science}%
A key difference with many scientific disciplines is that \cds{} is
fundamentally an engineering science.  Unlike natural science, whose goal is
to understand nature, the goal of engineering science is to understand and
develop new systems that can benefit mankind.  Typical examples are systems
for transportation, electricity, communication and entertainment that have
contributed dramatically to the comfort of life.  While engineering originally
emerged as traditional disciplines such as mining, civil, mechanical,
electrical and computing, \cds{} emerged as a {\em systems} discipline around
1950 and cut across these traditional disciplines.  The pinnacle of
achievement in engineering science is to find new systems principles that are
essential for dealing with complex man-made systems.  Feedback is such a
principle and it has had a profound impact on engineering systems.
\index{feedback}
\index{systems perspective}
\index{control!principles}%

\index{physics!relationship to control}%
\index{input/output models}
\index{stability}
\index{interconnection}
\index{modeling}%
Perhaps the strongest area of overlap between \cds{} and other disciplines is
in modeling of physical systems, which is common across all areas of
engineering and science.  One of the fundamental differences between
\cds{}-oriented modeling and modeling in other disciplines is the way in
which interactions between subsystems 
(components) are represented.  \CDS{} relies on input/output modeling that
allows many new insights into the behavior of systems, such as disturbance
rejection and stable interconnection.  
\index{dynamics}
\index{model reduction}
Model reduction, where a simpler
(lower-fidelity) description of the dynamics is derived from a high fidelity
model, is also very naturally described in an input/output framework.  Perhaps
most importantly, modeling in a control context allows the design of {\em
robust} interconnections between subsystems, a feature that is crucial in the
operation of all large, engineered systems.
\index{disturbance rejection}

\index{operations research}
\index{optimization}
\index{game theory}
\index{resource allocation}
Control share many tools with the field of operations research.  Optimization
and differential games play central roles in each, and both solve problems of
asset allocation in the face of uncertainty.  The role of dynamics and
interconnection (feedback) is much more ingrained within control, as well as
the concepts of stability and dynamic performance. 

\index{computer science!relationship to control}%
\index{real-time systems}
\CDS{} is also closely associated with computer science, since virtually all
modern control algorithms are implemented in software.  However, \cds{}
algorithms and software are very different from traditional computer software.
The physics (dynamics) of the system are paramount in analyzing and designing
them and their (hard) real-time nature dominates issues of their
implementation.  From a software-centric perspective, an F-16 is simply
another peripheral, while from a control-centric perspective, the computer is
just another implementation medium for the feedback law.  Neither of these are
adequate abstractions, and this is one of the key areas identified in this
report as both an opportunity and a need.

\section{Control System Examples}

\index{consumer electronics}
\index{control!as enabling technology}
Control systems are all around us in the modern technological world.  They
maintain the environment, lighting, and power in our buildings and factories,
they regulate the operation of our cars, consumer electronics, and
manufacturing 
processes, 
they enable our transportation and communications systems, and they are
critical elements in our military and space systems.  For the most part, they
are hidden from view, buried within the code of processors, executing their
functions accurately and reliably.  Nevertheless, their existence is a major
intellectual and engineering accomplishment that is still evolving and
growing, promising ever more important consequences to society.
\index{control!as hidden technology}

\subsection*{Early Examples}
\index{control!history of!early examples}%

\begin{figure}
  \centerline{
    \begin{tabular}{ccc}
      \epsfig{figure=honeywell-t86.eps,width=0.36\figwidth} &&
      \epsfig{figure=chrysler-autopilot.eps,width=0.48\figwidth} \\
      (a) && (b)
    \end{tabular}
  }
  \index{centrifugal governor}%
  \caption{Early control devices: (a) Honeywell T86 thermostat, originally
    introduced in 1953, (b) Chrysler cruise control system, introduced in the
    1958 Chrysler Imperial (note the centrifugal governor)~\cite{Row58:psm}.}
  \index{thermostat}
  \index{Honeywell}
  \index{cruise control}
  \label{thermocruise.fig}
\end{figure}

The proliferation of control in engineered systems has occurred primarily in
the latter half of the 20th Century.  There are some familiar exceptions,
such as the Watt governor described earlier and the thermostat
(Figure~\ref{thermocruise.fig}a), 
designed at the turn of 
the century to regulate temperature of buildings.
\index{buildings, control of}

\index{temperature control}
\index{feedback}
\index{thermostat}%
The thermostat, in particular, is often cited as a simple example of feedback
control that everyone can understand.  Namely, the device measures the
temperature in a building, compares that temperature to a desired set point,
and uses the ``feedback error'' between these two to operate the heating plant,
e.g., to turn heating on when the temperature is too low and to turn if off
when temperature is too high.  This explanation captures the essence of
feedback, but it is 
a bit too simple even for a basic device such as the thermostat.  Actually,
because lags and delays exist in the heating plant and sensor, a good
thermostat does a bit of anticipation, turning the plant off before the error
actually changes sign.  This avoids excessive temperature swings and cycling
of the heating plant. 
\index{time delay} 

\index{dynamics}
This modification illustrates that, even in simple cases, good control system
design it not entirely trivial. It must take into account the dynamic behavior
of the object being controlled in order to do a good job. The more complex the
dynamic behavior, the more elaborate the modifications. In fact,
the development of a thorough theoretical understanding of the relationship
between dynamic behavior and good controllers constitutes the most significant
intellectual accomplishment of the \cds{} community, and the codification of
this understanding into powerful computer aided engineering design tools makes
all modern control systems possible.

There are many other control system examples, of course, that have developed
over the years with progressively increasing levels of sophistication and
impact. An early system with broad public exposure_{public awareness} was the
``^{cruise control}'' option introduced on ^{automobiles} in 1958 (see
Figure~\ref{thermocruise.fig}b). With cruise control, ordinary people
experienced the dynamic behavior of ^{closed loop} ^{feedback} systems in
action---the slowdown error as the system climbs a grade, the gradual
reduction of that error due to ^{integral action} in the controller, the small
(but unavoidable) overshoot at the top of the climb, etc.  More importantly,
by experiencing these systems operating reliably_{reliability} and robustly,
the public learned to trust and accept feedback systems, permitting their
increasing proliferation all around us.  Later control systems on automobiles
have had more concrete impact, such as emission_{emissions control} controls
and fuel metering systems that have achieved major reductions of
pollutants_{pollution} and increases in fuel economy.

\index{manufacturing}
\index{control!as enabling technology}
\index{computer numerically controlled (CNC) machining}
In the industrial world, control systems have been key enabling technologies
for everything from factory automation (starting with numerically controlled
machine tools), to ^{process control} in oil refineries and chemical plants, to
integrated circuit manufacturing, to power generation and distribution.
They now also play 
critical roles in the routing of messages across the Internet (TCP/IP) and in
power management for wireless communication systems.
\index{Internet}
\index{communications!systems}
\index{wireless networks}
\index{power control, in communications}

\subsection*{Aerospace Applications}
\index{aerospace systems|(}%

\index{World War II}
\index{control!successes of}
\index{control!as enabling technology}
Similarly, control systems have been critical enablers in the aerospace and
military world. 
We are familiar, for example, with the saturation bombing
campaigns of World War II, which dropped unguided explosives almost
indiscriminately on population centers in order to destroy selected industrial
or military targets.  
\index{precision guided weapons}%
These have been replaced with precision guided weapons
with uncanny accuracy, a single round for a single target.  This is enabled by
advanced control systems, combining inertial guidance
sensors_{sensing!inertial}, radar 
and infrared homing seekers, satellite navigation updates from the global
positioning system, and sophisticated processing of the ``feedback
error,''
all combined in an affordably disposable package.
\index{radar}
\index{global positioning system (GPS)}

\index{space systems}
\index{rockets}
\index{control!successes of}
We are also familiar with early space launches. Slender rockets balanced
precariously on the launch pad, failing too often in out-of-control tumbles or
fireballs shortly after ignition.  Robust, reliable, and well-designed control
systems are not optional here, because boosters themselves are
unstable. And control systems have lived up to this challenge.  We now take
routine launch operations for granted, supporting manned space stations,
probes to the outer planets, and a host of ^{satellites} for communications,
navigation, surveillance, and earth observation missions.  Of course, these
payloads are themselves critically dependent on robust, reliable and
well-designed control systems for everything from attitude control, to on-orbit
station-keeping, thermal management, momentum management, communications, etc.
\index{satellites}

\subsection*{Flight Control}
\index{flight control|(}%
\index{control!successes of}
Another notable success story for control in the aerospace world comes from
the control of flight.  This example
illustrates just how significant the intellectual and technological
accomplishments of control have been and how important their continued
evolution will be in the future.

\begin{figure}
  \centerline{
    \begin{tabular}{cc}
      \epsfig{figure=wrightflyer.eps,width=0.495\figwidth} &
      \epsfig{figure=x29.eps,width=0.44\figwidth} \\
      (a) & (b)
    \end{tabular}
  }
  \caption{Flight systems: (a) 1903 Wright Flyer, (b) X-29 forward swept wing
    aircraft, in 1987.
    X-29 photograph courtesy of NASA Dryden Flight Research Center.
  }
  \index{X-29}
  \index{Wright Flyer}
  \label{airplanes.fig}
\end{figure}

\index{Wright Flyer}%
\CDS{} has played a key role in the development of
aircraft from the very beginning.  Indeed, the Wright brother's first powered
flight was successful only because the aircraft included control surfaces
(warpable wings and forward-mounted vertical and horizontal fins) that were
adjusted continuously by the pilot_{pilots} to stabilize the
flight~\cite{hughes93} 
(see Figure~\ref{airplanes.fig}a). 
These adjustments
were critical because the Wright Flyer itself was unstable, and could not
maintain steady flight on its own.  

\index{aircraft}%
\index{stability!augmentation}
\index{dynamics!aircraft}
\index{World War II}
Because pilot_{pilots} workload is high when flying unstable aircraft, most
early 
aircraft that followed the Wright Flyer were designed to be statically
stable. Still, as the size and performance capabilities of aircraft grew,
their handling characteristics deteriorated.  Designers then installed
so-called ``stability augmentation systems''---automatic control systems
designed to modify dynamic behavior of aircraft slightly in order to make them
easier to fly.  These systems first appeared during the World War II years.
They used early inertial sensors_{sensing!inertial} to measure flight motions, analog
electronic systems to construct and process feedback errors, and 
hydraulic systems to actuate the linkages of selected control surfaces
(vertical and horizontal tails, ailerons, etc).

\index{reliability}
Two issues surfaced immediately as these systems were being fielded: (1) how
to design the control logic systematically (early systems were essentially
developed by trial-and-error), and (2) how to build the systems such that they
would operate reliably. Early systems proved to be quite unreliable. Hence,
only a small fraction of the full authority of the control surfaces was
typically allocated to the automatic system, with the bulk of authority
reserved for manual control, so the pilot could always override the
automation.

\index{modeling}
\index{transfer functions}
\index{classical control}
\index{differential equations}
Control theorists provided the solution for the first issue.  They developed
modeling and ^{simulation} methods (based on differential equations and
transfer 
functions) that accurately describe aircraft dynamics, and they
developed increasingly powerful generations of control analysis and
design methods to design control laws.  Classical control methods enabled the
systematic design of early stability augmentation_{stability!augmentation}
systems, while modern 
control and robust multi-variable control are critical in all of today's
modern flight systems.
\index{modern control}
\index{control!as enabling technology}
\index{multi-variable control}
\index{stability!augmentation}

\index{redundant architectures}
But analysis and design methods alone could not address the second issue of
early stability augmentation systems, namely the need for highly
reliable_{reliability} 
control implementations.  That issue was resolved with the development of
airborne digital computers and redundant architectures.  These are now
routinely used on all commercial and military aircraft.  They have become so
highly reliable that the old solution of granting only partial authority to
automation has long been abandoned.  In fact, most modern flight control
implementations do not even include mechanical linkages between ^{pilots} and
control surfaces.  All sensed signals and control commands go through the
digital implementation (e.g., fly-by-wire).  
\index{fly-by-wire}%

\index{auto-land}
\index{dynamics!aircraft}
Today, we even entrust the very survival of aircraft to
automation. Examples include the all weather auto-land functions of commercial
transports, in which safe_{safety} go-around maneuvers are not available if
failures 
were to occur at certain critical flight phases.  Other examples include the
F-16, B-2, and X-29 military aircraft (see Figure~\ref{airplanes.fig}), 
\index{F-16}\index{B-2}\index{X-29}\index{Wright Flyer}%
whose basic dynamics are unstable like the Wright Flyer, but so much more
violently that manual stabilization is not possible.  Finally, in modern
flight systems there is a growing trend to automate more and more
functions---all the way to removing the pilot_{pilots} entirely from the
cockpit.  This 
is already commonplace in certain military ^{reconnaissance} and
^{surveillance} 
missions and will soon be extended to more lethal ones, such as suppressing
enemy air defenses with unmanned aerial vehicles (UAVs).
\index{unmanned vehicles}

The following vignette describes some of these advances,
from the perspective of one of its successful practitioners.

%% \input{wise-flightcontrol.tex}
% Master File: cdspanel.tex

\index{Wise, K. A.}
\index{aircraft|(}
\index{flight control}
\index{control!successes of}
\index{fighter aircraft|see{aircraft}}
\index{missiles}
\index{global positioning system (GPS)}
\index{Boeing Company}
\vignette{Fighter Aircraft and Missiles (Kevin A. Wise, The Boeing Company)}
{
\label{flightcontrol.vig}
The 1990s has been a decade of significant accomplishments and change for the
aerospace community.  New systems such as unstable, tailless aircraft,
propulsion controlled ejection seats, and low-cost, accurate, GPS guided
munitions were developed.  Fly-by-wire flight control systems have become the
standard, making control system design and analysis central to military
aircraft and missile system development.  Improving pilot_{pilots} ^{safety}
and 
reducing 
costs were key focus areas in industry. 
\index{fly-by-wire}
\index{tailless aircraft}

\index{F-15 ACTIVE}
\index{X-36}
\index{feedback!linearization}
Flight control system design methods using feedback linearization paved the
way for new gain scheduled flight control systems for aircraft.  This method,
applied to the X-36 Tailless Agility Research aircraft and the F-15 ACTIVE,
uniquely allows engineers to better design flying qualities into the aircraft,
reducing design and development costs and improving pilot acceptance.
Advances in robustness theory improved analysis tools allowing engineers to
accurately predict and thus expand departure boundaries for these highly
unstable aircraft.  To further improve ^{safety}, these control laws were
augmented with neural networks for reconfigurable and damage adaptive flight
control. 
\index{neural networks}%
\index{adaptive control}%
\index{reconfigurable systems}

\index{missiles}
\index{munitions}
\index{Joint Direct Attack Munition (JDAM)}
\index{optimal control}
Missile systems, such as the Joint Direct Attack Munition (JDAM) and the
Miniaturized Munition Technology Demonstrator (MMTD) developed their flight
control designs using state feedback optimal control, and then projecting out
those states not measured by {sensors}_{sensing}.  This method
eliminated sensor hardware, 
reducing weight and costs, and proved to be completely automatable.  The
Fourth Generation Escape System (GEN4) ejection seat also used this approach
for its control laws.  In addition to needing optimal performance, advances
in robustness theory were used to characterize the seat's control system
performance to uncertain crew member size and weight (95\% male to 5\%
female).
\index{autocode}%
\index{uncertainty}
Autocode software tools for implementing controls systems also emerged in the
1990s.  These computer aided design tools provide a single environment for
control design and analysis as well as software design and test.  They have
greatly reduced the implementation and testing costs of flight control
systems. 
\index{software}

\index{unmanned vehicles}
The new challenge faced by the control community is the development of
unmanned combat systems (munitions as well as aircraft) and concepts of
operations for these systems to address the intelligent, increasingly hostile,
rapidly changing environments faced by our war fighters.  These systems must
detect, identify, locate, prioritize, and employ ^{ordinance} to achieve
permanent destruction of high value targets.  New developments in ^{intelligent
control}, vision based control, 
\index{vision guided systems}
mission planning, ^{path planning}, decision
aiding, communication architectures, logistics and support concepts, and last
but not least, software development, validation, and verification are needed
to support these systems and make them affordable. 
\index{verification and validation}
\index{logistics}
}
\index{aircraft|)}

\index{flight control|)}%
\index{aerospace systems|)}%

\section{The Increasing Role of Information-Based Systems} 
\index{information-based systems|(}

\index{disk drives}
Early applications of control focused on the ^{physics} of the system being
controlled, whether it was the thermal dynamics of buildings, the flight
mechanics of an airplane, or the tracking properties of a disk drive head.
The situation we now face is one in which pervasive computing, sensing,
and communications are common and the way that we interact
with machines and they interact with each other is changing rapidly.  The
consequences of this 
tremendous increase in information are also manifest in \cds{}, where we are
now facing the challenges of controlling large-scale systems_{complex systems}
and networks that
are well beyond the size and complexity of the traditional applications of
\cds{}.
\index{control!challenges}

\index{decision making!logic}
\index{software!embedded}
\index{software}
One indication of this shift is the role that embedded systems and software
play in modern technology, described briefly above.  Modern computer control
systems are capable of enormous amounts of decision making and control logic.
Increasingly, these software systems are interacting with physical processes
and introducing feedback algorithms to improve performance and robustness.
Already, the amount of logic-based code is overshadowing the traditional
control algorithms in many applications.  Much of this logic is interwoven
with the closed loop performance of the system, but systematic methods for
analysis, verification, and design have yet to be developed.
\index{verification and validation}

\index{resource allocation}
Another area where control of information-based systems will be increasingly
important is in resource allocation systems.  In this context, \cds{} can be
described as the science and engineering of optimal dynamic resource
allocation under uncertainty.  
We start with a mathematical model, of a system
that describes how current actions or decisions can affect the future behavior
of the system, including our uncertainty in that behavior.  ``Resource
allocation'' means that our decisions can be interpreted as managing a
tradeoff between competing goals, or choosing from a limited set of possible
actions.  
\index{uncertainty!in resource allocation}
``Uncertainty'' is critical: there is some possible variation in the
system's behavior, so that decisions have to be made taking different
possibilities into account.  Sources of uncertainty include incomplete or
corrupted information available to the decision maker, uncertainty in the
mathematical model used to model the system, and unpredictability of commands
due to noise and disturbance signals that affect the system.  
\index{disturbances}
While often
considered an operations research problem, the role of dynamics and
instabilities points to a clear need for control theory as well.
\index{operations research}

One of the consequences of this shift toward information-based systems is
that we are moving from an era where physics was the bottleneck to progress to
one in which complexity is the bottleneck.
\index{complexity}

\index{Internet}
\index{power control, in communications}
\index{congestion control}
\index{wireless networks}
\index{supply chains}
\index{air traffic control}
There are already many examples of this new class of systems that are being
deployed.  Congestion control in routers for the Internet, power control in
wireless communications systems, and real-time_{real-time systems} use of
information in service 
and supply chains are a few examples.  In all of these systems, it is the
interaction of information flow with the underlying physics that is
responsible for the overall performance.  Another example is the air traffic
control network, where the density of flights, demand for efficiency, and
intolerance for failure have created a situation that couples vast amounts of
information---everything from the location of the planes to the individual
customer itineraries---that must be managed to maintain high performance,
robust, and reliable operation at all times.  
\begin{figure}
  \centerline{\psfig{figure=tfcflow-east.eps,width=0.85\figwidth}}
  \caption{San Francisco Bay area aircraft arrival and departure routes.
  Figure courtesy of Federal Aviation Authority.}
  \label{tfcflow.fig}
\end{figure}
Figure~\ref{tfcflow.fig} shows just one small part of this problem, the local
departure and arrival routes in the San Francisco Bay area.

There is an important role for \cds{} in many of these applications.  As
in traditional application areas, \cds{} serves as a mechanism for getting
both information and, more importantly, {\em action} out of data.
Furthermore, the theory of \cds{} provides insights and tools for
analyzing and designing interconnected systems with desirable stability and
robustness properties.

\index{communications!role of, in control}
\index{feedback}
\index{distributed control}
\index{networking!role of, in control}
One fundamental change in the use of \cds{} is the role of communications and
networking.  This will radically change the use of feedback in modern systems,
through increased access to large amounts of information as well as the new
environment in which control systems will have to operate.  Control
computations must increasingly be done in a distributed, partially
asynchronous environment where tight control over the timing of data and
computation is not available, due for example to the existence of packet-based
communications networks between sensing, actuation, and computational
nodes.  Many traditional approaches may no longer work in this context and
we anticipate the need to develop new paradigms for designing robust, high
performance, feedback systems in this information rich environment.
\index{asynchronous systems}
\index{packet-based systems}

\index{uncertainty}
\index{uncertainty management}
\index{information-based systems}
\index{complexity!managing}
The role of uncertainty in information rich systems is also critical (and
largely unexplored) and concepts 
from \cds{} will play an important role in managing this uncertainty in the
analysis, design, and operation of large-scale, interconnected systems.
Uncertainty must be represented in order to build
tractable models for answering questions that take into account the whole
range of possible variations in the details of components and their
interconnections.  Control ideas will be increasingly important as a tool for
managing both the complexity and uncertainty in these systems, and must be made
available to the designers of such systems, through education and design
software.  One aspect of this that is likely to be particularly important is
the exploration of fundamental limits of performance, robustness, and
stability, since tradeoffs between these will be the primary design
challenge in this space.

\index{airspace management}
\index{vehicle management}
\index{transportation systems}
\index{mission management}
\index{airspace management}
\index{biology!biological networks}
Examples of the need for increased development in this area can be seen in the
applications discussed in the next chapter.  Vehicle, mission, and airspace
management systems for transportation; source, power, and router control for
networks; and genetic, cellular, and integrative feedback networks in
biological systems are just a few examples.  The simplest of these problems
lies at the boundaries of current tools and understanding, and future
progress will require a much deeper understanding of the integration between
control, communications, computing, and networks as well as modeling,
analysis, and control of complex decision systems.
\index{decision making!systems}

\index{information-based systems|)}

\section{Opportunities and Challenges Facing the Field}
\index{control!challenges|(}

\CDS{} has developed into a major field in which generations of engineers are
able to solve problems of practical importance and enormous impact.  Over the
past few years, the opportunities for \cds{} have expanded enormously, but
there are many challenges that must be addressed to realize the potential for
impact.  In this section we attempt to characterize some of the overarching
themes that describe these opportunities and challenges, and recommend an
approach for moving forward.

\subsection*{Characteristics of the New Environment}

The future of \cds{} will be driven by a new environment that differs
substantially from that of the past 40 years.  Some of the features of this
new environment are already apparent and provide insight into the new research
directions that must be pursued.

\index{computation!ubiquitous computing}
\index{ubiquitous computing}
\index{MEMS}
\index{microelectronics}
\index{actuation}
\partitle{Ubiquitous Computation, Communication and Sensing} 
The dominant change in the engineering environment is the presence of ever
more powerful computation and cheaper communication.  The new software and
storage products that these developments have spawned have further changed the
engineering landscape in many areas. In addition, microelectronics and MEMS
have made available inexpensive sensors_{sensing}, such as those shown in
Figure~\ref{sensors.fig}, and new actuator concepts that can be 
made available via communication networks, allowing increasingly sensor rich
and actuator rich control.
\index{data rich systems}
\begin{figure}
  \centerline{
    \begin{tabular}{ccc}
      \psfig{figure=ccd1024.eps,height=1.1in} &
      \psfig{figure=microgyro2.eps,height=1.1in} &
      \psfig{figure=swpod-s.eps,height=1.1in} \\
      (a) \quad & \quad (b) \quad & \quad (c)
    \end{tabular}
  }
  \caption{Examples of current sensor technology: 
  (a) 1024x1024 CCD array,
  (b) ^{MEMS}-based microgryoscope, and 
  (c) sensor web_{sensing!sensor webs} pod.
  All photographs courtesy of Jet Propulsion Laboratory.
  }
  \label{sensors.fig}
  \index{sensing}
\end{figure}

\index{telephone system}
\index{power grid, electrical}
\index{Internet}
It will require decades to take full advantage of these developments. Some
innovation will involve standalone improvements to individual systems and some
will involve extreme interconnectedness of the type seen in the telephone
system, the power grid, the Internet, and their descendants. Both types may,
and probably will, depend on the use of \cds{}. The new ideas required to be
successful in the two cases are, however, likely to be qualitatively different
because we do not yet have a great deal of experience in building and
operating safe, reliable, highly interconnected systems.

\index{new domains, control in}
\index{biology}
\index{biological engineering}
\index{education}
\partitle{New Application Domains} 
In addition to the revolutionary changes in information technology, future
control systems will involve interactions between physical, chemical,
biological, and information sciences, integrated with algorithms and feedback.
This will open up new application domains for \cds{}, such as biological
engineering and quantum systems.  While there are already researchers within
the \cds{} community that are attacking problems in these areas, it will be
necessary to educate new generations of researchers in both \cds{} and other
disciplines in order to make advances in these applications.  The
possibilities for \cds{} are potentially very fundamental in nature, as
illustrated in the following vignette.
\index{quantum systems}
\index{discipline boundaries}

\index{Mabuchi, H.}
\vignette{Quantum Measurement and Control (Hideo Mabuchi, Caltech)}
{
\label{quantum.vig}%
To illustrate the applications of \cds{} in new domains, consider the
research of Hideo Mabuchi, who is exploring the use of feedback and control in
quantum systems and its implications for unifying quantum and classical
^{physics}:
\begin{quote}
A grand enigma, which is perhaps our primary legacy from 20th Century
physics, is that the states and dynamics we ascribe to microscopic (quantum)
systems 
seem incompatible with macroscopic (classical) phenomenology.  For example,
physical theory claims that it should be illogical simultaneously to assign
definite values to certain sets of measurable properties of a quantum system.
And yet we want to believe that quantum mechanics is a correct description of
microscopic physics, which evolves robustly into classical dynamics for
systems of sufficiently large size and with a sufficiently high degree of
interconnection among their manifold degrees of freedom.  How can we
understand the consistency of quantum mechanics, as a microscopic theory, with
classical physics as a manifestly valid description of macroscopic phenomena?
\end{quote}
\index{dynamics}

\index{model reduction}
Control theory provides a new set of tools for understanding quantum systems.
One set of tools is through systematic techniques for model reduction:
\begin{quote}
\index{multiscale systems}
Viewed from a ``multiscale'' perspective, our challenge in explaining the
quantum-classical transition will be to show that classical physics can
rigorously be obtained as a robust and parsimonious approximation to the
dynamics of certain aggregate degrees of freedom for generic complex quantum
systems.  In the language of control theory, one would like to derive
classical physics as an optimal model reduction of quantum physics. A number
of fundamental questions arise as soon as the problem is posed this way.  How
can this model reduction be so general and robust, depending only upon the
structure of quantum theory and not the details of any particular dynamical
system?  What are the general parameters that control the error bounds on this
model reduction?  What impact will this program have, if successful, on our
basic interpretation of quantum mechanics? 
\end{quote}

\index{experiments}
In addition, control can provide new techniques for doing experiments,
allowing 
us to better explore physical understanding:
\begin{quote}
... we hope that feedback control will provide a crucial
experimental methodology for scrutinizing the validity of quantum measurement
theory in realistic laboratory scenarios, especially with regard to the
equations for conditional evolution of a system under continuous observation.
Such equations could be used as the starting point for controller synthesis,
for example, and their validity would be assessed by comparison of
experimentally observed closed-loop behavior with theoretical expectations.   
\end{quote}

Mabuchi's work illustrates the potential power of control theory as a
disruptive technology for understanding the world around us.
}

\index{reliability}
\index{reconfigurable systems}
\index{component failures, robustness to}
\partitle{Reliable Systems with Unreliable Parts} 
Most reasonably complex man-made systems are not rendered inoperable by the
failure of any particular component and biological systems often demonstrate
remarkable robustness in this regard. Simple redundancy, or the spare parts
approach to such problems, is of limited effectiveness because it is
uneconomical.  Designs that allow the system to reconfigure itself when a
component fails, even if this degrades the performance roughly in proportion
to the magnitude of the failure, are usually preferred. Although computer
memory chips and disk drive controllers often take advantage of strategies of
this type, it is still true that the design of self healing systems is not well
studied or analyzed.
\index{disk drives}

This issue takes on considerable significance when dealing with interconnected
systems of the complexity of the Internet. In this case there are billions of
components and yet the system is so essential that little downtime can be
tolerated.

\index{complexity|(}
\partitle{Complexity} 
Air traffic control systems, power grid control systems and other large-scale,
interconnected systems_{complex systems} are typical of a class of problems
whose complexity is 
fixed not by the designer but rather by economic considerations and the
natural scale of the problem.  An acceptable solution in this context must be
capable of dealing with the given complexity. In deciding if a system can be
built or not, it is important to correctly gauge the feasibility because there
is no value in a product that ``almost'' works.

\index{complexity!managing}
Every discipline has methods for dealing with some types of complexity. In the
physical sciences, for example, the tools developed for studying statistical
mechanics have lead to a very substantial body of literature, effective for
solving some problems.  However, in discussing complexity it is one thing to
find a point of view from which aspects of the behavior is compressible (e.g.,
``the entropy of a closed system can only increase'') but it is another to have
a ``theory of complex systems''. The latter is something of an oxymoron, in
that it suggests that the system is not really complex. On the other hand, it
does make sense to seek to understand and organize the methodologies which
have proven to be useful in the design of highly interconnected systems and to
study naturally occurring systems with this in mind. Engineers looking at the
immune system may very well be able to suggest new methods to defeat Internet
viruses and ideas from neuroscience may inspire new developments in building
reliable systems using unreliable components.
\index{complexity|)}
\index{neuroscience}

\subsection*{Vision for the Future}

This new environment for \cds{} presents many challenges, but also many
opportunities for impact across a broad variety of application areas.  The
future directions in control, dynamics, and systems must continue to address
fundamental issues, guided 
by new applications.

\index{ubiquitous computing}
\index{computation!ubiquitous computing}
\index{communications!integration of control and}
\index{computation!integration of control and}
One of the biggest challenges facing the field is the integration of
computation, 
communications, and control.  As computing, communications, and sensing become
more ubiquitous, the use of \cds{} will become increasingly
ubiquitous as well.  However, many of the standard paradigms that allow the
separation of these different disciplines will no longer be valid.  For
example, 
the ability to 
separate the computational architecture from the functions that are
being computed is already beginning to unravel as we look at distributed
systems with redundant, intermittent, and sometimes unreliable computational
elements.  Beyond simply looking at hybrid systems, a theory must be developed
that integrates computer science and \cds{}.
\index{hybrid systems}

\index{networks}
Similarly, the simplification that two nodes that are connected can
communicate with sufficient reliability and bandwidth such that the properties of
the communications channel can be ignored no longer holds in the highly
networked environment of the future.  \CDS{} must become more integrated with
the ^{protocols} of communications so that high response feedback loops are
able 
to use the same channels as high throughput, lower bandwidth information,
without interfering with each other.

\index{autonomous systems}
\index{path planning}
\index{regulation}
\index{tracking}
\index{obstacle avoidance}
\index{decision making!higher level}
Another element of the future of \cds{} is to begin to understand analysis and
synthesis of \cds{} using higher levels of decision making.  Traditionally
\cds{} has dealt with the problem of keeping a few variables constant
(regulation) or making variables follow specified time functions
(tracking). In ^{robotics}, \cds{} was faced with more complicated problems
such 
as obstacle avoidance and path planning (task-based control).  Future systems
will require that \cds{} be applied to problems that cannot necessarily be
expressed in terms of continuous variables, but rather have symbolic,
linguistic, or 
protocol-based descriptions.  
\index{symbolic dynamics}%
\index{protocols}%
\index{linguistics}%
This is required as we move to more
sophisticated autonomous and semi-autonomous systems that require high-level
decision making capabilities.

\index{new domains, control in}
\index{quantum systems}
\index{environmental science}
\index{biology}
At the same time as \cds{} moves to higher levels of decision making, it will
also move to new domains that are only beginning to emerge at the present
time.  This includes biological, quantum and environmental systems;
software systems; enterprise level systems; and economic and financial
systems.
\index{economic systems}
\index{enterprise level systems}
In all of these new problem domains, it will be necessary to develop a {\em
rigorous} theory of \cds{}.  This has been a historical strength of the field
and has allowed it to be successful in an enormous number of systems.

\index{education}
Finally, we envision an increased awareness of \cds{} principles in science
and engineering, including much more exposure to feedback systems in ^{math and
science education}.

\subsection*{Approach}

The opportunities and challenges describe here should be addressed on two
fronts. There is a need for a broadly supported, active research program whose
goals are to explore and further develop methodologies for design and
operation of reliable and robust highly interactive systems, and there is a
need 
to make room in the academic programs for material specific to this area.

\index{software}
\index{systems engineering}
\index{biology}
\index{computer science}
\index{discipline boundaries}
\index{economics}
\index{molecular biology}
\index{environmental science}
\index{operations research}
The research program must be better integrated with research activities in
other disciplines and include scientists trained in software engineering,
molecular biology, statistical mechanics, systems engineering and
^{psychology}. 
\CDS{} researchers must continue to branch out beyond traditional discipline
boundaries and become participants and contributors in areas such as computer
science, biology, economics, environmental science, materials science and
operations research.  There is particular need for increased \cds{} research
in information-based systems, including communications, software, verification
and validation, and logistics.
\index{software}
\index{verification and validation}
\index{logistics}

\index{compactification}
\index{accessibility, broadening}
\index{education}
To support this broader research program, a renewed academic program must also
be developed.  This program should strengthen the systems view and stretch
across traditional discipline boundaries.  To do so, it will be necessary to
provide better dissemination of tools to new communities and provide a broader
education for \cds{} engineers and researchers.  This will require
considerable effort to present current knowledge in a more compact way and to
allow new results in 
software, communications, and emerging application domains to be added, while
maintaining the key principles of \cds{} on which new results will rest.
Simultaneously, the \cds{} community must seek to increase exposure to
feedback in ^{math and science education} at all levels, even K-12.  Feedback
is 
a fundamental principle that should be part of every technically literate
person's knowledge base.
\index{K-12 education}
\index{control!principles}%

\index{mathematics}
One of the characteristics of the \cds{} field has been an emphasis on theory
and mathematical 
formulations of the problems being considered. This discipline has resulted in
a body of work that is reliable and unambiguous.  Moreover, because this style
appeals to some very able graduate students, it has been an important factor
in maintaining the flow of talent into the field. However, for engineers
and scientists this has been a barrier to entry and can make it difficult for
outsiders to assimilate and use the work in their own field. In addition, it
has sometimes had a chilling effect on the development of ideas that are not
easily translated into mathematical form. The challenge presented by the need
to steer a course between the possible extremes here is not new, it has always
been present.  What is new is that the availability of easily used
^{simulation}
tools has made the use of heuristic reasoning both more appealing and more
reliable. 
\index{optimization}%
In particular, optimization involving problems that are so large
and/or so badly non-convex that rigorous analysis is infeasible can now be
approached using principled heuristics. Because of the software and computing
power now available this may be the most effective way to proceed. It is
important to find a place for effective heuristics in the training of students
and the highest level professional meetings of the field.
\index{heuristics}

\index{experiments}
Finally, experimentation on representative systems must be an integral part of
the \cds{} community's approach.  The continued growth of experiments, both
in education and research, should be supported and new experiments that
reflect the new environment will need to be developed.  These experiments are
important for the insight into application domains that they bring, as well as
the development of software and algorithms for applying new theory.  But they
also form the training ground for systems engineers, who learn
about modeling, robustness, interconnection, and data analysis through
their experiences on real systems.
\index{systems engineering}

The recommendations of the Panel, detailed in Chapter~\ref{recomm.sec}, provide
a high level plan for implementing this basic approach.  The recommendations
focus on the need to pursue vigorously new application domains and, in
particular, those domains in which the principles of \cds{} will be essential
for future progress.  They also highlight the need to maintain the field's
strong 
theoretical base and historical rigor, while at the same time finding new ways
to broaden the exposure and use of \cds{} to a broader collection of
scientists and engineers.

The new environment that \cds{} faces is one with many new challenges and an
enormous array of opportunities.  Advancing the state of the art will require
that that the community accelerate its integration across disciplines and look
beyond the current paradigms to tackle the next generation of applications.
In the next chapter, we explore some of the application areas in more detail
and identify some of the specific advancements that will be required.

\index{control!challenges|)}
%% \include{challenges}		% Opportunities and Challenges
\clearpage
% Master File: cdspanel.tex
\ifx\siam\undefined
\chapter[Applications]{Applications, Opportunities, and Challenges}
\else
\chapter{Applications, Opportunities, and Challenges}
\fi
\index{control!challenges}

In this chapter, we consider some of the opportunities and challenges for
\cds{} in different application areas.  The Panel decided to organize the
treatment of applications around five main areas to identify the overarching
themes that would guide its recommendations.  These are:
\begin{itemize}
  \item Aerospace and transportation
  \item Information and networks
  \item Robotics and intelligent machines
  \item Biology and medicine
  \item Materials and processing
\end{itemize}
In addition, several additional areas arose over the course of the Panel's
deliberations, including environmental science and engineering, economics and
finance, and molecular and quantum systems.  Taken together, these
represent an enormous collection of applications and demonstrate the breadth
of applicability of ideas from \cds{}.

The opportunities and challenges in each of these application areas form the
basis for the major recommendations in this report.  In each area, we have
sought the advice and insights not only of \cds{} researchers, but
also experts in the application domains who might not consider themselves to
be \cds{} researchers.  In this way, we hoped to identify the true challenges
in each area, rather than simply identifying interesting \cds{} problems that
may not have a substantial opportunity for impact.  We hope that the findings
will be of interest not only to \cds{} researchers, but also to
scientists and engineers seeking to understand how \cds{} tools might be
applied to their discipline.

\index{overarching themes}
\index{communications!integration of control and}
\index{computer science!interaction of control and}
There were several overarching themes that arose across all of the areas
considered by the Panel.  The use of systematic and rigorous_{rigor} tools is
considered critical to future success and is an important trademark of the
field.  At the same time, the next generation of problems will
require a paradigm shift in \cds{} research and education.  The increased
information available across all application areas requires more integration
with ideas from computer science and communications, as well as improved tools
for modeling, analysis, and synthesis for complex decision systems that
contain a mixture of symbolic and continuous dynamics.  
The need to continue
research in the theoretical foundations that will underly future advances was
also common across all of the applications.
\index{decision making!systems}
\index{complex systems}
\index{symbolic dynamics}

In each section that follows we give a brief description of the
background and history of \cds{} in that domain, followed by a selected set of
topics which are used to explore the future potential for control and the
technical challenges that must be addressed.  As in the rest of the report, we
do not attempt to be comprehensive in our choice of topics, but rather
highlight some of the areas where we see the greatest potential for impact.
Throughout these sections, we have limited the references to those that
provide historical context, future directions, or broad overviews in the topic
area, rather than specific technical contributions (which are too numerous to
properly document).

\clearpage
%% \input{aerotran.tex}
% Master File: cdspanel.tex
\section{Aerospace and Transportation}
\index{aerospace systems|(}
\index{transportation systems|(}

\index{Wright, Wilbur}
\begin{itemize}
  \small\item[]
  {\em Men already know how to construct wings or airplanes, which when driven
  through the air at sufficient speed, will not only sustain the weight of the
  wings themselves, but also that of the engine, and of the engineer as
  well. Men also know how to build engines and screws of sufficient lightness
  and power to drive these planes at sustaining speed ... Inability to balance
  and steer still confronts students of the flying problem. ... When this one
  feature has been worked out, the age of flying will have arrived, for all
  other difficulties are of minor importance.}

  Wilbur Wright, lecturing to the Western Society of Engineers in
  1901~\cite{mcfarland53}.
\end{itemize}  

\index{control!as enabling technology}
Aerospace and transportation encompasses a collection of critically important
application areas where \cds{} is a key enabling technology.  These
application areas represent a very large part of the modern world's overall
technological capability.  They are also a major part of its economic
strength, and they contribute greatly to the well being of its people. The
historical role of control in these application areas, the current challenges
in these areas, and the projected future needs all strongly support the 
recommendations of this report.

\subsection*{The Historical Role}
\index{control!history of!aerospace}

\index{aircraft}
\index{Wright Flyer}
In aerospace, specifically, \cds{} has been a key technological capability
tracing back to the very beginning of the 20th Century.  Indeed, the
Wright 
brothers are correctly famous not simply for demonstrating powered
flight---they actually demonstrated {\em controlled} powered flight. Their early
Wright
Flyer incorporated moving control surfaces (vertical fins and canards) and
warpable wings that allowed the pilot_{pilots} to regulate the aircraft's
flight.  In 
fact, the aircraft itself was not stable, so continuous pilot corrections were
mandatory.  This early example of controlled flight is followed by a
fascinating success story of continuous improvements in flight control
technology, culminating in the very high performance, highly reliable
automatic flight control systems we see on modern commercial and military
aircraft today (see Fighter Aircraft and Missiles Vignette,
page~\pageref{flightcontrol.vig}).
\index{control!successes of}
\index{flight control}
\index{autopilot}

\index{World War II}
\index{radar}
\index{precision guided weapons}
\index{space systems}
Similar success stories for control technology occurred in many other
aerospace application areas.  Early World War II bombsights and fire control
servo systems have evolved into today's highly accurate radar guided guns and
precision guided weapons.  Early failure-prone space missions have evolved
into routine launch operations, manned landings on the moon, permanently
manned space stations, robotic vehicles roving Mars, orbiting vehicles at the
outer planets, and a host of commercial and military satellites serving
various surveillance, communication, navigation and earth observation needs.
\index{control!successes of}

\index{automobiles}
\index{fuel economy}
Similarly, control technology has played a key role in the continuing
improvement and evolution of transportation---in our cars, highways,
trains, ships and air transportation systems. \CDS{}'s contribution
to the dramatic increases of ^{safety}, ^{reliability} and fuel economy of
the automobile is particularly noteworthy.  Cars have advanced from
manually tuned mechanical/pneumatic technology to computer controlled
operation of all major functions including fuel injection, emission
control, cruise control, braking, cabin comfort, etc.  Indeed, modern
automobiles carry dozens of individual processors to see to it that
these functions are performed accurately and reliably over long
periods of time and in very tough environments.  A historical
perspective of these advances in automotive applications is provided
in the following vignette.
\index{emissions control}
\index{cruise control}
\index{control!successes of}

\index{automotive systems}
%% \input{bp96-emissions.tex}
\vignette{Emissions Requirements and Electronic Controls for
Automotive Systems (Mark Barron and William Powers, Ford Motor Company)}
{
\index{Barron, M. B.}
\index{Powers, W. F.}
\index{control!successes of}
One of the major success stories for electronic controls is the
development of sophisticated engine controls for reducing emissions
and improving efficiency.  Mark Barron and Bill Powers described some
of these advances in an article written in 1996 for the inaugural
issue of the {\em IEEE/ASME Transactions on
Mechatronics}~\cite{BP96-mech}.
\index{engine control}
\index{emissions control}
\index{mechatronics}

In their article, Barron and Powers describe the environment that led
up to the introduction of electronic controls in automobile engines:
\index{environmental regulations}
\begin{quote}
Except for manufacturing technology, the automobile was relatively
benign with respect to technology until the late 1960s.  Then two
crises hit the automotive industry.  The first was the environmental
crisis.  The environmental problems led to regulations which required
a reduction in automotive emissions by roughly an order of magnitude.
The second crisis was the oil embargo in the early 1970s which created
fuel shortages, and which lead to legislation in the U.S.\ requiring a
doubling of fuel economy.  ...

\index{fuel economy}
\index{microprocessors!use, in control}
Requirements for improved fuel efficiency and lower emissions demanded
that new approaches for controlling the engine be investigated.  While
today we take for granted the capabilities which have been made
possible by the microprocessor, one must remember that the
microprocessor wasn't invented until the early 1970s.  
When the first
prototype of a computerized engine control system was developed in
1970, it utilized a minicomputer that filled the trunk of a car.  But
then the microprocessor was invented in 1971, and by 1975 engine
control had been reduced to the size of a battery and by 1977 to the
size of a cigar box.
\end{quote}

These advances in hardware allowed sophisticated control laws that
could deal with the complexities of maintaining low emissions and high
fuel economy:
\index{temperature control}
\index{catalytic converter}
\begin{quote}
The introduction in the late 1970s of the platinum catalytic converter
was instrumental in reducing emissions to meet regulations.  The
catalytic converter is an impressive passive device which operates
very effectively under certain conditions.  One of the duties of the
engine control system is to maintain those conditions by patterning
the exhaust gases such that there are neither too many hydrocarbons
nor too much oxygen entering the catalyst.  If the ratio of air to fuel
entering the engine is kept within a very tight range (i.e., a few
percent) the catalyst can be over 90\% efficient in removing
hydrocarbons, carbon monoxide, and oxides of nitrogen.  However, the
catalyst isn't effective until it has reached a stable operating
temperature greater than 600$^\circ$F (315$^\circ$C), and a rule of
thumb is that 80\% of emissions which are generated under federal test
procedures occur during the first two minutes of operation while the
catalyst is warming to its peak efficiency operating temperature.  On
the other hand if the catalyst is operated for an extended period of
time much above 1000$^\circ$F (540$^\circ$C) it will be destroyed.
Excess fuel can be used to cool the catalyst, but the penalty is that
fuel economy gets penalized.  So the mechatronic system_{mechatronics} must
not only 
control air-fuel ratios so as to maintain the catalyst at its optimum
operating point, it must control the engine exhaust so that there is
rapid lightoff of the catalyst without overheating, while
simultaneously maintaining maximum fuel efficiency.
\end{quote}

The success of \cds{} in meeting these challenges is evident in the
reduction of emissions that has been achieved over the last 30
years~\cite{pn00-cep}:
\begin{quote}
US, European and Japanese Emission Standard continue to require
significant reductions in vehicle emissions.  Looking closely at US
passenger car emission standards, the 2005 level of hydrocarbon (HC)
emissions is less than 2\% of the 1970 allowance.  By 2005, carbon
monoxide (CO) will be only 10\% of the 1970 level, while the permitted
level for oxides of nitrogen will be down to 7\% of the 1970 level.
\end{quote}

Furthermore, the experience gained in engine control provided a path
for using electronic controls in many other
applications~\cite{BP96-mech}: 
\index{antilock brakes}%
\begin{quote}
Once the industry developed confidence in on-board computer control,
other applications rapidly followed.  Antilock brake systems, computer
controlled suspension, steering systems and air bag passive restraint
systems are examples.  The customer can see or feel these systems, or
at least discern that they are on the vehicle, whereas the engine
control system is not an application which is easily discernible by
the customer.  Computers are now being embedded in every major
function of the vehicle, and we are seeing combinations of two or more
of these control systems to provide new functions.  An example is the
blending of the engine and antilock brake system to provide a
traction control system, which controls performance of the vehicle
during acceleration whereas antilock brakes control performance of the
vehicle during deceleration.
\end{quote}
}

\index{reliability}
An important consequence of the use of \cds{} in automobiles was
its success in demonstrating that \cds{} provided safe and reliable
operation.  The cruise control option introduced in the late 1950s was
one of the first servo systems receiving very broad public exposure.
\index{public awareness}
Our society's inherent trust in control technology traces back to the
success of such early control systems.

\index{flight control}
Certainly, each of these successes owes its debt to improvements in many
technologies, e.g.\ propulsion, materials, electronics, computers, sensors,
navigation instruments, etc.  However, they also depend in no small part on
the continuous improvements that have occurred over the century in the theory,
analysis methods and design tools of \cds{}.  As an example, ``old timers''
in 
the flight control engineering community still tell the story that early
control systems (circa World War II) were designed by manually tuning
feedback gains in flight---in essence, trial-and-error design performed on
the actual aircraft.  Dynamic modeling methods for aircraft were in their
infancy at that time, and formal frequency-domain design theories to stabilize
and shape single-input single-output feedback loops were still only subjects
of academic study.  Their incorporation into engineering practice
revolutionized the field, enabling successful feedback systems designed for
ever 
more complex applications, consistently, with minimal trial-and-error, and
with reasonable total engineering effort.
\index{modeling}
\index{dynamics!aircraft}

\begin{figure}
  \centerline{
    \begin{tabular}{cc}
      \psfig{figure=fa1802.eps,height=1.85in} &
      \psfig{figure=x45.eps,height=1.85in} \\
      (a) & (b)
    \end{tabular}
  }
  \caption{(a) The F-18 aircraft, one of the first production military fighters
    to use ``fly-by-wire'' technology, and (b) the X-45 (UCAV) unmanned aerial
    vehicle.  Photographs courtesy of NASA Dryden Flight Research Center.
  }
  \index{F-18}
  \index{X-45}
  \index{fly-by-wire}
  \index{unmanned vehicles}
  \label{f18.fig}
\end{figure}

Of course, the formal modeling, analysis and control system design methods
described above have advanced dramatically since mid-century.  As a result of
significant R\&D activities over the last fifty years, the state of the art
today allows controllers to be designed for much more than single-input
single-output 
systems.  The theory and tools handle many inputs, many outputs, complex
uncertain dynamic behavior, difficult disturbance environments, and ambitious
performance goals.
\index{dynamics}
In modern aircraft and transportation vehicles, dozens of
feedback loops are not uncommon, and in process control the number of loops
reaches well into the hundreds. Our ability to design and operate such systems
consistently, reliably, and cost effectively rests in large part on the
investments and accomplishments of \cds{} over the latter half of the century.
\index{feedback!loop}

\subsection*{Current Challenges and Future Needs}
\index{control!challenges}

Still, the control needs of some engineered systems today and those of
many in the future outstrip the power of current tools and theories.
This is so because current tools and theories apply most directly
to problems whose dynamic behaviors are smooth and continuous,
governed by underlying laws of physics and represented mathematically
by (usually large) systems of differential equations.  Most of the
generality and the rigorously provable features of existing methods can be
traced to this nature of the underlying dynamics.
\index{differential equations}
\index{dynamics}

\index{aircraft}
\index{vehicle management}
\index{health status, system}
\index{inner loop}
\index{outer loop}
Many new control design problems no longer satisfy these underlying
characteristics, at least in part.  Design problems have grown from so-called
``inner loops'' in a control hierarchy (e.g.\ regulating_{regulation} a
specified flight 
parameter) to various ``outer loop'' functions which provide logical
^{regulation} 
of operating modes, vehicle configurations, payload configurations, health
status, etc~\cite{Ban+97-uav}.  For aircraft, these functions are collectively
called ``vehicle 
management.''  They have historically been performed by ^{pilots} or other
human 
operators and have thus fallen on the other side of the man-machine boundary
between humans and automation.  Today, that boundary is moving! 

\index{unmanned vehicles}
There are compelling reasons for the boundary to move. They include economics
(two, one or no crew members in the cockpit versus three), ^{safety} (no
operators exposed to dangerous or hostile environments), and performance (no
operator-imposed maneuver limits).  A current example of these factors in
action is the growing trend in all branches of the military services to field
unmanned vehicles.  Certain benign uses of such vehicles are already
commonplace (e.g.\ ^{reconnaissance} and ^{surveillance}), while other more
lethal 
ones are in serious development (e.g.\ combat UAVs for suppression of enemy air
defenses)~\cite{af95-vistas}.
\index{defense systems}
\CDS{} design efforts for such applications must necessarily tackle the entire
problem, including the traditional inner loops, the vehicle management
functions, 
\index{vehicle management}
and even the higher-level ``mission management'' functions
coordinating groups of vehicles intent on satisfying specified mission
objectives.
\index{mission management}

Today's engineering methods for designing the upper layers of this hierarchy
are far from formal and systematic.  In essence, they consist of collecting
long lists of logical if-then-else rules from experts, programming these
rules, and simulating their execution in operating environments.  Because the
logical rules provide no inherent smoothness (any state transition is
possible) only ^{simulation} can be used for evaluation and only exhaustive
simulation can guarantee good design properties.  Clearly, this is an
unacceptable circumstance---one where the strong system-theoretic background
and the tradition of rigor held by the \cds{} community can make substantial
contributions. 

\index{hybrid systems}
\index{dynamical systems}
One can speculate about the forms that improved theories and tools for
non-smooth (hybrid) dynamical systems might take.  For example, it may be
possible to impose formal restrictions on permitted logical operations, to
play a regularizing role comparable to laws of physics. If rigorously obeyed,
these restrictions could make resulting systems amenable to formal analyses
and proofs of desired properties.  This approach is similar to computer
language design, and provides support for one of the recommendations of this
report, 
namely that the \cds{} and computer science disciplines need to grow
their intimate interactions.  It is also likely that the traditional standards
of formal rigor must expand to firmly embrace computation, algorithmic
solutions, and heuristics. 
\index{heuristics}

\index{aircraft}
\index{auto-land}
\index{reliability}
\index{B-2}\index{F-16}
However, one must not ever lose sight of the key distinguishing features of the
\cds{} discipline, including the need for hard real time execution of control
laws and the need for ultra-reliable operation of all hardware and software
control components.  Many controlled systems today (auto-land systems of
commercial transports, launch boosters, ^{F-16} and ^{B-2} aircraft, certain
power 
plants, certain chemical process plants, etc.)\ fail catastrophically in the
event of control hardware failures, and many future systems, including the
unmanned vehicles mentioned above, share this property.  But the future of
aerospace and transportation holds still more complex challenges. We noted
above that changes in the underlying dynamics of control design problems
from continuous to hybrid are well under way.  An even more dramatic trend on
the horizon is a change in dynamics to large collections of distributed
entities with local computation, global communication connections, very little
regularity imposed by laws of physics, and no possibility to impose
centralized control actions.  Examples of this trend include the national
airspace management problem, automated highway and traffic management, and
command and control for future battlefields (Figure~\ref{battlespace.fig}).
\index{distributed control}
\index{battlefield management}
\index{intelligent highways}
\index{command and control}
\begin{figure}
  \centerline{\psfig{figure=battlespace.eps,width=0.9\figwidth}}
  \caption{Battle space management scenario illustrating distributed command
  and control between
  heterogeneous air and ground assets.
  Figure courtesy of DARPA. 
  }
  \label{battlespace.fig}
\end{figure}

\index{airspace management}%
\index{distributed control}
\index{air traffic control}
The national airspace problem is particularly significant today, with eventual
gridlock and congestion threatening the integrity of the existing air
transportation system.  Even with today's traffic, ground holds and airborne
delays in flights due to congestion in the skies have become so common that
airlines automatically pad their flight times with built-in delays.  The
structure of the air traffic control (ATC) system is partially blamed for
these delays: the control is distributed from airspace region to airspace
region, yet within a region the control is almost wholly centralized, with
sensory information from aircraft sent to a human air traffic controller who
uses ground-based navigation and surveillance equipment to manually route
aircraft along sets of well-traveled routes.  In today's system, bad weather,
aircraft failure, and runway or airport closure have repercussions throughout
the whole country.  Efforts are now being made to improve the current system
by developing cockpit ``{sensors}_{sensing}'' such as augmented GPS navigation
systems and
datalinks for aircraft to aircraft communication.  Along with these new
technologies, new hierarchical control methodologies are being proposed, which
automate some of the functionality of ATC.  
\index{hierarchical control}
This opens up a set of new
challenges: the design of information-sharing mechanisms and new, distributed,
{\em verified} embedded control schemes for separation assurance between
aircraft, and the design of dynamic air traffic network topologies which aid
in the safe routing of aircraft from origin to destination and which adapt to
different traffic flows, are two areas which provide a tremendous opportunity
to researchers in the \cds{} community.
\index{global positioning system (GPS)}
\index{verification and validation}

\index{automotive systems}
\index{hybrid cars}
\index{automobiles}
\index{inner loop}
\index{fuel cells}
Finally, it is important to observe that the future also holds many
applications that fall under the traditional control design paradigm, yet are
worthy of research support because of their great impact.  Conventional ``inner
loops'' in automobiles, but for non-conventional power plants, are examples.
Hybrid cars combining electrical drives with low-power internal combustion
engines and fuel cell powered cars combining electrical drives with fuel cell
generation both depend heavily of well-designed control systems to operate
efficiently and reliably.  Similarly, increased automation of traditional
transportation systems such as ships and railroad cars, with added
instrumentation and cargo-tracking systems will rely on advanced control and
schedule optimization to achieve maximum economic impact. Another conventional
area is general aviation, where control systems to make small aircraft easy
and safe_{safety} to fly and increased automation to manage them are essential
needs. 

\subsection*{Other Trends in Aerospace and Transportation}

In addition to the specific areas highlighted above, there are many
other trends in aerospace and transportation that will benefit from
and inform new results in \cds{}.  We briefly describe a few of these
here.

\index{automotive systems|(}
\index{microprocessors!use, in control}
\index{engine control}
\index{emissions control}
\paragraph{Automotive Systems} With 60 million vehicles produced each
year, automotive systems are a major application area for control.
Emission control regulations passed in the 1970s created a need for
more sophisticated engine control systems that could provide clean and
efficient operation in a variety of operating environments and over
the lifetime of the car.  The development of the microprocessor at
that same time allowed the implementation of sophisticated algorithms
that have reduced the emissions in automobiles by as much as a factor
of 50 from their 1970 levels.

Future automobile designs will rely even more heavily on electronic
controls~\cite{pn00-cep}.  Figure~\ref{futureauto.fig} shows some of
the components that are being considered for next generation vehicles.
\begin{figure}
  \centerline{\psfig{figure=futureauto.eps,height=4.3in}}
  \caption{Major future components for basic automotive vehicle
  functions~\cite{pn00-cep}.}
  \editor{update figure}
  \label{futureauto.fig}
\end{figure}
\index{radar}
\index{chassis control}
Many of these components will build on the use of \cds{} techniques,
including radar-based speed and spacing control systems, chassis
control technologies for stability enhancement and improved suspension
characteristics, active control of suspension and braking, and active
restraint systems for ^{safety}.  In addition, more sophisticated use of
networking and communications devices will allow enhanced energy
management between components and vehicle diagnostics with
owner/dealer notification.
\index{energy systems}
\index{networks}
\index{diagnostics}

\index{chassis control}
\index{intelligent highways}
\index{communications!systems}
These new features will require highly integrated control systems that
combine multiple components to provide overall stability and
performance.  Systems such as chassis control will require combining steering,
braking, 
powertrain and suspension subsystems, along with adding new sensors.
One can also imagine increased interaction between vehicles and the
roadway infrastructure, as automated highways and self-controlled
vehicles move from the research lab into applications.  These latter
applications are particularly challenging since they begin to link
heterogeneous vehicles through communications systems that will
experience varying bandwidths and latency (time delays) depending on the local
environment.  Providing safe, reliable, and comfortable operation for
such systems is a major challenge for \cds{} and one that will have
application in a variety of consumer, industrial, and military
applications. 
\index{automotive systems|)}
\index{time delay}

\index{propulsion systems}
\index{gas turbine engines}
\index{health status, system}
\index{asset management}
\paragraph{Aircraft Propulsion Systems} Much more effective use of
information in propulsion systems is possible as the price/performance
ratio of computation and sensing continues to drop. Intelligent turbine
engines will 
ultimately lower lifetime operating and maintenance costs, similar to current
and upcoming automotive systems. They will provide advanced health,
performance, and life 
management by embedding models of their operation and optimizing based
on condition and mission. 
\index{modeling!embedded}
They will be more flexible and more tolerant
of component faults, and will integrate into the owner�s asset
management system, lowering maintenance and fleet management costs by
making engine condition information available to the owner on demand
and ensuring predictable asset availability.
\index{fault tolerance}

\index{diagnostics}
\index{prognostics}
Detection of damage (diagnostics) and
prediction of the implications (prognostics) are the heart
of an intelligent engine.  Detailed modeling of the thermofluid,
structural, and mechanical systems, as well as the operational
environment, is needed for such assessments. To allow on-product use
accounting for system interactions, physics-based models will be
constructed using advanced techniques in reduced-order modeling_{model
reduction}. This 
approach significantly extends recent engine component modeling.

\index{modeling!embedded}
Embedded models can also be used for online optimization and control in real
time. The benefit is the ability to customize engine performance to changes in
operating conditions and the engine's environment through updates in the
cost function, onboard model, and constraint set.  Many of the challenges of
designing controllers that are robust to a large set of uncertainties can thus
be embedded in the online optimization, and robustness through a compromise
design is replaced by always-optimal performance.
\index{optimization!online}

\index{flow control}
\index{combustion systems}
\index{gas turbine engines}
\paragraph{Flow Control}
Flow control involves the use of reactive devices
for modifying fluid flow for the purposes of enhanced operability.  Sample
applications for flow control include increased lift and reduced drag on
aircraft wings, engine nacelles, compressor fan blades, and helicopter rotor
blades; higher performance diffusers in gas turbines, industrial heaters and
chillers, and engine inlets; wake management for reduction of resonant stress
and blade vortex interaction; and enhanced mixing for combustion and noise
applications.  A number of devices have been explored in the past several
years for actuation of flow fields.  
\index{actuation!flow control}%
These range from novel air injection
mechanisms for control of rotating stall and separation, to synthetic jets
developed for mixing enhancement and vectoring, to MEMS devices for modulating
boundary layers and flow around stagnation points.  In addition, new
^{sensing}
technology, such as micro anemometers, is also becoming available.
\index{MEMS}

\index{hypersonic flight systems}%
\index{control!as enabling technology}
These changes in sensing and actuation technology are enabling new
applications of \cds{} to unstable shear layers and separated flow,
thermoacoustic instabilities, and compression system instabilities such as
rotating stall and surge (see~\cite{Bew01-pas} for a recent survey).  An
emerging area of interest in hypersonic flight systems, where flow control
techniques could 
provide a larger toolbox for design of vehicles, including drag reduction,
novel methods for producing control forces, and better understanding of the
complex physical phenomena at these speeds.

\index{space systems}
\index{tracking}
\paragraph{Space Systems\protect\footnote{The Panel would like to thank
Jonathan How and Jesse Leitner for their contributions to this section.}}
The exploitation of space systems for civil, commercial, defense, scientific,
or intelligence purposes gives rise to a unique set of challenges in the area
of \cds{}.  For example, most space missions cannot be
adequately tested on the ground prior to flight, which has a direct impact on
many dynamics and control problems.  A three-pronged approach is required to
address these challenging space system problems: (1) detailed modeling,
including improved means of characterizing, at a very small scale, the
fundamental physics of the systems; (2) flight demonstrations to characterize
the behavior of representative systems; and (3) design of navigation and
control approaches that maintain ^{performance} (disturbance rejection and
tracking) 
even with uncertainties, failures, and changing dynamics.
\index{modeling!space systems}
\index{disturbance rejection}

\index{flexible structures}
\index{formation control}
There are two significant areas that can revolutionize the achievable
performance from future space missions: flexible structure analysis and
control, and space vehicle formation flying.  These both impact the allowable
size of the effective aperture, which influences the ``imaging'' performance,
whether it is optical imaging or the collection of signals from a wide range
of wavelengths. There are fundamental limitations that prevent further
developments with monolithic mirrors (with the possible exception of
inflatable and foldable membranes, which introduce their own extreme
challenges) and the various segmented approaches---deployed arrays, tethered
or freeflyer formations---provide the only solution.  However, these
approaches introduce challenging problems in characterizing the
realistic dynamics and developing sensing and control schemes to maintain the
necessary optical tolerances.

\index{flexible structures}
A significant amount of work has been performed in the area of flexible
structure dynamics and control under the auspices of the Strategic Defense
Initiative Organization (SDIO) in the 1970s and
80s. However, at the performance levels required for future missions
(nanometers), much research remains to develop models at the micro-dynamics
level and control techniques that can adapt to system changes at these small
scales.
\index{dynamics!space systems}

\index{formation control}
\index{networks}
\index{actuation!space systems}%
Similar problems exist with formation control for proposed imaging
^{interferometry} missions.  These will require the development of control
algorithms, actuators, and computation and communications
networks. Sensors_{sensing} 
will also have to be developed to measure deflections on the scale of
nanometers over distances hundreds of meters through kilometers.  Likewise,
actuation systems of various types must be developed that can control on the
scale of nanometers to microns with very low noise levels and fine resolution.
The biases and residuals generally accepted due to particular approximations
in navigation and control algorithms will no longer be acceptable.
Furthermore, the ^{simulation} techniques used for verification must, in some
cases, maintain precision through tens of orders of magnitude separation in
key states and parameters, over both long and short time-scales, and with
stochastic noise inputs.  
\index{verification and validation}
In summary, in order to enable the next generations
of advanced space systems, the field must address the micro- and nanoscale
problems 
in analysis, sensing, control, and simulation, for individual elements and
integrated systems.

\index{aerospace systems|)}
\index{transportation systems|)}

\clearpage
%% \input{infonet.tex}
% Master File: cdspanel.tex

\section{Information and Networks}
%
% Pictures: Internet, supply chain, wireless testbed
%

\index{Jacobson, V.}
\begin{itemize}
  \small\item[] {\em A typical congested gateway looks like a fire hose
  connected to a soda straw through a small funnel (the output queue).  If, on
  average, packets arrive faster than they can leave, the funnel will fill up
  and eventually overflow. RED [Random Early Detection] is [a] simple
  regulator_{regulation} that monitors the level in 
  the funnel and uses it to match the input rate to the output (by dropping
  excess traffic). As long as its control law is monotone non-decreasing and
  covers the full range of 0 to 100\% drop rate, RED works for {\em any} link,
  {\em any} bandwidth, {\em any} type of traffic.}

  Van Jacobson, North American Network Operators' Group meeting,
  1998~\cite{Jac98-nanog}.
\end{itemize}

\index{control!challenges}
The rapid growth of communications networks provides several major
opportunities and challenges for \cds{}.  Although there is overlap, we can
divide these roughly into two main areas: control of networks and control over
networks.

\subsection*{Control of Networks}
\index{networks!control of|(}
\index{networking!integration of control and}

Control of networks is a large area, spanning many topics, a few of which are
briefly described here.  The basic problems in control of networks include
controlling congestion across network links, routing the flow of packets
through the network, caching and updating data at multiple locations, and
managing power levels for wireless networks.

\index{Internet}
\index{decentralized control}
Several features of these control problems make them very challenging.  The
dominant feature is the extremely large scale of the system; the Internet is
probably the largest feedback control system man has ever built.  Another is
the decentralized nature of the control problem: local decisions must be made
quickly, and based only on local information.  Stability is complicated by the
presence of varying time lags, as information about the network state can only
be observed or relayed to controllers after a time delay, and the effect of a
local control action can be felt throughout the network after
substantial delay.  
\index{time delay}
\index{uncertainty!in networks}
Uncertainty and variation in the network, through network
topology, transmission channel characteristics, traffic demand, available
resources, etc., may change constantly and unpredictably.  Another
complicating issue is the diverse traffic characteristics, in terms of
arrival statistics at both the packet and flow time scales, and different
requirements for quality of service, in terms of delay, bandwidth, and loss
probability, that the network must support.

\index{resource allocation}
Resources that must be managed in this environment include computing, storage
and transmission capacities at end hosts and routers.
Performance_{performance} of such
systems is judged in many ways: throughput, delay, loss rates, fairness,
reliability, as well as the speed and quality with which the network adapts to
changing traffic patterns, changing resource availability, and changing
network congestion.

To illustrate these characteristics, we briefly describe the control
mechanisms that can be invoked in serving a file request from a client:
network caching, congestion control, routing and power control.
Figure~\ref{uunet.fig} shows a typical map for the networking infrastructure
that is used to process such a request.
\begin{figure}
  \centerline{\epsfig{figure=uunet.eps,width=0.9\figwidth}}
  \caption{UUNET network backbone for North America.  Figure courtesy
  WorldCom.} 
  \label{uunet.fig}
\end{figure}

\index{optimal network caching}
\index{time delay}
\index{caching, optimal network}
The problem of optimal network caching is to copy documents (or services) that
are likely to be accessed often, from many different locations, on multiple
servers.  When the document is requested, it is returned by the nearest
server.  Here, proximity may be measured by geographical distance, hop count,
network congestion, server load or a combination.  The goal is to reduce
delay, relieve server load, balance network traffic, and improve service
reliability.  If changes are made to the source document, those changes (at a
minimum) must be transmitted to the servers, which consume network bandwidth.

\index{decentralized control}
The \cds{} problem is to devise a decentralized scheme for how often to
update, where to cache copies of documents, and to which server a client
request is directed, based on estimation and prediction of access patterns,
network congestion, and server load.  Clearly, current decisions affect the
future state, such as future traffic on links, future buffer levels, delay and
congestion, and server load.  Thus a web of caches is a decentralized
feedback system that is spatially distributed and interconnected, where
control decisions are made asynchronously based on local and delayed
information.
\index{asynchronous systems}

\index{adaptation}
\index{packet-based systems}
When a large file is requested, the server that is selected to return the file
breaks it into a stream of packets and transports them to the client in
a rate-adaptive manner.  This process is governed by the Transport Control
Protocol (TCP).  
\index{transport control protocol (TCP)}
\index{protocols}
\index{TCP|see{transport control protocol}}
The client acknowledges successful reception of each packet
and the stream of acknowledgment carries congestion information to the server.
\index{congestion control}
\index{distributed control}
Congestion control is a distributed algorithm to share network resources among
competing servers.  It consists of two components: a source algorithm that
dynamically adjusts the server rate in response to congestion in its path, and
a router algorithm_{routing algorithms, for networks} that updates a congestion
measure and sends it back to
sources that go through that router.  Examples of congestion measures are loss
probability and queuing delay.  They are implicitly updated at the routers
and implicitly fed back to sources through delayed end-to-end observations of
packet loss or delay.  The equilibrium and dynamics of the network depends on
the pair of source and router algorithms.
\index{dynamics!networks}

A good way to understand the system behavior is to regard the source rates as
primal variables and router congestion measures as dual variables, and the
process of congestion control as an asynchronous distributed primal-dual
algorithm carried out by sources and routers over the Internet in real time to
maximize aggregate source utility subject to resource capacity constraints.
Different protocols all solve the same prototypical problem, but they use
different utility functions and implement different iterative rules to optimize
them.  Given any source algorithm, it is possible to derive explicitly the
utility function it is implicitly optimizing.

\index{Internet protocol (IP)}
\index{protocols}
\index{IP|see{Internet protocol}}
\index{routing algorithms, for networks}
\index{asynchronous systems}
While TCP controls the rate of a packet flow, the path through the network is
controlled by the Internet Protocol (IP).  In its simplest form, each router
must decide which output link a given packet will be sent to on its way to
its final destination.
\index{uncertainty!in networks}
\index{time delay}
\index{distributed control}
Uncertainties include varying
link congestion, delays, and rates, and even varying network topology (e.g., a
link goes down, or new nodes or links become available), as well as future
traffic levels.  A routing algorithm is an asynchronous distributed algorithm
executed at routers that adapts to node and link failures, balances network
traffic and reduces congestion.  It can be decomposed into several time
scales, with very fast decisions made in hardware using lookup tables, which
in turn are updated on a slower time scale.  At the other extreme in time
scale from the routing problem, we have optimal network planning, in which new
links and nodes are proposed to meet predicted future traffic demand.

\index{wireless networks}
\index{load balancing, in networks}
The routing problem is further exacerbated in wireless networks. Nodes with
wireless modems may be mobile, and the address of a node may neither indicate
where it is located nor how to reach it. Thus the network needs to either
search for a node on demand, or it must keep track of the changing locations
of nodes.  Further, since link capacities in wireless networks may be scarce,
routing may have to be determined in conjunction with some form of load
balancing. This gives rise to the need for distributed asynchronous algorithms
which are adaptive to node locations, link failures, mobility, and changes in
traffic flow requirements.

\index{ad hoc networks@{\it ad hoc} networks}
\index{wireless networks}
\index{power control, in communications}
Finally, if the client requesting the file accesses it through an
{\it ad hoc} wireless network, then there also arises the problem of power
control: at what transmission power level should each packet broadcast be
made?  Power control is required because {\it ad hoc} networks do not come
with ready made links; the topology of the network is formed by individual
nodes choosing the power levels of their broadcasts.  This poses a conceptual
problem in the current protocol_{protocols} hierarchy of the Internet since it
simultaneously affects the physical layer due to its effect on signal quality,
the network layer since power levels determine which links are available for
traffic to be routed, and the transport layer since power levels of broadcasts
affect congestion. Power control is also a good challenge for multi-objective
control since there are many cost criteria involved, such as increasing the
traffic carrying capacity of the network, reducing the battery power used in
relaying traffic, and reducing the contention for the common shared medium by
the nodes in geographical vicinity.

\index{air traffic control}
Control of networks extends beyond data and communication networks.  Optimal
routing and flow control of commercial aircraft (with emphasis on guaranteeing
safe inter-vehicle distances) will help maximize utilization of airports.  The
(network and software) infrastructure for supply chain systems is being built
right now, and simple automated supply chain management systems are beginning
to be deployed.  In the near future, sophisticated optimization and control
methods can be used to direct the flow of goods and money between suppliers,
assemblers and processors, and customers.
\index{supply chains}
\index{networks!control of|)}

\subsection*{Control over Networks}
\index{networks!control over|(}

\index{automobiles}
\index{sensing!sensor webs}
\index{intelligent highways}
\index{supply chains}
\index{logistics}
\index{enterprise level systems}
While the advances in information technology to date have led to a global
Internet that allows users to exchange information, it is clear that the next
phase will involve much more interaction with the physical environment.
Networks of sensory or actuator nodes with computational capabilities,
connected wirelessly or by wires, can form an orchestra which controls our
physical environment.  Examples include automobiles, smart homes, large
manufacturing systems, intelligent highways and networked city services, and
enterprise-wide supply and logistics chains.  Thus, this next phase of the
information technology
revolution is the convergence of communication, computing and control.  The
following vignette describes a major architectural challenge in achieving this
convergence.

%% \input{kumar-comms.tex}
\index{Kumar, P.~ R.}
\index{abstraction}
\index{communications!integration of control and}
\index{computation!integration of control and}
\index{networking!integration of control and}

\vignette{The importance of abstractions and architecture for the
convergence of communications, computing, and control (P. R. Kumar,
Univ. of Illinois, Urbana-Champaign)}
{
\index{Internet}
Communication networks are very diverse, running over copper, radio, or
optical links, various computers, routers, etc.  However, they have an
underlying architecture which allows one to just plug-and-play, and not
concern oneself with what lies underneath.  In fact, one reason for the
anarchic proliferation of the Internet is precisely this architecture---a
hierarchy of layers together with peer-to-peer ^{protocols} connecting the
layers 
at different nodes.  On one hand, nodes can be connected to the Internet
without regard to the physical nature of the communication link, whether it be
infrared or copper, and this is one reason for the tremendous growth in the
number 
of nodes on the Internet.  
\index{plug-and-play architecture}
On the other hand, the architecture allows
plug-and-play at all levels, and thus each layer can be designed separately,
allowing a protocol at one level to be modified over time without
simultaneously necessitating a redesign of the whole system.  This has
permitted the Internet protocols to evolve and change over time.

This raises the issue: What is the right architecture for the convergence of
communication, control, and computing?  Is there an architecture which is
application and context independent, one which allows proliferation, just as
the Open Systems Interconnect (OSI) architecture did for communication
networks?  What are the right 
abstraction layers? How does one integrate information, control, and
computation?  If the overall design allows us to separate algorithms from
architecture, then this convergence of control with communication and
computation will rapidly proliferate.  
}

As existing networks continue to build out, and network technology becomes
cheaper and more reliable than fixed point-to-point connections, even in small
localized systems, more and more control systems will operate over networks.
We can foresee sensor, actuator, diagnostic, and command and coordination
signals all traveling over data networks.  The estimation and control
functions can be distributed across multiple processors, also linked by data
networks.  (For example, smart sensors_{sensing!in communications networks}
can perform substantial local signal 
processing before forwarding relevant information over a network.)

\index{communications!systems}
\index{uncertainty!component or parameter variation}
\index{time delay}
Current control systems are almost universally based on synchronous, clocked
systems, so they require communications networks that guarantee delivery of
sensor, actuator, and other signals with a known, fixed delay.  While current
control systems are robust to variations that are included in the design
process (such as a variation in some aerodynamic coefficient, motor constant,
or moment of inertia), they are not at all tolerant of (unmodeled)
communication delays, or dropped or lost sensor or actuator packets.  Current
control system technology is based on a simple communication architecture: all
signals travel over synchronous dedicated links, with known (or worst-case
bounded) delays, and no packet loss.  Small dedicated communication networks
can be configured to meet these demanding specifications for control systems,
but a very interesting question is:
\begin{quote}
  Can one develop a theory and practice for control systems that operate in a
  distributed, asynchronous, packet-based environment?
\end{quote}
\index{asynchronous systems}
\index{packet-based systems}
\index{distributed control}

\index{self-configuring systems}
\index{component failures, robustness to}
It is very interesting to compare current control system technology with
current packet-based data networks.  Data networks are extremely robust to
gross, unpredicted changes in topology (such as loss of a node or a link);
packets are simply re-sent or re-routed to their destination.  Data networks
are self-configuring: we can add new nodes and links, and soon enough packets
are flowing through them.  One of the amazing attributes of data networks is
that, with good architecture and protocol_{protocols} design, they can be far
{more} 
reliable_{reliability} than their components.  This is in sharp contrast with modern control
systems, which are only as reliable as their weakest link.  Robustness to
component failure must be designed in, by hand (and is, for ^{safety} critical
systems).

\index{packet-based systems}
\index{distributed control}
Looking forward, we can imagine a marriage of current control systems and
networks.  The goal is an architecture, and design and analysis methods, for
distributed control systems that operate in a packet-based network.  If this
is done correctly, we might be able to combine the good qualities of a robust
control system, i.e., high performance and robustness to parameter variation
and model mismatch, with the good qualities of a network: self-configuring,
robust to gross topology changes and component failures, and reliability
exceeding that of its components.
\index{reliability}

\index{sensing}
\index{time delay}
One can imagine systems where sensors asynchronously burst packets onto the
network, control processors process the data and send it out to actuators.
Packets can be delayed by varying amounts of time, or even lost.
Communication links can go down, or become congested.  Sensors and actuators
themselves become unavailable or available.  New sensors, actuators, and
processors can be added to the system, which automatically reconfigures itself
to make use of the new resources.  As long as there are enough sensors and
actuators available, and enough of the packets are getting though, the whole
system works (although we imagine not as well as with a dedicated, synchronous
control system).  This is of course very different from any existing current
high performance control system.

It is clear that for some applications, current control methods, based on
synchronous clocked systems and networks that guarantee arrival and bounded
delays for all communications, are the best choice.  There is no reason not to
configure the controller for a jet engine as it is now, i.e., a synchronous
system with guaranteed links between sensors, processors, and actuators.  But
for consumer applications not requiring the absolute highest performance, the
added robustness and self-reconfiguring_{self-configuring systems} abilities
of a
packet-based control 
system could make up for the lost performance.  In any case what will emerge
will probably be something in between the two extremes, of a totally
synchronous system and a totally asynchronous packet-based system.

\index{Bode plots}
\index{amplifiers}
\index{transfer functions}
Clearly, several fundamental control concepts will not make the transition to
an asynchronous, packet-based environment.  The most obvious casualty will be
the transfer function, and all the other concepts associated with linear time
invariant (LTI) systems (impulse and step response, frequency response,
spectrum, bandwidth, etc.)  This is not a small loss, as this has been a
foundation of control engineering since about 1930.  With the loss goes a lot
of intuition and understanding.  For example, Bode plots were introduced in
the 1930s to understand and design feedback amplifiers, were updated to handle
discrete-time control systems in the 1960s, and were applied to robust MIMO
control systems in 
the 1980s (via singular value plots).  Even the optimal control methods in the
1960s, which appeared at first to be quite removed from frequency domain
concepts, were shown to be nicely interpreted via transfer functions.
\index{frequency domain control}

\index{Lyapunov functions}
So what methods will make the transition?  Many of the methods related to
optimal control and optimal dynamic resource allocation will likely transpose
gracefully to an asynchronous, packet-based environment.  A related concept
that is likely to survive is also one of the oldest: Lyapunov functions (which
were introduced in 1890).  The following vignette describes some of the
possible changes to \cds{} that may be required.

%% \input{boyd-lyapunov.tex}
\index{Boyd, S. P.}
\vignette{Lyapunov Functions in Networked Environments (Stephen Boyd,
Stanford)}
{
Here is an example of how an ``old'' concept from control will update
gracefully.  The idea is that of the Bellman value function, which gives the
optimal value of some control problem, posed as an optimization problem, as a
function of the starting state.  It was studied by Pontryagin, Bellman, and
other 
pioneers of optimal control in the 1950s, and has recently had a resurgence
(in generalized form) under the name of control Lyapunov function.  It is a
key concept in dynamic programming.
\index{Lyapunov functions}
\index{control Lyapunov functions}
\index{value function}
\index{Bellman, R. E.}
\index{Pontryagin, L.\ S.}

The basic idea of a control Lyapunov function (or the Bellman value function)
is this: If you knew the function, then the best thing to do is to choose
current actions that minimize the value function in the current step, without
any regard for future effects.  (In other words, we ignore the dynamics of the
system.)  By doing this we are actually carrying out an optimal control for
the problem.  In other words, the value function is the cost function whose
greedy minimization actually yields the optimal control for the original
problem, taking the system dynamics into account.  In the work of the 1950s
and 60s, the value function is just a mathematical stepping stone toward the
solution of optimal control problems.

\index{asynchronous systems}
But the idea of value function transposes to an asynchronous system very
nicely.  If the value function, or some approximation, were broadcast to the
actuators, then each actuator could take independent and separate action,
i.e., each would do whatever it could to decrease the value function.  If the
actuator were unavailable, then it would do nothing.  In general the actions
of multiple actuators has to be carefully coordinated; simple examples show
that turning on two feedback systems, each with its own sensor and actuator,
simultaneously, can lead to disastrous loss of performance, or even
instability.  But if there is a value or control Lyapunov function that each
is separately minimizing, everything is fine; the actions are automatically
coordinated (via the value function).
\index{control Lyapunov functions}
}

\index{model predictive control}
\index{packet-based systems}
\index{optimization!online}
Another idea that will gracefully extend to asynchronous packet-based control
is model predictive control.  The basic idea is to carry out far more
computation at run time, by solving optimization problems in the real-time
feedback control law.  Model predictive control has played a major role in
^{process control}, and also in supply-chain management, but not (yet) in other
areas, mainly owing to the very large computational burden it places on the
controller implementation.  The idea is very simple: at each time step we
formulate the optimal control problem, up to some time horizon in the future,
and solve for the whole optimal trajectory (say, using quadratic programming).
We then use the current optimal input as the actuator signal.  The
sensor signals can be used to update the model, and carry the same process out
again.  A 
major extension required to apply model predictive control in networked
environments would be the distributed solution of the underlying optimization
problem.
\index{networks!control over|)}

\subsection*{Other Trends in Information and Networks}

While we have concentrated in this section on the role of control in
communications and networking, there are many problems in the broader field of
information science and technology for which control ideas will be important.
We highlight a few here; more information can also be found in a recent
National Research Council report on embedded systems~\cite{nrc01-embedded}.
\index{software!embedded}

\index{adversarial environments}
\index{vigilant software}
\index{software!systems}
\index{high confidence systems}
\paragraph{Vigilant, high confidence software systems} Modern
information systems are required to operate in environments where the
users place high confidence on the availability and correctness of the
software programs.  This is increasingly difficult due to the
networked and often adversarial environment in which these programs
operate.  One approach that is being explored by the computer science
community is to provide confidence through {\em vigilance}.  Vigilance
refers to continuous, pervasive, multi-faceted monitoring and
correction of system behavior, i.e., \cds{}.

\index{modeling!embedded}
The key idea in vigilant software is to use fast and accurate
sensing_{sensing!in software systems}
to monitor the execution of a system or algorithm, compare the
performance of the algorithm to an embedded model of the computation,
and then modify the operation of the algorithm (through adjustable
parameters) to maintain the desired performance.  This
``sense-compute-act'' loop is the basic paradigm of feedback control
and provides a mechanism for online management of uncertainty.  
\index{uncertainty management}
Its
power lies in the fact that rather than considering every possible
situation at design time, the system reacts to specific situations as they
occur.  An essential element of the strategy is the use of either an
embedded model, through which an appropriate control action can be
determined, or a predefined control strategy that is analyzed offline
to ensure stability, performance, and robustness.

\index{sorting}
\index{feedback}
As an indication of how vigilance might be used to achieve high confidence,
consider an example of feedback control for distributed sorting, as shown in
Figure~\ref{vhcssort.fig}.  
\begin{figure}
  \centerline{\psfig{figure=vhcssort.eps,width=0.9\figwidth}}
  \caption{An example of a vigilant high confidence software system:
  distributed sorting using feedback.}
  \label{vhcssort.fig}
\end{figure}
We envision a situation in which we have a collection of partial sort
algorithms that are interconnected in a feedback structure.  Suppose that each
sorter has multiple inputs, from which it chooses the best sorted list, and a
single output, to which it sends an updated list that is more ordered.  By
connecting these modules together in a feedback loop, it is possible to get a
completely sorted list at the end of a finite number of time steps.

\index{self-configuring systems}
While unconventional from a traditional computer science perspective, this
approach gives robustness to failure of individual sorters, as well as
self-reconfiguring operation.  Robustness comes because if an individual
module {\em 
unsorts} its data, this data will not be selected from the input streams by
the other modules.  Further, if the modules have different loads (perhaps due
to other processing being done on a given processor), the module with the most
time available will automatically take on the load in performing the
distributed sorting.  Other properties such as disturbance rejection,
performance, and stability could also be studied by using tools from \cds.
\index{disturbance rejection}

\index{verification and validation}
\index{software!systems}
\index{protocols}
\index{formal methods}
\paragraph{Verification and validation of protocols and software} The
development of complex software systems is increasing at a rapid rate and our
ability to design such systems so that they give provably correct performance
is increasingly strained.  Current methods for verification and validation of
software systems require large amounts of testing and many errors are not
discovered until late stages of development or even product release.  Formal
methods for verification of software are used for systems of moderate
complexity, but do not scale well to large software systems.

\index{optimization}
Control theory has developed a variety of techniques for giving provably
correct behavior by using upper and lower bounds to effectively break
computational complexity bounds.  Recent results in convex optimization of
semialgebraic problems (those that can be expressed by polynomial equalities
and inequalities) are providing new insights into verification of a diverse
set of continuous and combinatorial optimization problems~\cite{Par00-phd}.
In particular, these new techniques allow a systematic search for ``simple
proofs'' of mixed continuous and discrete problems and offer ideas for
combining 
formal methods in computer science with stability and robustness results in
control.
\index{hybrid systems}
\index{formal methods}

\index{supply chains}
\index{enterprise level systems}
\index{reconfigurable systems}
\index{HVAC systems}
\index{energy systems}
\index{real-time systems}
\index{buildings, control of}
\index{electrical power}
\index{asset management}
\paragraph{Real-time supply chain management} As increasing numbers of
enterprise systems are connected to each other across networks, there is an
enhanced ability to perform enterprise level, dynamic reconfiguration of high
availability assets for achieving efficient, reliable, predictable operations.
As an example of the type of application that one can imagine, consider the
operation of a network of HVAC systems for a regional collection of buildings,
under the control of a single operating company.  In order to minimize overall
energy costs for its operation, the company makes a long-term arrangement with
an energy broker to supply a specified amount of electrical power that will be
used to heat and cool the buildings.  In order to get the best price for the
energy it purchases, the company agrees to purchase a fixed amount of energy
across its regional collection of buildings and to pay a premium for energy
usage above this amount.  This gives the energy broker a fixed income as well
as a fixed (maximum) demand, for which it is willing to sell electricity at a
lower price (due to less uncertainty in future revenue as well as system
loading).

\index{uncertainty!in resource allocation}
\index{command and control}
\index{distributed control}
Due to the uncertainty in the usage of the building, the weather in different
areas across the region, and the reliability of the HVAC subsystems in the
buildings, a key element in implementing such an arrangement is a distributed,
real-time_{real-time systems} command and control system capable of performing
distributed 
optimization of interconnected assets.  The power broker and the company must
be able to communicate information about asset condition and mission between
the control systems for their electrical generation and HVAC systems and the
subsystems must react to sensed changes in the environment (occupancy,
weather, equipment status) to optimize the fleet level performance of the
network. 
\index{optimization}

\index{enterprise level systems}
\index{modeling!embedded}
\index{fault tolerance}
\index{networks!control over}
Realization of enterprise-wide optimization of this sort will require
substantial progress in a number of technical areas: distributed, embedded
modeling tools that allow low resolution modeling of the external system
combined with high resolution modeling of the local system, resident at each
node in the enterprise; distributed optimization algorithms that make use of
the embedded modeling architecture to produce near optimal operations;
fault tolerant, networked control systems that allow control loops to operate
across unreliable network connections; and low cost, fault tolerant,
reconfigurable hardware and software architectures.

\index{C4ISR} 
\index{command and control}
\index{adversarial environments}
\index{asset management}
A very closely related problem is that of C4ISR (command, control,
communications, computers, intelligence, ^{surveillance}, and
^{reconnaissance}) in 
military systems.  Here also, networked systems are revolutionizing the
capabilities for continuous planning and asset allocation, but new research is
needed in providing robust solutions that give the required performance in the
presence of uncertainty and adversaries.  
\index{uncertainty!in resource allocation}
The underlying issues and techniques
are almost identical to enterprise level resource allocation, but the
environment in which they must perform is much more challenging for military
applications.  \CDS{} concepts will be essential tools for providing robust
performance in such dynamic, uncertain, and adversarial environments.
\index{adversarial environments}

\clearpage
%% \input{robotics.tex}
% Master File: cdspanel.tex
\section{Robotics and Intelligent Machines}
\label{robotics.sec}

\index{intelligent machines|see{robotics}}
\index{machines, intelligent|see{robotics}}
\index{robotics|(}

\begin{itemize}
  \small\item[]
  {\em It is my thesis that the physical functioning of the living individual
  and the operation of some of the newer communication machines are precisely
  parallel in their analogous attempts to control entropy through
  feedback. Both of them have sensory_{sensing!in biological systems} receptors
  as one stage in their cycle of 
  operation: that is, in both of them there exists a special apparatus for
  collecting information from the outer world at low energy levels, and for
  making it available in the operation of the individual or of the machine. In
  both cases these external messages are not taken neat, but through the
  internal transforming powers of the apparatus, whether it be alive or
  dead. The information is then turned into a new form available for the
  further stages of performance. In both the animal and the machine this
  performance is made to be effective on the outer world. In both of them,
  their performed action on the outer world, and not merely their intended
  action, is reported back to the central regulatory apparatus.} 

  Norbert Wiener, from {\it The Human Use of Human Beings: Cybernetics
  and   Society}, 1950~\cite{Wei50}.
  \index{Wiener, Norbert}%
  \index{feedback}
\end{itemize}

\index{manufacturing}
Robotics and intelligent machines refer to a collection of applications
involving the development of machines with human-like behavior.  While early
robots were primarily used for manufacturing, modern robots include wheeled
and legged machines capable of participating in robotic competitions and
exploring 
planets, unmanned aerial vehicles for surveillance and combat, 
\index{unmanned vehicles}
and medical
devices that provide new capabilities to doctors.  Future applications will
involve both increased autonomy and increased interaction with humans and with
society. \CDS{} is a central element in all of these applications and will be
even more important as the next generation of intelligent machines are
developed.

\subsection*{Background and History}
\index{control!history of!robotics}
\index{robotics!history of}

\index{cybernetics}
\index{Wiener, Norbert}
The goal of cybernetic engineering, already articulated in the 1940s and even
before, has been to implement systems capable of exhibiting highly flexible or
``intelligent'' responses to changing circumstances.  In 1948, the MIT
mathematician Norbert Wiener gave a widely read, albeit completely
non-mathematical, account of cybernetics~\cite{Wei48}.  A more mathematical
treatment of the elements of engineering cybernetics was presented by
H.~S.~Tsien
\index{Tsien, H.~S.}%
in 1954, driven by problems related to control of missiles~\cite{Tsi54}.
Together, these works and others of that time form much of the intellectual
basis for modern work in robotics and control.
\index{missiles}

\index{computer numerically controlled (CNC) machining}%
\index{World War II}
\index{master-slave mechanisms}%
The early applications leading up to today's robotic systems began after World
War II with the development of remotely controlled mechanical manipulators,
which used master-slave mechanisms.  Industrial robots followed shortly
thereafter, starting with early innovations in computer numerically controlled
(CNC) machine tools.  Unimation, one of the early robotics companies,
installed its first robot in a General Motors plant in 1961.
Sensory_{sensing!in robotic systems} systems
were added to allow robots to respond to changes in their environment and by
the 1960s many new robots were capable of grasping, walking, seeing (through
binary vision), and even responding to simple voice commands.

\index{manufacturing!robots for}%
\index{automation|see{manufacturing}}
\index{computer science}
\index{manufacturing}
\index{artificial intelligence}
The 1970s and 80s saw the advent of computer controlled robots and the field
of robotics became a fertile ground for research in computer science and
mechanical engineering.  Manufacturing robots became commonplace (led by
Japanese companies) and a variety of tasks ranging from mundane to high
precision, were undertaken with machines.  Artificial intelligence
(AI) techniques were also developed to allow higher level reasoning, including
attempts at interaction with humans.  At about this same time, new research
was undertaken in mobile robots for use on the factory floor and remote
environments.
\index{reasoning, higher level}

\index{Mars rover}
\index{Sony AIBO}
\index{AIBO robot}%
\index{control!successes of}
Two accomplishments that demonstrate the successes of the field are the Mars
Sojourner robot and the Sony AIBO Entertainment Robot, shown in Figure~\ref{robots.fig}.
\begin{figure}
  \centerline{
    \begin{tabular}{cc}
      \psfig{figure=sojourner.eps,width=0.4\figwidth} &
      \quad\psfig{figure=aibo.ps,width=0.5\figwidth} \\
      (a) & (b)
    \end{tabular}
  }
  \caption{(a) The Mars Sojourner rover and (b) Sony AIBO Entertainment
    Robot. Photographs courtesy of Jet Propulsion Laboratory and Sony
    Electronics Inc.  
}
  \label{robots.fig}
\end{figure}
Sojourner successfully maneuvered on the surface of Mars for 83 days starting
in July 1997 and sent back live pictures of its environment.  The Sony AIBO
robot debuted in June of 1999 and was the first ``entertainment'' robot that
was mass marketed by a major international corporation.  It was particularly
noteworthy because of its use of AI technologies that allowed it to act in
response to external stimulation and its own judgment.
\index{artificial intelligence}
\index{entertainment}

\index{Robotics and Automation Society, IEEE}%
\index{Computer Society, IEEE}
\index{Control Systems Society (CSS), IEEE}%
It is interesting to note some of the history of the \cds{} community in
robotics.  The IEEE Robotics and Automation Society was jointly founded in the
early 1980s by 
the Control Systems Society and the Computer Society, indicating the mutual
interest in robotics by these two communities.  Unfortunately, while many
\cds{} researchers were active in robotics, the \cds{} community did
not play a leading role in robotics research throughout much of the 1980s and
90s.  This was a missed opportunity since robotics
represents an important collection of applications that combines ideas from
computer science, artificial intelligence, and \cds.  New applications in
(unmanned) flight control, underwater vehicles, and satellite_{satellites}
systems are 
generating renewed interest in robotics and many \cds{} researchers are
becoming active in this area.
\index{artificial intelligence}
\index{unmanned vehicles}
\index{underwater vehicles}
\index{satellites}


\index{adaptation}%
\index{reasoning, higher level}
\index{artificial intelligence}
Despite the enormous progress in robotics over the last half century, the
field is very much in its infancy.  Today's robots still exhibit extremely
simple behaviors compared with humans and their ability to locomote, interpret
complex sensory inputs, perform higher level reasoning, and cooperate together
in teams is limited.  
\index{teams!control of}
\index{Wiener, Norbert}
Indeed, much of Wiener's vision for robotics and
intelligent machines remains unrealized.  While advances are needed in many
fields to achieve this vision---including advances in sensing, actuation, and
energy storage---the opportunity to combine the advances of the AI community
in planning, adaptation, and learning with the techniques in the \cds{}
community for modeling, analysis, and design of feedback systems presents a
renewed path for progress.
\index{sensing!in robotic systems}
\index{actuation}
This application area is strongly linked with the
Panel's recommendations on the integration of computing, communication and
control, development of tools for higher level reasoning and decision making,
and maintaining a strong theory base and interaction with mathematics.
\index{learning}%
\index{adaptation}
\index{reasoning, higher level}
\index{decision making!higher level}

\subsection*{Challenges and Future Needs}
\index{control!challenges}

\index{automotive systems}
\index{automobiles}
\index{manufacturing}
\index{autonomous systems}
The basic electromechanical engineering and computing capabilities
required to build practical robotic systems have evolved over the last
half-century to the point where today there exist rapidly expanding
possibilities for making progress toward the long held goals of intelligence
and autonomy.  The implementation of principled and moderately sophisticated
algorithms is already possible on available computing hardware and more
capability will be here soon.  The successful demonstration of vision guided
automobiles operating at high speed, 
\index{vision guided systems}
the use of robotic devices in
manufacturing, and the commercialization of mobile robotic devices attest to
the practicality of this field.
\index{control!successes of}


\index{computer science}
\index{psychology}
\index{neuroscience}
Robotics is a broad field; the perspectives afforded by computer science,
control, electrical engineering, mechanical engineering, psychology, and
neuroscience all yield important insights.  Even so, there are pervasive
common threads, such as the understanding and control of spatial relations and
their time evolution.  The emergence of the field of robotics has provided the
occasion to analyze, and to attempt to replicate, the patterns of movement
required to accomplish useful tasks.  On the whole, this has been a sobering
experience.  Just as the ever closer examination of the physical world
occasionally reveals inadequacies in our vocabulary and mathematics,
roboticists have found that it is quite awkward to give precise, succinct
descriptions of effective movements using the syntax and semantics in common
use.  Because the motion generated by a robot is usually its {\it raison
d'etre}, it is logical to regard motion control as being a central problem.
Its study has raised several new questions for the control engineer relating
to the major themes of feedback, stability, optimization, and estimation.  
\index{motion control}
For
example, at what level of detail in modeling (i.e.\ kinematic or dynamic,
linear or nonlinear, deterministic or stochastic, etc.)\ does optimization
enter in a meaningful way?  Questions of coordination, sensitivity reduction,
stability, etc.\ all arise.
\index{modeling!robotics}

\index{software!motion control}
In addition to these themes, there is the need for development of appropriate
software for controlling the motion of these machines.  At present there is
almost no transportability of robotic motion control languages.  The idea of
vendor independent languages that apply with no change to a wide range of
computing platforms and peripherals has not yet been made to work in the field
of robotics.  The clear success of such notions when applied to operating
systems, languages, networks, disk drives, and printers makes it clear that
this is a major stumbling block.  What is missing is a consensus about how one
should structure and standardize a ``motion description language.''  Such a
language should, in addition to other things, allow one to implement
compliance control in a general and natural way.

\index{adaptation}%
\index{learning}%
Another major area of study is adaptation and learning.  As robots become
more commonplace, they will need to become more sophisticated in the way they
interact with their environment and reason about the actions of themselves and
others.  The robots of science fiction are able to learn from past experience,
interact with humans_{man-machine systems} in a manner that is dependent on
the situation, and 
reason about high level concepts to which they have not been previously
exposed.  In order to achieve the vision of intelligent machines that are
common in our society, major advances in machine learning and cognitive
systems will be required.  Robotics provides an ideal testbed for such
advances: applications in remote surveillance, search and rescue,
entertainment, and personal assistance are all fertile areas
for driving forward the state of the art.
\index{entertainment}

\index{cooperative control}
\index{defense systems}
\index{teams!control of}
In addition to better understanding the actions of individual robots, there is
also considerable interest and opportunity in cooperative control of teams of
robots.  The U.S.\ military is considering the use of multiple vehicles
operating in a coordinated fashion for surveillance, logistical support, and
combat, to offload the burden of dirty, dangerous, and dull missions from
humans.  Over the past decade, several new competitions have been developed in
which teams of robots compete against each other to explore these concepts.
Perhaps the best known of these is RoboCup, which is described briefly in the
following vignette.
\index{surveillance}
\index{logistics}

%% \input dandrea-robocup.tex
% Master File: cdspanel.tex
\begin{figure}
  \centerline{\epsfig{figure=robocup.eps,width=0.9\figwidth}}
  \caption{F180 league RoboCup soccer.  Photograph courtesy Raffaello
  D'Andrea.} 
  \label{robocup.fig}
\end{figure}

\index{D'Andrea, R.}
\index{RoboCup}
\index{adversarial environments}
\index{artificial intelligence}
\index{autonomous systems}
\vignette{
RoboCup---A testbed for autonomous collaborative
behavior in adversarial environments (Raffaello D'Andrea, Cornell 
University)}
{
\index{testbeds}
\index{robotics}
\index{competitions, student}
RoboCup is an international collection of robotics and artificial
intelligence (AI) competitions.  The competitions are fully autonomous (no
human intervention) head-to-head games, whose rules are loosely modeled after
the human game of soccer; each team must attempt to score more goals than the
opponent, subject to well defined rules and regulations (such as size
restrictions, collision avoidance, etc.)   The three main competitions are
known as the Simulation League, the F2000 League, and the F180 League,

\index{teams!control of}
The F180 League is played by 6 inch cube robots on a 2 by 3 meter table (see
Figure~\ref{robocup.fig}, and 
can be augmented by a global vision system; the addition of global vision
shifts the emphasis away from object localization and computer vision, to
collaborative team strategies and aggressive robot maneuvers.  
In what
follows, we will describe Cornell's experience in the F180 League at the 1999
competition in Stockholm, Sweden and the 2000 competition in Melbourne,
Australia.

Cornell was the winner of the F180 League in both 1999, the first year it
entered the competition, and 2000.
\index{systems engineering}
\index{dynamics}
The team's success can be directly attributed to the adoption of a
systems engineering approach to the problem, and by emphasizing system
dynamics and control.  The systems engineering approach 
was instrumental in the complete development of a competitive team in only 9
months (for the 1999 competition).  Twenty-five students, a mix of first year
graduate students and seniors representing computer science, electrical
engineering, and mechanical engineering, were able to construct two fully
operational teams by effective project management, by being able to capture
the system requirements at an early stage, and by being able to cross
disciplinary boundaries and communicate among themselves.   A hierarchical
decomposition was the means by which the problem complexity was rendered
tractable; in particular, the system was decomposed into estimation and
prediction, real time trajectory generation and control, and high level
strategy.
\index{hierarchical control}
\index{discipline boundaries}
\index{estimation}
\index{trajectory generation}

\index{time delay}
\index{latency|see{time delay}}
Estimation and prediction entailed relatively simple concepts from filtering,
tools known to most graduate students in the area of control.  In particular,
smoothing filters for the vision data and feedforward estimators to cope with
system latency were used to provide an accurate and robust assessment of the
game state.  Trajectory generation and control consisted of a set of
primitives that generated feasible robot trajectories;  various relaxation
techniques were used to generate trajectories that (1) could quickly be
computed in real time (typically less than 1000 floating point operations),
and (2) took full advantage of the inherent dynamics of the vehicles.  In
particular, feasible but aggressive trajectories could quickly be generated by
solving various relaxations of optimal control problems.  These primitives
were then used by the high level strategy, essentially a large state-machine.

\index{heuristics}
The high-level strategy was by far the most ad-hoc and heuristic component
of the Cornell RoboCup team.  The various functions that determined whether
passes and interceptions were possible were rigorous, in the sense that they
called upon the provably effective trajectory and control primitives, but the
high level strategies that determined whether a transition from defense to
offense should be made, for example, or what play should be executed, relied
heavily on human judgment and observation. As of March 2001,
most of the efforts at Cornell have shifted to understanding how the design
and verification of high level strategies that respect and fully utilize the
system dynamics can take place.    
\index{dynamics}
}


\index{communications!integration of control and}
\index{vision guided systems}
Certain robotic applications, such as those that call for the use of vision
systems to guide robots, 
now require the use of computing, communication and
control in an integrated way.  The computing that is to be done must be
opportunistic, i.e. it must be tailored to fit the needs of the specific
situation being encountered.  The data compression that is needed to transmit
television signals to a computer must be done with a view toward how the
results will be used by the control system.  It is both technologically
difficult and potentially dangerous to build complex systems that are
controlled in a completely centralized way.  For this reason we need to decide
how to distribute the control function over the communication system.  Recent
work on the theory of communication protocols has made available better
methods for designing efficient distributed algorithms.  This work can likely
be adapted in such a way as to serve the needs of robotic applications.
\index{protocols}
\index{distributed control}
\index{communications!systems}

\index{unstructured environments}
\index{reasoning, higher level}
Finally, we note the need to develop robots that can operate in highly
unstructured environments.  This will require considerable advances in visual
processing and understanding, complex reasoning and learning, and dynamic
motion planning and control.  Indeed, a framework for reasoning and planning
in these unstructured environments will likely require new mathematical
concepts that combine dynamics, logic, and geometry in ways that are not
currently available.  One of the major applications of such activities is in
the area of remote exploration (of the earth, other planets, and the solar
system), where human proxies will be used for continuous exploration to expand
our understanding of the universe.
\index{remote exploration}

\subsection*{Other Trends in Robotics and Intelligent Machines}

In addition to the challenges and opportunities described above, there are
many other trends that are important for advances in robotics and intelligent
machines 
and that will drive new research in \cds.

\index{mixed initiative systems}
\index{human interfaces|see{man-machine systems}}
\paragraph{Mixed Initiative Systems and Human Interfaces}
It seems clear that more extensive use of computer control, be it 
for factories, automobiles or homes, will be most effective if it comes with a
natural human interface.  Having this goal in mind, one should look for
interfaces which are not only suitable for the given application but which are
sufficiently general so that, with minor modification, they can serve in
related applications as well.  Progress in this area will not only require new
insights into processing of visual data (described above), but a better
understanding of the {\em interactions} of humans with machines and computer
controlled systems.
\index{man-machine systems}

One program underway in the United States is exploring the use of ``^{variable
autonomy}'' systems_{autonomous systems}, in which machines controlled by
humans are given varying levels of command authority as the task evolves.
Such systems involve humans_{man-machine systems} that are integrated with a
computer-controlled system in such a way that the humans may be simultaneously
receiving instructions from and giving instructions to a collection of
machines.  One application of this concept is a semi-automated ^{air traffic
control} system, in which ^{command and control} computers, human air traffic
controllers, flight navigation systems, and ^{pilots} have varying levels of
responsibility for controlling the airspace_{airspace management}.  Such a
system has the possibility of combining the strengths of machines in rapid
data processing with the strengths of humans in complex reasoning_{reasoning,
higher level}, but will require substantial advances in understanding of
man-machine systems.

\paragraph{Control Using High Data-Rate Sensors}
\index{sensing!in robotic systems}
\index{data rich systems}
Without large expenditure, we are able to gather and store more pictures and
sounds, temperatures and particle counts, than we know how to use.  We
continue to witness occasional catastrophic failures of our man-machine
systems, such as those used for transportation, because we do not correctly
interpret or appropriately act on the information available to us.  It is
apparent that in many situations collecting the information is the easy part.
Feedback control embodies the idea that performance can be improved by
coupling measurement directly to action.  
\index{physiology}
Physiology provides many examples
attesting to the effectiveness of this technique.  However, as engineers and
scientists turn their attention to the highly automated systems currently
being built by the more advanced ^{manufacturing} and service industries, they
often find that the direct application of feedback control is frustrated by a
web of interactions which make the smallest conceptual unit too complex for
the usual type of analysis.  
\index{vision guided systems}
In particular, vision guided systems are
difficult to design and often fail to be robust with respect to lighting
conditions and changes in the environment. In order to proceed, it seems,
design and performance evaluation must make more explicit use of ideas such as
adaptation, self-configuration_{self-configuring systems}, and
self-optimization. 
\index{adaptation}

\index{active vision} 
Indications are that the solution to the problems raised above will involve
active feedback control of the perceptual processes, an approach which is
commonplace in biology.  One area that has
received considerable attention is the area of active vision in which the
vision sensor is controlled on the basis of the data it generates.  Other work
involves tuning the vision processing algorithms on basis of the data
collected.  The significant progress now being made toward the resolution of
some of the basic problems results, in large part, from the discovery and
aggressive use of highly nonlinear signal processing techniques. Examples
include the variational theories that have been brought to bear on the image
segmentation problem, the theories of learning based on computational
complexity, and information theoretic based approaches to perceptual problems.
Attempts to incorporate perceptual modules into larger systems, however, often
raise problems about communication and distributed computation which are not
yet solved.

Related to this is the problem of understanding and interpreting visual data.
The technology for recognizing voice commands is now sophisticated enough to
see use in many commercial systems.  However, the processing and
interpretation of image data is in its infancy, with very few systems capable
of decision making and action based on visual data.  One specific example is
understanding of human motion, which has many applications in robotics.  
\index{human motion, understanding}
While
it is possible for robots to react to simple gestures, we do not yet have a
method for describing and reasoning about more complex motions, such as a
person walking down the street, stooping to pick up a penny, and being bumped
by someone that did not see them stop.  This sort of interpretation requires
representation of complex spatial and symbolic relationships that are beyond
currently available tools in areas such as ^{system identification}, state
estimation, and signal to symbol translation.
\index{reasoning, higher level}

\index{medicine!robotics for|(}
\paragraph{Medical Robotics}  
Computer
and robotic technology is having a revolutionary impact on
the practice of medical surgery.  By extending surgeons' ability to plan and
carry out surgical interventions more accurately and in a minimally
invasive
manner, computer-aided and robotic surgical systems can reduce surgical and
hospital costs, improve clinical outcomes, and improve the efficiency of
health care delivery.  The ability to consistently carry out surgical
procedures and to comprehensively log key patient and procedure outcome data
should also lead to long term improvements in surgical practice.

\index{minimally invasive surgery}
\index{Robodoc}
Robotic technology is useful in a variety of surgical contexts.  For example,
the ``Robodoc'' surgical assistant uses the precision positioning and drilling
capabilities of robots to improve the fit of implants during total hip
replacement~\cite{BBB98-corr}.  The improved fit leads to significantly fewer
complications and 
longer lasting implants.  Similarly, 3-dimensional imaging data can drive the
precision movement of robot arms during stereotactical brain surgery, thereby
reducing the risk of collateral brain damage.  
\index{DaVinci surgical system}
The DaVinci system from
Intuitive Surgical uses teleoperation and force-reflecting feedback methods to
enable minimally invasive coronary procedures that would otherwise require
massively invasive chest incisions~\cite{MFDA99-jtcs}.
\index{ZEUS robotic surgical system}
Figure~\ref{cmizeus.fig} 
shows the ZEUS 
system developed by Computer Motion, Inc.\, a modified version of which was
used in 2001 to allow a surgeon in New
York to operate on a 68 year old woman in Strasbourg,
France~\cite{Mar+01-nature}.
\begin{figure}
  \centerline{\psfig{figure=cmizeus.eps,width=0.9\figwidth}}
  \caption{The ZEUS (tm) Robotic Surgical System, developed by Computer
  Motion 
  Inc., is capable of performing minimally invasive microsurgery
  procedures from a remote location.  Photograph courtesy of Computer Motion
  Inc.} 
  \label{cmizeus.fig}
\end{figure}
These are only a few of the currently approved robotic surgical
systems, with many, many more systems in clinical trials and laboratory
development.

\index{fault tolerance}
While medical robotics is becoming a reality, there are still many open
research and development questions.  Clearly, medical robotics will benefit
from the same future advances in computing, communication, sensing, and
actuation technology that will broadly impact all future control systems.
However, the issue of system and software ^{reliability} is fundamental to the
future of medical robotics.  
\index{verification and validation}
Formal methods for system verification of these
highly nonlinear, hybrid, and uncertain systems, as well as strategies for
extreme fault tolerance are clearly needed to ensure rapid and widespread
adoption of these technologies.  Additionally, for the foreseeable future,
robotic medical devices will be assistants to human surgeons.  Consequently,
their human/machine interfaces_{man-machine systems} must be able to deal with
the complex contexts 
of crowded operating rooms in an absolutely reliable way, even during
unpredictable surgical events.

\index{medicine!robotics for|)}
\index{robotics|)}
\clearpage
%% \input{biomed.tex}
% Master File: cdspanel.tex
\section{Biology and Medicine}
\index{biology|(}

\begin{itemize}
  \small\item[]
\index{temperature control}
{\em Feedback is a central feature of life. The process of feedback governs
how we grow, respond to stress and challenge, and regulate_{regulation}
factors such as 
body temperature, blood pressure, and cholesterol level. The mechanisms
operate at every level, from the interaction of proteins in cells to the
interaction of organisms in complex ecologies.} 

Mahlon B. Hoagland and B. Dodson, from {\it The Way Life Works},
1995~\cite{HD95}.
\end{itemize}
\index{Hoagland, M. B.}
\index{Dodson, B.}
\index{feedback!in biological systems}

At a variety of levels of organization---from molecular to
cellular to organismal---biology is becoming more accessible to
approaches that are commonly used in engineering: mathematical
modeling, systems theory, computation, and abstract approaches to
synthesis.  Conversely, the accelerating pace of discovery in
biological science is suggesting new design principles that may have
important practical applications in man-made systems.  This synergy
at the interface of biology and engineering offers
unprecedented opportunities to meet challenges in both areas.  The
principles of \cds{} are central to many of the key questions in
biological engineering and will play a enabling role in the future of
this field.
\index{biological engineering}
\index{control!challenges}

\index{locomotion}
\index{reverse engineering}
\index{active vision}
\index{population models}
\index{modeling!population models}
\index{epidemics}
A major theme identified by the Panel was the science of reverse (and
eventually forward) engineering of biological control networks.  There are a
wide variety of biological phenomena that provide a rich source of examples
for control, including gene regulation and signal transduction; hormonal,
immunological, and cardiovascular feedback mechanisms; muscular control and
locomotion; active sensing, vision, and proprioception; attention and
consciousness; and population dynamics and epidemics.  Each of these
(and many more) provide opportunities to figure out what works, how it works,
and what can be done to affect it.

\index{medicine}
\index{flow control}
The Panel also identified potential roles for \cds{}
in medicine and biomedical research.  These included
intelligent operating rooms and hospitals, from raw data to decisions;
image guided surgery and therapy; hardware and soft tissue integration;
fluid flow control for medicine and biological assays; and the development of
physical and neural prosthesis.  Many of these areas have substantial overlap
with robotics 
and some have been discussed already in
Section~\ref{robotics.sec}.
\index{image guided surgery}
\index{neural prosthesis}

We focus in this section on three interrelated aspects of
biological systems: molecular biology, integrative biology, and medical
imaging.  These areas are representative of a larger class of biological
systems and demonstrate how principles from \cds{} can be used to understand
nature and to build engineered systems.

\index{molecular biology|(}
\index{biomolecular systems|see{molecular biology}}
\subsection*{Molecular Biology\protect\footnote{The Panel would like to thank
Eduardo Sontag for his contributions to this section, based on his Reid Prize
plenary lecture at the 2001 SIAM Annual Meeting.}}

\index{genomics}%
\index{DNA}
The life sciences are in the midst of a major revolution that will have
fundamental implications in biological knowledge and medicine.  
Genomics has as its objective the complete decoding of DNA sequences,
providing a ``parts list'' for the proteins
present in every cell of the organism being studied.  Proteomics is the study
of the three-dimensional structure of these complex proteins.
\index{proteomics}%
The shape of a protein determines its function:
^{proteins} interact with each other through ``lego-like'' fitting of parts in
``lock and key'' fashion, and their conformation also enhances or represses
DNA expression through selective binding.

\index{cellular signaling networks}
\index{networks!biological cell}
One may view cell life as a huge ``wireless'' network of interactions among
proteins, DNA, and smaller molecules involved in signaling and energy transfer.
As a large system, the external inputs to a cell include physical signals (UV
radiation, temperature) as well as chemical signals (drugs, hormones,
nutrients).  Its outputs include chemicals that affect other
cells. 
Each cell can be thought of, in turn, as composed of a large number of
subsystems involved in cell growth, maintenance, division, and death.
A typical diagram describing this complex set of interactions is shown in
Figure~\ref{cancernet.fig}.
\begin{figure}
  \centerline{\psfig{figure=cellIC.eps,width=0.9\figwidth}}
  \caption{The wiring diagram of the growth signaling circuitry of the
  mammalian cell~\cite{HW00-cell}.} 
  \label{cancernet.fig}
\end{figure}
%% Need copyright permit from journal ``Cell.''

\index{cellular signaling networks}
The study of cell networks leads to the formulation of a large number of
questions.  For example, what is special about the information-processing
capabilities, or input/output behaviors, of such biological networks?  Can one
characterize these behaviors in terms familiar to control theory (e.g.,
transfer functions or Volterra series)?  What ``modules'' appear repeatedly in cellular signaling
cascades, and what are their system-theoretic properties?  Inverse or
``reverse engineering'' issues include the estimation 
\index{reverse engineering}\index{estimation}%
of system parameters (such as reaction constants) as well as the estimation of
state variables (concentration of protein, RNA, and other chemical substances)
from input/output experiments.  Generically, these questions may be viewed
respectively as the identification_{system identification} and observer (or
filtering) problems which 
are at the center of much of control theory.

\index{stability!cellular signaling networks}
One can also attempt to better understand the stability properties of the
various cascades and feedback loops that appear in cellular signaling
networks.  Dynamical properties such as stability and existence of
oscillations in such networks are of interest, and techniques from control
theory such as the calculation of robustness margins will play a central role
in the future.  At a more speculative (but increasingly realistic) level, one
wishes to 
study the possibility of using control strategies (both open and closed loop)
for therapeutic purposes, such as drug dosage scheduling.
\index{drug dosage scheduling}

\index{modeling!biology}
The need for mathematical models in cellular biology has long been recognized,
and indeed many of the questions mentioned above have been studied for the
last 20 or 30 years.  What makes the present time special is the availability
of huge amounts of data---generated by the genomics and proteomics projects,
\index{genomics}%
\index{proteomics}%
and research efforts in characterization of signaling networks---as well as
the possibility for experimental design afforded by genetic engineering tools
(gene knock-outs and insertions, PCR) and measurement technology (green
fluorescent protein and other reporters, and gene arrays).  Control-oriented
modeling and analysis of feedback interconnections is an integral component of
building effective models of biological systems.
\index{feedback!in biological systems}

\index{uncertainty!in biological systems}
\index{feedback!loop}
\index{E.~coli@{\it E.~coli}}
\partitle{Feedback and uncertainty} From a theoretical perspective, feedback
serves to minimize uncertainty and increase accuracy in the presence of noise.
The cellular environment is extremely noisy in many ways, while at the same
time variations in levels of certain chemicals (such as transcriptional
regulators) may be lethal to the cell.  Feedback loops are omnipresent in the
cell and help regulate_{regulation} the appropriate variations.  It is
estimated, for 
example, that in {\it E.~coli} about 40\% of transcription factors
self-regulate.  One may ask whether the role of these feedback loops is indeed
that of 
reducing variability, as expected from principles of feedback theory.  Recent
work tested this hypothesis in the context of tetracycline repressor
protein_{proteins} 
(TetR)~\cite{BS00:nature}.  An experiment was designed in which feedback loops
in TetR production were modified by genetic engineering techniques, and the
increase in variability of gene expression was correlated with lower feedback
``gains,'' verifying the role of feedback in reducing the effects of
uncertainty. 
\index{uncertainty!in biological systems}
Modern experimental techniques will afford the opportunity for 
testing experimentally (and quantitatively) other theoretical predictions, and
this may be expected to be an active area of study at the intersection of
control theory and molecular biology.
\index{experiments}

\partitle{Necessity of embedded structures in ^{regulation} loops}  
Another illustration of the interface between feedback theory and modern
molecular biology is provided by recent work on chemotaxis in bacterial
motion.  
\index{chemotaxis}
\index{E.~coli@{\it E.~coli}}
\index{bacteria}
{\it E.~coli} moves, propelled by flagella, in response to gradients
of chemical attractants or repellents, performing two basic types of motions:
{\em tumbles} (erratic turns, with little net displacement) and {\em runs}.
In this process, {\it E.~coli} carries out a stochastic gradient search
strategy: when sensing increased concentrations it stops tumbling (and keeps
running), but when it detects low gradients it resumes tumbling motions (one
might say that the bacterium goes into ``search mode'').

\index{adaptation}
\index{control!principles}%
The chemotactic signaling system, which detects chemicals and directs motor
actions, behaves roughly as follows: after a transient nonzero signal (``stop
tumbling, run toward food''), issued in response to a change in
concentration, the system adapts and its signal to the motor system converges
to zero (``OK, tumble'').  This adaptation happens for any constant nutrient
level, even over large ranges of scale and system parameters, and may be
interpreted as robust (structurally stable) rejection of constant
disturbances.  The {internal model principle} of control theory implies
(under appropriate technical conditions) that there must be an {embedded
integral controller}_{integral action} whenever robust constant disturbance
rejection is achieved.  
\index{disturbance rejection}
Recent models and experiments succeeded in finding, indeed, this
embedded structure~\cite{BL97-nature,YHSD00-pnas}.

This work is only one of the many possible uses of control theoretic knowledge
in reverse engineering of cellular behavior.  Some of the deepest parts of the
theory concern the necessary existence of embedded control structures, and in
this manner one may expect the theory to suggest appropriate mechanisms and
validation experiments for them.

\index{genetics}
\index{hybrid systems}
\partitle{Genetic circuits}
Biomolecular systems provide a natural example of {hybrid} systems, which
combine discrete and logical operations (a gene is either turned on or off for
transcription) and continuous quantities (such as concentrations of chemicals)
in a given cell or in a cell population).  Complete hybrid
models_{modeling!biology} of basic circuits have been formulated, such as the 
lysogeny/lysis decision circuit in bacteriophage
$\lambda$~\cite{MS95-science}.

Current research along these lines concerns itself with the identification of
other naturally occurring circuits, as well as with the engineering goal of
{\em designing} circuits to be incorporated into delivery vehicles
(^{bacteria}, for example), for therapeutic purposes.  This last goal is, in
principle, mathematically in the scope of {realization theory}, that branch of
systems theory which deals with the synthesis of ^{dynamical systems} which
implement a specified behavior.

\index{molecular biology|)}

\subsection*{Integrative Biology\protect\footnote{The Panel would like to
thank Michael Dickinson for his contributions to this section.}}

\index{integrative biology|(}
\index{locomotion}
\index{control!principles}%
\index{control!as enabling technology}
\index{animals|see{integrative biology}}
\index{insects|see{integrative biology}}
\CDS{} also has a role to play in understanding larger scale
organisms, such as insects and animals.  The components of these {\em
integrative} biological systems are becoming much better understood
and, like molecular systems, it is becoming evident that systems
principles are required to build the next level of understanding.
This understanding of natural systems will enable new approaches to
engineered systems, as we begin to build systems with the efficiency,
robustness, and versatility of the natural world.  We focus here on
the problem of locomotion, for which there has been substantial recent
work (see~\cite{Dic+00-science} for a review).

Integrative studies of locomotion have revealed several general
principles that underly a wide variety of organisms.  These include
energy storage and exchange mechanisms in legged locomotion and
swimming, nonpropulsive lateral forces that benefit stability and
maneuverability, and locomotor control systems that combine mechanical
reflexes with multimodal sensory feedback and feedforward control.
Locomotion, especially at the extremes of what is found in nature,
provides a rich set of examples that have helped elucidate a variety of
structure-function relationships in biological systems.
\index{feedforward control}

\index{dynamics!locomotion systems}
\index{feedback!in biological systems}
Control systems and feedback play a central role in locomotion.  A
suite of neurosensory devices are used within the musculoskeletal
system and are active throughout each cycle of locomotion.  In
addition, the viscoleastic dynamics of the musculoskeletal system play a
critical role in providing rapid feedback paths that enable stable
operation.  Rapid feedback from both mechanical and neural pathways is
integrated with information from eyes, ears, noses and other
sensing_{sensing!in biological systems}
organs used to control the overall motion of an animal and provide robust
operation in a wide variety of environments.

\index{flight control!insect}
The process that gives rise to locomotion is a complex one, as
illustrated in Figure~\ref{flymd.fig} for the flight behavior of a fruit fly.
\begin{figure}
  \centerline{\psfig{figure=flymd.eps,width=0.8\figwidth}}
  \caption{Overview of flight behavior in a fruit fly, {\it Drosophila}.  (a)
  Cartoon of the adult fruit fly showing the three major sensor_{sensing!in
  biological systems} strictures used
  in flight: eyes, antennae, and halteres (detect angular rotations). (b)
  Example flight trajectories over a 1 meter circular arena, with and without
  internal targets.  (c) A schematic control model of the flight system.
  Figure and description courtesy of Michael Dickinson.}
  \label{flymd.fig}
  \index{Drosophila}
\end{figure}
Each element of the flight control system has enormous complexity in
itself, with the interconnection (grossly simplified in the figure)
allowing for a very rich set of behaviors.  The sensors, actuators,
and control systems for insects such as the fly are highly evolved, so
that the dynamics of the system play strongly into the overall
capabilities of the organism.

\index{autonomous systems}
\index{fault tolerance}
From the perspective of control theory, the performance, robustness,
and fault tolerance of the fly's flight control system represents a
gold standard by which all autonomous systems might be judged. Flies
can manage to stay in the air with torn wings, missing legs, blind
eyes, or when burdened with twice their weight in additional mass. The
fact that the control algorithms yielding this behavior reside in a
brain the size of a sesame seed raises the bar for any biomimetic
effort attempting to match its performance. If the principles that
engender a fly with such robust agility could be discovered and
formalized for general use, the results might catalyze a revolution in
the design, fabrication, and implementation of control systems.

Similarly, the use of \cds{} tools to understand the fly's flight
control system represents a systems approach to biology that will be
important for understanding the general principles of locomotion
systems and allow new understanding of integrative biological
principles.  

\index{modeling!biology}
\index{multiscale systems}
\index{communications!integration of control and}
This synergy between biology and \cds{} in insect flight is but one example of
many that are possible and that form a rich source of scientific and
engineering activity.  Additional areas of overlap include multiresolution
modeling and analysis of (nongeneric, designed, evolved, heterogeneous)
multiscale systems, and integrated communications and computing for control of
and with pervasive, distributed, embedded networks.  Biological systems are
also exceptionally capable of transforming raw data into information and
knowledge, and eventually into decision and action.  These are precisely the
problems that confront us in building engineering systems and the interaction
of biologists and \cds{} researchers is destined to be fruitful.

\index{integrative biology|)}

\index{medicine|(}
\subsection*{Medical Imaging\protect\footnote{The Panel would like to thank
Allen Tannenbaum for his contributions to this section.}}
\index{biomedicine|see{medicine}}

\CDS{} is also an essential element in the burgeoning field of biomedicine.
Some of these applications, such as robot surgery, have already been discussed
in the context of robotics and intelligent machines (see
Section~\ref{robotics.sec}).  We consider two additional examples here: image
guided therapy (IGT) and image guided surgery (IGS).
\index{image guided surgery}

\index{minimally invasive surgery}
Image guided therapy and surgery provide illustrations of how to use
biomedical engineering principles to develop general-purpose software methods
that can be integrated into complete therapy delivery systems.  Such systems
will support more effective delivery of many image-guided procedures---biopsy,
minimally invasive surgery, and radiation therapy, among others.  A key
element is controlled active vision.  
\index{active vision}
To understand the its role in the
therapeutic process, and to appreciate the current usage of images before,
during, and after treatment, one must consider the four main components of IGT
and IGS: localization, targeting, monitoring and control.

\index{segmentation of images}
\index{registration of images}
To use controlled active imaging one must first develop robust algorithms
for {\em segmentation}, automated methods that create patient-specific models
of relevant anatomy from multimodal imagery, and {\em registration},
automated methods that align multiple data sets with each other and with the
patient.  These technologies must then be integrated into complete and coherent
image guided therapy delivery systems and validated using performance measures
established in particular application areas.  \CDS{} enters at almost every
stage of the process. For example, control-theoretic methods can be essential
for the success of the {\em deformable or active contours} technique in active
vision for therapeutic and surgical procedures.  
\index{active vision}
These are autonomous
processes that employ image coherence in order to track features of interest
over time. They have been used for segmentation and edge detection as
well. For dynamically changing imagery in a surgical environment, Kalman
filtering has been important in estimating the position of an active contour
at a given time given its previous position. This estimated data may be used
then in a closed loop visual tracker.
\index{Kalman filter}
\index{data rich systems}

\index{magnetic resonance imagery (MRI)}
Further, speed and robustness are very important in interventional magnetics,
which uses magnetic resonance imagery (MRI) during surgery.  Here surgeons can
operate in an open MRI device, and use the images to guide their procedure.
Fast segmentation is of paramount importance, and one can use active contours
very effectively when coupled with an estimation scheme to extract key
features (such as a brain tumor or breast cyst).

\index{registration of images}
Image registration is the process of establishing a common geometric reference
frame between two or more data sets from the same or different imaging
modalities possibly taken at different times.  Multimodal registration
proceeds in several steps.  First, each image or data set to be matched should
be individually calibrated, corrected from imaging distortions, and cleaned
from noise and imaging artifacts.  Next, a measure of dissimilarity between
the data sets must be established, so we can quantify how close an image is
from another after transformations are applied to them.  Once features have
been extracted from each image, they must be paired to each other.  Then, a
similarity measure between the paired features is formulated which can be
formulated as an optimization problem of the type many times used in control.

Optimal transport methods have proved very useful for this. Optimal transport
ideas have been used in nonlinear stability analysis, and very similar
concepts lead to a measure of similarity between images which can be employed
in registration and data fusion.

In general, IGT and IGS will benefit enormously from systems oriented ideas.
At this point most of the control is being done by the computer vision and
medical imaging community.  By building stronger ties between these groups and
the \cds{} community, it will be possible to make more rapid progress and to
leverage advances from other applications.  In addition, the specific features
of this class of problems will drive new advances in \cds{} theory and
technology, which can then be exported to other areas.

\index{biology|)}
\index{medicine|)}
\clearpage
%% \input{matproc.tex}
% Master File: cdspanel.tex

\index{materials|(}
\index{processing, chemical|(}
\section[Materials and Processing]
  {Materials and Processing\footnote{The Panel would like to
  thank Richard Braatz and Frank Doyle for their contributions to this
  section.}}

The chemical industry is among the most successful industries in the United
States, producing \$400 billion of products annually and providing over one
million U.S.\ jobs. Having recorded a trade surplus for forty consecutive
years, it is the country's premier exporting industry: chemical industry
exports totaled \$72.5 billion in 2000, accounting for more than 10\% of all
U.S.\ exports, and generated a record trade surplus in excess of \$20 billion
in 1997.  

\index{manufacturing}
\index{process control}
Process manufacturing operations will require a continual infusion of advanced
information and process control technologies if the chemical industry is to
maintain its global ability to deliver products that best serve the customer
reliably at the lowest cost.  In addition, a number of new technology areas
are being explored that will require new approaches to \cds{} in order to be
successful.  These range from nanotechnology in areas such as electronics,
chemistry, and biomaterials, to thin film processing and design of integrated
microsystems, to supply chain management and enterprise resource allocation.
The payoffs for new advances in these areas are substantial, and the use of
\cds{} is critical to future progress in sectors from semiconductors to
pharmaceuticals to bulk materials.
\index{thin films}
\index{microelectronics}
\index{biomaterials}
\index{supply chains}
\index{enterprise level systems}
\index{resource allocation}

\begin{figure}
  \centerline{
    \begin{tabular}{ccc}
    \psfig{figure=p4wafer.eps,width=0.45\figwidth} &&
    \psfig{figure=p4die.eps,width=0.45\figwidth} \\
      (a) && (b)
    \end{tabular}
  }
  \caption{(a) Intel Pentium IV wafer and (b) die. Photographs courtesy of
    Intel. 
  }
  \label{pentium.fig}
\end{figure}

\subsection*{Background and History}
\index{control!history of!process control}
\index{process control!history of}

\index{automobiles}
\index{disk drives}
\index{microprocessors!manufacturing of}
At least one materials or chemicals process is involved in the
manufacture of nearly every commercial product, including
microprocessors, consumer products such as detergents and shampoo,
books, diskettes, disk drives, video cassette recorders, food,
pharmaceuticals, adhesives, automobile dashboards, and aircraft
interiors.  Feedback controllers for these processes provide
improved product quality, reduced materials and energy usage, reduced
environmental impact, improved safety, and the reduced costs needed for
U.S.\ industry to be competitive in the global economy.

By the late 1960s, process control had been implemented liberally to
chemical and materials processes, primarily in the form of single-loop
controllers with little communications between controllers.
\index{multi-variable control}
Multi-variable control began to be implemented in the 1970s, including
some rather large scale processes such as the control of uniformity in
plastic film and paper machines.  
\index{paper machine control}
The use of multi-variable control
grew rapidly throughout the 1980s and 1990s.  Over the last 25 years,
multi-variable optimal control in the form of model predictive control
has become a standard control technique in the process industries for
addressing actuator and state constraints, 
\index{constraints, actuator}
which are quite prevalent
in chemicals and materials processes.  
\index{model predictive control}
Model predictive control
explicitly takes constraints into account during the online
calculation of the manipulated variable actions.  
In 2000, more than 5000 applications of model predictive control were
reported by the control vendors of that time (e.g., Adersa, Aspen Technology,
Honeywell Hi-Spec, Invensys, and Shell Global Solutions)~\cite{qb02-cep}.
\index{Honeywell}%
Applications have been reported in a wide range of industries including
refining, petrochemical, pulp and paper, air separation, food processing,
furnaces, aerospace, and automotive.  In recent years model predictive control
algorithms have been developed that enable their application to very large
scale 
process control problems.

This should not be taken, however, to indicate that all process control
problems have been solved.  New control techniques are needed that
address all of the characteristics of the most challenging chemicals and
materials processes.

\subsection*{Current Challenges and Future Needs}
\index{control!challenges}

\index{modeling!materials and processing}
The Panel identified a number of common features within materials and
processing that pervade many of the applications.  Modeling plays a crucial
role and there is a clear need for better solution methods for
multidisciplinary systems combining chemistry, fluid mechanics,
thermal sciences and other disciplines at a variety of temporal and spatial
scales.  Better numerical methods for traversing these scales_{multiscale
systems} and designing,
controlling and optimizing under uncertainty are also needed.  And \cds{}
techniques must make use of increased {\it in situ} measurements to
control increasingly complex phenomena.

\index{microelectronics}
\index{control!as enabling technology}
Advances in materials and processing are important for a variety of industries
in which control of complex process systems enables growth in the world
economy.  One example is the microelectronics industry, which has an average
annual growth of 20\%, with sales of \$200 billion in 2001.  As described by
the International Technology Roadmap for Semiconductors,\footnote{\tt
http://public.itrs.net} 
high performance feedback control will be needed to achieve the small length
scales required for the next generation of microelectronic devices that are
predicted (and hence demanded) by Moore's Law.

\index{pharmaceuticals}
A second example is the pharmaceuticals industry, which is growing at 10-20\%
annually, with sales of \$150 billion in 2000.  The primary bottleneck to the
operation of production-scale drug manufacturing facilities is associated with
difficulties in controlling the size and shape distribution of crystals
produced by complex crystallization processes (see Figure \ref{fig:crystals}).
\begin{figure}
  \centerline{
    \begin{tabular}{ccc}
    \psfig{figure=PCMPIC2tif.ps,height=1.5in} &\qquad\qquad&
    \psfig{figure=tylenol.eps,width=1.5in} \\
      (a) &\qquad& (b)
    \end{tabular}
  }
  \caption{\label{fig:crystals} (a) Microscope image of paracetamol crystals
  (paracetamol is the active ingredient in Tylenol (b)), which shows the
  variability in crystal shape that can occur at a single time instance
  in a pharmaceutical crystallizer.  Image courtesy of Richard Braatz.}
  \label{tylenol.fig}
\end{figure}
\index{crystallization}%
Crystallization processes typically involve growth, agglomeration, nucleation,
and attrition mechanisms which can be affected by particle-particle
collisions.  Poor control of this crystal size distribution can completely
halt the production of pharmaceuticals, causing both economic and medical
hardship.

\index{environmental regulations}
In addition to the continuing need to improve product quality, there are
several other factors in the ^{process control} industry that are drivers for
the 
use of \cds{}.  Environmental regulations continue to place stricter
limitations 
on the production of pollutants, forcing the use of sophisticated ^{pollution}
control devices.  Environmental safety considerations have led to the
design of smaller storage capacities to diminish the risk of major chemical
leakage, requiring tighter control on upstream processes and, in some cases,
supply chains.  
\index{supply chains}
Large increases in energy costs have encouraged engineers
to design plants which are highly integrated, coupling many processes that
used to operate independently.  All of these trends increase the complexity
of these processes and the performance requirements for the control systems,
making the control system design increasingly challenging.

\index{sensing!in process control systems}
As in many other application areas, new sensor technology is creating new
opportunities for \cds{}.  Online sensors---including laser backscattering,
video microscopy, ultraviolet, infrared, and Raman spectroscopy---are becoming
more robust and less expensive, and are appearing in more manufacturing
processes.  Many of these sensors are already being used by current process
control systems, but more sophisticated signal processing and \cds{}
techniques are needed to more effectively use the real-time information
provided by these sensors.  Control engineers can also contribute to the
design of even better sensors which are still needed, for example, in the
microelectronics industry.  As elsewhere, the challenge is making use of the
large amounts of data provided by these new sensors in an effective manner.
In addition, a control-oriented approach to modeling the essential physics of
the underlying processes is required to understand fundamental limits on
observability of the internal state through sensor data.
\index{data rich systems}
\index{sensor rich systems|see{data rich systems}}

\index{complexity}
\index{uncertainty}
\index{dynamics!processing systems}
Another common feature in materials and process control is the inherent
complexity of the underlying physical processing.  Modern process systems
exhibit very complex nonlinear dynamics, including substantial model
uncertainty, actuator and state constraints, 
\index{constraints, actuator}
and high dimensionality (usually
infinite).  
\index{stochastic systems}
\index{differential equations}
\index{partial differential equations (PDEs)|see{differential equations}}
\index{modeling!materials and processing}
These systems are often best described by tightly coupled systems
of algebraic equations and stochastic partial integrodifferential equations
with 
widely varying time and length scales and significant nonlinearities.  
\index{microelectronics}
This is
especially true in the microelectronics industry, where hundreds of stiff
partial ^{differential equations} can be required for predicting product
quality, 
for example, during the modeling of cluster formation and dissolution during
fast-ramp annealing after ion bombardment.  Other processes are best described
by kinetic Monte Carlo simulations_{simulation}, with or without coupling to continuum
equations, which can be run on serial or parallel computers.  Both
identification and control algorithms are needed that can {\em simultaneously}
address the high complexity, high nonlinearity, and high dimensionality of
these complex process systems.  Furthermore, there is significant uncertainty
associated with many of the kinetic parameters, even with improved sensors, so
these algorithms need to be robust to model uncertainties.
\index{uncertainty}

Two specific areas that illustrate some of the challenges and future needs are
control of particulate systems and biotechnology.

\subsection*{Control of Particulate Systems}
\index{particulate systems|(}

\index{agriculture}
\index{pharmaceuticals}
\index{population models}
\index{modeling!population models}
Particulate processes are prevalent in a number of process industries
including agricultural, chemical, food, minerals, and pharmaceutical. By some
estimates, 60\% of the products in the chemical industry are manufactured as
particulates with an additional 20\% using powders as ingredients. One of the
key attributes of such systems is the distributed characterization of physical
and chemical properties such as size, shape, morphology, porosity, and
molecular 
weight. The underlying mechanisms which describe the evolution of such
systems are captured by population balance models, which are coupled sets of
hyperbolic partial differential and algebraic equations.
\index{differential equations}

There are a number of challenges in the numerical solution of such equations,
particularly when considering real-time_{real-time systems} applications such as model-based
control. Critical in such models are the kernels or driving forces (e.g.,
nucleation, growth, agglomeration, and breakup)\ that are typically not well
characterized, and are often determined from process data via various
identification techniques. 
\index{population models}
\index{modeling!population models}
These problems become increasingly complex as one
considers higher-dimension population balance models (e.g., size and shape),
where the number of parameters in the kernels grows rapidly with the increase
in additional degrees of freedom. 
\index{sensing!in process control systems}
At the same time, there have been
substantial advances in the domain of sensor technology, such that attributes
like the particle size distribution can be measured in real-time by a variety
of techniques including light scattering, ultrasound spectroscopy and
hydrodynamic capillary separation.  This leads to control formulations
involving distributed measurement variables, highly nonlinear process models,
nonlinear operating constraints, and complex hierarchical operating
objectives.

To explore some of the major challenges, we consider three selected
application areas---polymerization, granulation, and profile control.

\index{polymers|(}
\partitle{Emulsion Polymerization} Increasing global
competition for the production of higher quality polymer products
at lower costs, coupled with a general trend away from new capital
investments in the U.S., has placed considerable pressure on the
process engineers in the U.S.\ to operate the existing polymer
plants more efficiently and to use the same plant for the
production of many different polymer products. Lack of sufficient
controllability is a barrier to better product quality control in
some polymer processes. In many polymer processes, better product
quality requires minimizing/maximizing several product quality
indices simultaneously. This multi-objective requirement may
result in narrow ranges of process trajectories, putting a premium
on the controllability of the process. For instance, in coatings,
the product's composition, molecular weight, and particle size
distributions should be maintained simultaneously in limited
ranges  to ensure the coating has a desired level of film
formation, film strength, and gloss.
\index{coatings}

The critical link between these product quality indices and the
operating process variables is often the distributed attribute
such as the size distribution. In the past, such attributes were
controlled indirectly using inferential control schemes, but
online sensor_{sensing!in process control systems} technology brings the
promise of real-time control 
of these properties. This motivates the development of refined
quantitative relationships between the distributed quantities and
the quality variables. While experimental techniques have been
used to develop relationships that hold for limited operating
conditions, these descriptions do not readily lend themselves to
optimization, either in terms of productivity or reduction in
variance.
\index{polymers|)}

\index{granulation|(}
\partitle{Granulation} Granulation is a key step in many
particulate processes where fine particles are agglomerated with
the aid of a liquid binder into larger granules. It is often used
to improve the visual appearance and/or taste of materials,
improve the flowability of the materials, enable compaction and
tableting, and reduce dustiness. The granulation process exhibits
many characteristics common to other particulate processes such as
crystallization and emulsion polymerization. Typically, a desired
product quality can be inferred from the Particle Size
Distribution (PSD) of a process. The ability to manipulate a PSD
allows for control of the end product quality, but PSD control can
pose a very difficult control problem due to the significant
multi-variable interacting character of PSD systems. In some
situations, values of the measured PSD may be constrained to a
specified acceptable region in order to achieve a desired product
quality. 

As with many particulate processes, there is a rich interplay
between mechanisms at the microscopic, mesoscopic and macroscopic
levels in granulation, however, the fundamental knowledge to link
these mechanisms for use in model-based control is rather limited.
In particular, the tradeoffs between model quality and complexity
for various model uses have not been investigated systematically,
leading to inadequate selections of model forms. 
\index{modeling!materials and processing}
\index{multiscale systems}
Furthermore,
granulation is a complex multiscale process, including
multi-number, dimension and time scales. The current status of
granulation research clearly shows significant gaps between
microscopic-level studies and plant-scale modeling, and also
between the model forms and the use of models. Given such models
and the already existing sensor technology, one can realize the
tight ^{regulation} of this complex unit operation.
\index{granulation|)}

\index{profile control|(}
\partitle{Profile Control} Though the systems described in
this area are not strictly particulate processes, they share the
attribute that a distributed variable is directly tied to product
performance, hence many of the underlying mathematical constructs
required for control are common to both classes of problems. The
problems of controlling a ``profile'' arise in a number of rather
different process industry unit operations, including polymer
extrusion, cross direction control (paper_{paper machine control}, aluminum,
glass, etc.), 
tubular chemical reactors, and advanced materials processing
(photovoltaic, microelectronic, etc.), to name a few. In some
instances the properties of interest are measured in the cross
direction (CD) giving rise to a 1-D profile control problem, or in
other cases the quality attribute is measured in both the machine
direction (MD) and CD, giving rise to a 2-D sheet control problem.
In reaction unit operations, the extent of reaction across a
spatial direction is a critical parameter that controls important
quality indices. For example, in a pulp digester, the control of
reaction extent profile (measured by the Kappa number) along the axial
direction in the reactor enables the tight ^{regulation} of critical
fiber properties, such as strength, which depend on the reaction
path as well as the final conversion.

One of the interesting challenges that arises, for example, in the
paper machine_{paper machine control} CD control problem is that hundreds of input/output
variables are involved, complex temporal and spatial constraints
must be maintained, and real-time requirements dictate solution
times on the order of seconds. This is further complicated by
non-ideal process behavior owing to paper shrinkage, lane
shifting, production grade and rate changes---all of which give
rise to significant plant-model mismatch, and hence a robust
controller is required. As with the particulate problems, the
sensor technology is changing rapidly, enabling richer
formulations of controlled operation. 

\index{profile control|)}
\index{particulate systems|)}

\subsection*{Biotechnology}
\index{biotechnology|(}

\index{control!as enabling technology}
While ^{process control} has played a relatively minor role in the
biotechnology industries in past years, its value as an enabling
technology is increasing, owing to novel developments in sensor
technology coupled with advances in the mathematical
characterization of intracellular behavior. Furthermore, the
potential to realize efficient process design by accounting for
operational issues (i.e., combined design and control), brings
promise of reducing the development time for new plants, and
maximizing the production interval in a patent window.

\index{sensing!in biological systems}
\index{bioreactor control}
\index{temperature control}
\index{intracellular signaling}
Classical bioreactor control focused on the manipulation of residence time,
nutrient feed and the reactor environment (agitation, temperature, pH, 
etc.)\ in response to averaged system measurements (dissolved oxygen,
temperature, 
pH, limited metabolite concentrations, etc.)  Advances in sensor technology
have enabled direct measurement and manipulation of intracellular mechanisms,
and recent advances in quantitative modeling of bioprocesses allow a much more
thorough understanding of the underlying biochemical processes. A number of
the resulting model structures are inspired from systems engineering concepts,
\index{systems engineering}%
such as the cybernetic_{cybernetics} model which incorporates optimal control
^{regulation} 
of 
cellular processes, or the flux balance analysis which accounts for the convex
constraint space available from a metabolic network. 
\index{modeling!biology}
\index{population models}
\index{modeling!population models}
Population balance models
also find application in this area, for example, in the description of age
distributions in oscillating microbial cell cultures. As with particulate
systems, one can construct high-order population balance descriptions by
accounting for the various elements of the physiological state space (DNA,
RNA, protein, etc.) Commensurate with this increase in structural complexity
is the possibility to achieve refined control objectives for bioprocessing,
such as the control of distinct metabolic pathways.
\index{DNA}

\index{proteins}
\index{E.~coli@{\it E.~coli}}
\index{genetics}
An example of the opportunities that emerge from such increased understanding
is the use of recombinant organizations to produce enzymes and
proteins. Typically, the genes corresponding to the desired product are
inserted into the microorganism through a plasmid. The first phase in the
recombinant protein production involves increasing the cell productivity
(biomass concentration), as the expression of the foreign protein imposes
additional metabolic burden on the organism and decreases the growth
rate. Once a sufficiently high biomass concentration is achieved, the inducer
that expresses the inserted gene is added in the second phase resulting in the
synthesis of the recombinant product. Therefore, the concentration of the
inducer and the time at which the inducer is added are key variables in
maximizing the amount of the recombinant protein formed. One specific example
is the recombinant bioprocess involving chloramphenicol acetyltransferase
(CAT) production in the genetically engineered strain of {\em E.~coli}
JM105. This strain produces both the green fluorescent protein (GFP) 
and CAT, when the metabolized inducer, arabinose, is added to the
bioreactor. The objective is the maximization of the amount of CAT formed at
the end of a batch. The manipulated variables are the glucose feed rate and
the feed rate of arabinose, the inducer which turns on the expression of the
desired product.

\index{biology}
\index{bacteria}
The use of GFP and its variants have revolutionized experimental biology by
enabling accurate real-time reporting of intracellular location and
interactions, which has proven valuable in determining the role and
interactions of a number of proteins. GFP was cloned from the jellyfish, {\em
Aequorea victoria}, in 1992 and has since been successfully expressed in a
number of host organisms as a reporter of expression, as well as to identify
the location of proteins within cells. GFP and its variants have been
successfully used to quantitate intracellular expression levels in insect
larvae, bacterial systems and mammalian cells. Owing to the optical nature of
the signal, the development of sensing devices for industrial application is
direct.
\index{biotechnology|)}

\index{materials|)}
\index{processing, chemical|)}
\clearpage
%% \input{other.tex}
% Master File: cdspanel.tex
\section{Other Applications}

The previous sections have described some of the major application
areas discussed by the Panel.  However, there are certainly many more
areas where ideas from \cds{} are being applied or could be applied.
In this section we collect a few such areas which are more specialized
than those discussed to this point.  As before, these areas are not
meant to be exhaustive, but rather representative of some of the new
and exciting directions within \cds{}.

\index{environmental science}
\subsection*{Environmental Science and Engineering\protect\footnote{The
Panel would like to thank Jared Leadbetter, Dianne Newman, and John Seinfeld
for their contributions to this section.}}

\index{global climate dynamics}
\index{dynamics!environmental systems}
\index{multiscale systems}
It is now indisputable that human activities have altered the environment on a
global scale.  Problems of enormous complexity challenge researchers in this
area and first among these is to understand the feedback systems that operate
on the global scale.  One of the challenges in developing such an
understanding is the multiscale nature of the problem, with detailed
understanding of the dynamics of microscale phenomena such as microbiological
organisms being a necessary component of understanding global phenomena,
such as the carbon cycle.  Two specific areas where \cds{} is relevant are
atmospheric systems and microbiological ecosystems.

\index{atmospheric systems}
\index{inverse modeling}
\index{modeling!inverse}
\index{pollution}
\partitle{Atmospheric systems and pollution}
Within the last few years ``inverse modeling'' has become an
important technique in atmospheric science when there are unknown sources or
sinks of a species.  The essential problem is to infer an optimal global
source (or sink) distribution of an atmospheric trace species from a set of
global observations.  This is equivalent to the following control problem:
given a system governed by a set of partial differential equations (PDEs) and
a set of noisy observations of 
the system, determine the optimal set of inputs that match the model to the
data.  Such a problem has relevance to atmospheric chemical transport models,
of which CO$_2$ is perhaps the most important at the present time.
\index{differential equations}

\index{inverse modeling}
\index{modeling!inverse}
At present, inverse modeling for atmospheric species has been applied only to
those compounds that are inert in the atmosphere or only react via simple
mechanisms.  One area that offers promise is the development of techniques for
inverse modeling to trace species that undergo nonlinear atmospheric
processes, such as ozone.  The inverse modeling problem is closely related
theoretically to the sensitivity analysis problem, wherein one seeks the
sensitivity of spatially and temporally varying concentrations to
uncertainties in input functions and variables.  
\index{uncertainty}
Atmospheric inverse modeling
is an important application of ideas from \cds{} to estimate global source
(and sink) distributions of trace species based on noisy, usually sparse
measurements.

\index{microbiological ecosystems}
\index{biofilms}
\partitle{Microbiological ecosystems}
To illustrate how ideas from \cds{} can play a role in microbiological
ecosystems, consider the example of microbial biofilms.  It is widely
recognized that microbial biofilms are ubiquitous, resilient,
responsive to their environment, and able to communicate through
chemical signaling.  
\index{genetics}
\index{gene regulatory networks}%
\index{bacteria}%
Furthermore, specific genes, gene products, and
regulatory_{regulation!gene regulatory networks} networks that control how
bacteria communicate have been 
described in a variety of bacteria.  To date, studies of biofilm
development have been largely limited to studies of pure cultures.
While much has been learned regarding the genetic pathways taken by a
variety of organisms when transitioning from the planktonic to the
sessile phase, little is known about how these pathways change in
response to changes in the environment.  Researchers now believe that
at the scale most relevant to bacteria (the microscale), one of the
most important environmental factors that affect biofilm development
by a given species is the presence of other organisms.  The study of
such ecological networks is at the forefront of research in this area
and the tools of \cds{} can play a major role developing systematic
understanding of their complex interactions.
\index{networks!ecological}
\index{ecological systems}

\index{bacteria}
Another example is in the area of bacterial cells that live inside
organisms.  Although they have limited conventional sensing_{sensing!in
biological systems} and
decision making abilities, bacterial cells are able to rapidly assess
and respond to changes in their metabolism by monitoring and
maintaining relative pool sizes of an extraordinary number (thousands)
of cellular building blocks/intermediates.  A common theme that has
emerged in understanding how this works is related to timing.  Many
changes in ^{physiology} are effected by responses to pauses brought
about by a binding site of an enzyme not being occupied by a given
building block. If a certain building block is depleted, the enzyme
that would incorporate it into cellular material pauses ``in
wait.'' Paused enzymes will often do or allow things that an occupied
one does not. On one hand, this might result in the increased
production of the missing metabolite to bring up the depleted pool to
better reflect the size of the other building block pools, keeping
things in balance. On the other, some enzymes pause as a result of
many pools being depleted in concert, this signals to the cell that it
has begun to exhaust its total resources and moves it into a
starvation survival phase.
 
To control for the possible overproduction of certain pools, many
enzymes involved in the early stages of building block synthesis
become inactive if a binding site becomes occupied by a later or final
product. With the knowledge of this control mechanism, industrial
microbiologists have been able to obtain feedback inhibition mutant
bacteria that over produce almost any desired amino acid.

\index{ecosystems|see{microbiological ecosystems}}
\index{feedback!in biological systems}
In the natural world, work done on termites provides a model system
for studying the role of feedback and control in such microbiological
ecosystems.  There is every reason to believe that termites can
control the delivery of oxygen to, and the consumption of it within
differing zones of the gut epithelium. By doing so, the termite should
be able to protect and even control the activities of its oxygen
sensitive microbiota---but the forms of feedback that the tissue
receives and processes from the gut and atmosphere are not known. One
could envision several ways in which the gut tissue might respond to
oxygen and acetate concentrations to control oxygen delivery to, and
diffusion into the gut compartment. An important question, and one
which \cds{} can help provide an answer, is how the insect and gut
tissues create, control, and maintain a very complex and fragile
ecosystem.

\index{economic systems}
\subsection*{Economics and Finance\protect\footnote{The Panel would like to
thank Jim Primbs for his contributions to this section.}}
Many \cds{} tools have found applications in economics and
there are many common mathematical methods between the fields in areas
such as game theory, stochastic modeling and control, and optimization
and optimal control.
\index{game theory}

\index{options, financial}
Control theory also became an important tool in finance with the
introduction of the Black-Scholes-Merton formula for pricing
options in the early 1970s.  In essence, they showed that the dynamic
control of a portfolio of risky assets could be used to change its
risk profile.  In the extreme case of options, the risk could be
completely eliminated by a dynamic trading strategy, and this led to a
fair price for the option. 

\index{stochastic systems}
\index{optimal control}
\index{dynamics!economic systems}
The general problem of pricing and hedging an option is one of optimal
stochastic control, and involves dynamically trading financial assets
to achieve desired payoffs or risk profiles.  When placed in this
control theory framework, the quantities of various assets held in a
portfolio become the decision variables, the price movements (often
random) of the assets are the dynamics of the system, and achieving a
desired risk profile is the objective.  In structure, they tend to
deviate from control problems involving physical systems due to the
fact that the dynamics of the system are dominated by uncertainty.
That is, the movement of prices is modeled in a highly stochastic
manner.
\index{uncertainty!in economic systems}

\index{operations research}
\CDS{} problems in finance, especially those related to hedging and
pricing derivative securities, present a number of interesting
challenges for the operations research and \cds{} communities.

\index{securities, financial}
The securities being offered in the financial marketplace are becoming
increasingly complex.  That means that the pricing and hedging of
these securities is also becoming increasingly complex.  Examples
already in existence include options that allow the holder to decide
when to exercise the option, options on averages of prices, options on
baskets of securities, options on options, etc.\ and these options can
be written on stocks, futures, interest rates, the weather,
earthquakes and catastrophes, etc.  Hedging of these options is a
challenging and rather daunting task for stochastic control theory.
\index{options, financial}

\index{modeling!economic systems}
The lack of robustness of dynamic schemes in use during the 1987 crash
was another critical factor.  Since modeling is itself a difficult
problem, it is important that control schemes work well in the
presence of modeling errors.  This is especially true in finance,
where uncertainties can be large and time varying. Often this
uncertainty is handled in a stochastic manner.  
\index{uncertainty!in economic systems}
For instance, some
models in finance assume that the volatility of an asset is
stochastic. This has been used to explain certain deviations between
markets and theory.  More recently, researchers have been developing
control and hedging schemes which explicitly account for model errors
and uncertainty, and are designed to work well in their presence.
This will be an area in which robust control theory has a great amount
to contribute.

\subsection*{Electromagnetics}  
\index{electromagnetics}

\index{adaptive optics}
\index{wavefront control}
\index{antenna arrays}
The development of adaptive optics and phased
array antennas have created new opportunities for active wavefront control in
a variety of applications.  Cancellation of atmospheric effects is already
being used in scientific and military applications and more sophisticated uses
of \cds{} are now being considered.  
%%
%% The following vignette describes the potential for \cds{} in stealth
%% applications.
%%
%% input{albanese-stealth.tex}
%%
One potential area for new work is in the area of active electromagnetic
nulling_{electromagnetic nulling, active} for stealth applications.

To avoid detection and targeting, great strides have been achieved in reducing
the ^{radar} cross section of military systems_{defense systems}.  Perhaps the best known advance
has been the use of angularity and radar absorbing materials to minimize the
detection of fighter aircraft.  The narrow forward profile of the ^{stealth}
fighter is very effective in minimizing radar reflection.  However, there are
many limitations of this approach.  Radar cross section increases whenever the
pilot performs a banking turn and radar absorbing materials used are expensive
and susceptible to moisture.  Furthermore, ^{multistatic radar systems}, which
can increasingly be built inexpensively, effectively track a stealth fighter,
and engine exhaust infrared signatures represent serious system
vulnerabilities.

\index{antenna arrays}
Rather than use angularity to deflect incoming tracking or targeting
radiation, a different approach is to develop inexpensive antenna arrays that
will actively null incoming radiation.  The use of ferrite structures in
antennas could allow extremely rapid change of their radiating and receive
properties.  This would in turn allow arrays of such antennas to be used to
intelligently respond to the surrounding electromagnetic environment by
increasing the self-absorption of impinging radiation and by in turn radiating
a field that will null further incoming radiation.

\index{directed energy systems}%
\index{cellular phones}
\index{distributed control}
\index{communications!systems}
\index{multistatic radar systems}%
\index{radar!multistatic}
Several challenges must be overcome to achieve this goal, including
distributed control system theory to define the currents applied to the
radiating antenna to null the incoming radiation.  The problem of field
sensing_{sensing!electromagnetic fields} and prediction in order to control
its subsequent evolution is a 
significant mathematical and electrical engineering challenge.  Advances in
this area could have other applications in cellular phone communications
systems, adaptive multistatic radar systems, and other directed energy
systems. 

\index{infrared signatures}
\index{gas turbine engines}
Infrared exhaust signatures are another possible application area for active
nulling.  The complex flame dynamics in a gas turbine engine result in a
statistically rich infrared field.  A fundamental question is whether a finite
array of infrared sensors and laser diodes could be used to sense,
characterize, and control this electromagnetic structure.

\subsection*{Molecular, Quantum and Nanoscale Systems} 
\index{nanotechnology}
\index{quantum systems}

Control of molecular,
quantum and nanoscale systems is receiving increased attention as our ability
to sense and actuate systems at this scale improves.  Recent progress in
computational chemistry and ^{physics} has enabled the predictive
^{simulation} of 
nanoscale materials behavior and processing, in systems ranging from
nanoparticles to semiconductor heterostructures to nanostructured bulk
materials.  With this physical understanding and a mathematical model, it is
now possible for formulate optimization and control questions for nanoscale
materials and systems.  Applications include design of nanostructured
materials, precision measurement, and quantum information processing.

\index{actuation!nanoscale}
\index{coatings}
Macroscopic ^{materials} are well-described by their bulk properties, but as a
structure's size shrinks to nanometers, bulk descriptions no longer capture
the relevant physics.  Surface effects become increasingly important and alter
the electronic properties.  These new properties may be exploited in a variety
of engineering applications, from quantum dot lasers to ultra-hard coatings. A
major challenge in exploiting these unique features is the ability to
manufacture materials at the nanometer scale using high-throughput
^{manufacturing} processes.  Improved first-principles models, new techniques
for 
data rich sensing and in-situ diagnostics, design of new actuation approaches,
and algorithms for controlling microscale phenomena are required and the
\cds{} community can be a major contributer to progress in this area.
\index{data rich systems}
\index{diagnostics}

\index{adaptation}%
\index{closed loop control}
\index{learning}%
There are many open questions in the control of phenomena at this scale.
Brown and Rabitz~\cite{BR02-jmc} divide these into three categories: control
law design, closed loop implementation, and identification of the system
Hamiltonian.  New results in controllability, optimal control theory,
adaptation and learning, and ^{system identification} are required to make
progress in this area.  What makes the problem difficult is the use of quantum
wave interference as a mechanism for achieving prescribed control objectives
such as the selective dissociation of a polyatomic molecule or the
manipulation of wavepackets in semiconductors.  Recent experimental successes
(see~\cite{BR02-jmc} for more details) include cleaving and rearranging
selected chemical bonds, control of fluorescence in polyatomic molecules and
enhanced radiative emission in high harmonic generation.

\index{quantum systems}
Control of quantum systems also provides a new set of tools for understanding
nature, as described in the vignette on quantum measurement and control
(page~\pageref{quantum.vig}).

\index{energy systems|(}
\subsection*{Energy Systems} \CDS{} has always been a central element in the
design of large scale energy systems.  From their origins as single generators
connected to associated loads, power systems began around 70 years ago to
evolve into more broadly interconnected systems, motivated among other things
by ^{reliability} (loads are now not totally dependent on any particular
generator).  Recent outage events have highlighted, however, that reliability
is a more subtle question, and in fact global connectivity can lead to the
multiplication of failures.

\index{power grid, electrical}
At same time, the industry is currently undergoing deregulation, which could
easily lead to a loosening of control and a shortage of system information
(even about neighbors), elements which are key to the successful containment
of failures. There is a significant risk that without a technological effort
to improve the ^{reliability} of the network against failure, we can expect
increased vulnerability of this fundamental infrastructure.  One aspect of
this effort concerns the design and management policies for the grid.  This
includes network expansions, or the deployment of new technological
capabilities such as Flexible AC Transmission Systems (FACTS), and the
decisions on load distribution to service the required demand.

\index{communications!systems}
\index{networks}
Another area where fundamental research can have significant impact is
real-time_{real-time systems} control for correct dynamic ^{regulation} and
event-management for the 
containment of failures.  There are increased linkages between computer and
communications systems with the electric power grid, including the Internet
and communications by the utilities to the in-orbit satellite_{satellites}
network used for 
the Wide Area Measurement System (WAMS).  This increased connectivity allows
the possibility for significant local processing power at each generator,
connected to a global data network providing information about the current
state of the grid.

\index{cascade failures}
\index{complex systems}
The technological challenges posed by such a system are multiple. A first
challenge, even assuming a free and instantaneous flow of information, is to
develop ^{simulation} capabilities for analysis of faults with the potential
for 
large-scale cascading failures. Note that after a major outage it is possible
to go back over the data and obtain a simple explanation. However,
going beyond post mortems and into prediction will require truly revolutionary
advances in basic simulation capability, indeed a completely new way of doing
simulations of complex systems, going beyond both the traditional
top-down approach that has dominated scientific computing, as well as
the more recent bottom-up style of agent-based methods.

\index{protocols}
\index{autonomous systems}
\index{cascade failures}
Secondly, distributed software protocols must be implemented for event-flow
management and distributed ^{simulation}. A great challenge in this regard is
that both the autonomy and the economic interests of the individual power
companies must be protected. In other words, the distributed software system
must be structured so that clear boundaries are drawn between serving the
common good and protecting proprietary and economic information.  Finally,
note that with this network we are actually increasing interconnectedness, a
driving factor of cascading failures. One should be wary of this process, which
could lead to failures propagating through both networks. In particular,
software systems should incorporate robustness against these kinds of events.

\index{distributed control}
\index{dynamical systems}
Progress in this area will require research progress in control theory and
dynamical systems, along with the insight of statistical physics and new
results in numerical methods, distributed software, computer networks and
power systems. Each of these areas is relevant, and experts in each domain
must expand their vision to tie to the other areas.  As we evolve toward a
``mega-infrastructure'' that includes energy (including the electric grid,
water, oil and gas pipelines), telecommunications, transportation, and
electronic commerce, concepts from \cds{} will become even more important and
will present significant challenges to the research community.

\index{fuel cells}
\index{gas turbine engines}
There are also many applications in energy systems that fall under more
traditional paradigms, such as advanced control of gas turbines for power
generation and fuel cells.  Both represent complex chemical and fluid systems
that must operate in a variety of environmental conditions with high
reliability and efficiency.  Fuel cells present particularly difficult
challenges due to the tightly coupled nature of the underlying thermal, fluid,
and chemical processes.  In mobile applications, such as automobiles, control
is important for quickly bringing the fuel cell to the desired operating point
and maintaining the operation of the cell under changing loads.

\index{energy systems|)}
%% \include{outreach}		% Education and Outreach
\clearpage
% Master File: cdspanel.tex
\cleardoublepage
\chapter{Education and Outreach}
\label{outreach.sec}

\index{education|(}

\index{National Science Foundation (NSF)}
\index{Control Systems Society (CSS), IEEE}
\CDS{} education is an integral part of the community's activities and
one of its most important mechanisms for transition and impact.  In
1998, the National Science Foundation (NSF) and the IEEE Control Systems
Society (CSS) jointly sponsored a workshop in control engineering education
which made a number of recommendations for improving \cds{} education
(see~\cite{nsfcss99} and Appendix~\ref{nsfcss.sec}).  This section is
based on the findings and recommendations of that report, and on
discussions between Panel members and the \cds{} community.  The Panel would
particularly like to thank Jim Batterson for his contributions to this
chapter. 

\section{The New Environment for Control Education}

\index{courses!shared}
\CDS{} is traditionally taught within the various engineering disciplines that
make use of its tools, allowing a tight coupling between the methods of \cds{}
and their applications in a given domain.  It is also taught almost
exclusively within engineering departments, especially at the undergraduate
level.  Graduate courses are often shared between various departments and in
some places are part of the curriculum in ^{applied mathematics} or
^{operations 
research} (particularly in regards to ^{optimal control} and ^{stochastic
systems}). 
This approach has served the field well for many decades and has trained an
exceptional community of control practitioners and researchers.

\index{systems engineering}
Increasingly, the modern control engineer is put in the role of being a
systems engineer, responsible for linking together the many elements of a
complex product or system.  This requires not only a solid grounding in the
framework and tools of \cds, but also the ability to understand the technical
details of a wide variety of disciplines, including physics, chemistry,
electronics, computer science, and operations research.

\index{new domains, control in}
In addition, \cds{} is increasingly being applied outside of its traditional
domains in aeronautics, chemical engineering, electrical engineering and
mechanical engineering.  Biologists are using ideas from \cds{} to model and
analyze cells and animals; computer scientists are applying \cds{} to
the design of routers and embedded software; physicists are using \cds{} to
measure and modify the behavior of quantum systems; and economists are
exploring the applications of feedback to markets and commerce.
\index{biology}
\index{computer science}
\index{physics}
\index{economics}
\index{quantum systems}

\index{discipline boundaries}
\index{control!challenges}
\index{curriculum development}
This change in the use of \cds{} presents a challenge to the community.  In
the United States, discipline boundaries within educational institutions are
very strong and it is difficult to maintain a strong linkage between \cds{}
educators and researchers across these boundaries.  While the \cds{}
community is large and prosperous, \cds{} is typically a small part of any
given discussion on curriculum since these occur within the departments.
Hence it is difficult to get the resources needed to make major changes in
the \cds{} curriculum.  In addition, many of the new applications of
\cds{} are outside of the traditional disciplines that teach \cds{} and it
is hard to justify developing courses that would appeal to this broader
community and integrate those new courses into the curricula of those
other disciplines (e.g., biology, physics, or medicine).
\index{courses!broader audiences}

In order for the opportunities described elsewhere in this report to be
realized, \cds{} education must be restructured to operate in this new
environment.  Several universities have begun to make changes in the way that
\cds{} is taught and organized and these efforts provide some insights into
how this restructuring might be done successfully.

\index{cross-disciplinary research centers}
\index{University of Maryland}
\index{University of Illinois, Urbana-Champaign}
\index{University of California, Santa Barbara}
Often the first step is establishing a cross-disciplinary research
center, where there is a larger critical mass of \cds{} researchers.
Examples include the Coordinated Science Laboratory (CSL) at the
University of Illinois, Urbana-Champaign, the Center for Control
Engineering and Computation (CCEC) at the University of California,
Santa Barbara, and the Institute of Systems Research (ISR) at the
University of Maryland.  
These centers coordinate research activities, organize
workshops and seminars, and provide mechanisms for continuing interactions
between \cds{} students and faculty in different departments.

\index{courses!shared}
A second step is the establishment of shared courses between the
disciplines, often at the graduate level.  These shared courses encourage a
broader view of \cds{} since the students come from varying
backgrounds.  They also provide an opportunity for the larger \cds{}
community at the university to establish active dialogs and provide a
mechanism for sharing students and building joint research activities.  Many
U.S.\ universities have adopted this model, especially for theory oriented
courses. 

\index{M.S.\ programs}
\index{Ph.D.\ programs}
\index{Caltech}
\index{discipline boundaries}
\index{Washington University}
Finally, some schools have established a separate M.S.\ or Ph.D.\ program in
\cds. 
These are common in Europe, but have been much less prevalent in the United
States, 
partly due to the traditional discipline structure around which most
universities are organized.  Examples in the U.S.\ include the Control and
Dynamical Systems (CDS) program at Caltech and the Department of Systems
Science and Mathematics (SSM) at Washington University.  The advantage of a
separate graduate program in \cds{} is that it gives the faculty better
control over the curriculum and allows a less discipline-centric approach to
control. 

\index{regional alliances}
One other mechanism, popular in Europe but not yet established in the United
States, is the creation of regional \cds{} alliances that build critical mass
by linking together multiple universities in a geographic region.  This
mechanism is used very effectively, for example, in the Netherlands through
the Dutch Institute of Systems and Control (DISC).\footnote{\tt
http://www.disc.tudelft.nl} With the increased availability of real-time
audio, video, and digital connectivity, it is even possible to create virtual
alliances---with shared
classes, reading groups, and seminars on specialized topics---linking sites
that are not physically near each other.

\index{outreach|(}

\section{Making \CDS{} More Accessible}
\index{accessibility, broadening|(}

\index{control!principles}%
Coupled with this new environment for \cds{} education is the clear
need to make the basic principles of feedback and control known to a
wider community.  As the main recommendations of the Panel illustrate,
many of the future opportunities for \cds{} are in new domains and the
community must develop the educational programs required to train the
next generation of researchers who will address these challenges.
\index{new domains, control in}

\index{textbooks}
\index{discipline boundaries}
A key element is developing new books and courses that emphasize feedback
concepts and the requisite mathematics, without requiring that students come
from a traditional engineering background.  As more students in ^{biology},
^{computer science}, environmental science_{environmental science},
^{physics}, and other disciplines 
seek 
to learn and apply the methods of \cds, the \cds{} community must explore new
ways of 
providing the background necessary to understand the basic concepts and apply
some of the advanced tools that are available.  
Textbooks that are aimed at
this more general audience could be developed and used in courses that target
first year biology or computer science graduate students, who may have very
little background in continuous mathematics beyond a sophomore course in
scalar ordinary differential equations (ODEs) and linear algebra.
\index{differential equations}

The following vignette describes one attempt to make \cds{} more
accessible to a broader community of research scientists and
engineers.

%% \input{murray-cds110.tex}
% Master File: cdspanel.tex

\index{Murray, R. M.}
\index{Morgansen, K. A.}
\vignette{CDS 110: Introduction to Control Concepts, Tools, and Theory
(Kristi Morgansen and Richard Murray, Caltech)} 
{
\index{accessibility, broadening}
\index{courses!undergraduate}
\index{Caltech}
The Control and Dynamical Systems Department at Caltech has recently
undertaken a revision of its entry level graduate courses in \cds{}
to make them accessible to students who do not have a traditional background
in chemical, mechanical, or electrical engineering.  The current
course, CDS 110, is taken by senior undergraduates and first year
graduate students from all areas of engineering, but has traditionally
not been easily accessible to students in scientific disciplines,
due to its heavy engineering slant.  With the increased interest in
\cds{} from these communities, it was decided to revise the course so
that it could not only continue to serve its traditional role, but also
provide an 
introduction to \cds{} concepts for first year graduate students in
biology, computer science, environmental engineering, and physics.
\index{discipline boundaries}

\index{control!principles}%
\index{feedback}
The goal of the course is to provide an understanding of the
principles of feedback and their use as a tool for altering the dynamics
of systems and managing uncertainty.  
\index{uncertainty management}
\index{dynamics}
\index{modeling}
The main topics of the course
are modeling, dynamics, interconnection, and robustness of
feedback systems.  On completion of the course, students are able to
construct control-oriented models for typical systems found in
engineering and the sciences, specify and describe performance for
feedback systems, and analyze open loop and feedback behavior of such
systems.  Central themes throughout the course include input/output
response, modeling and ^{model reduction}, linear versus nonlinear
models, and local versus global behavior.

The updated version of the course has two ``tracks'': a conceptual
track and an analytical track.  The conceptual track is geared toward
students who want a basic understanding of feedback systems and the
computational tools available for modeling, analyzing, and designing
feedback systems.  The analytical track is geared toward a more
traditional engineering approach to the subject, including the use of
tools from linear algebra, complex variables, and ordinary
differential equations (ODEs).  Both tracks share the same lectures, but the
supplemental reading and homework sets differ.
\index{differential equations}

In addition to the main lectures, optional lectures are given by
faculty from other disciplines whose research interests include
\cds{}.  Hideo Mabuchi (Physics) and Michael Dickinson (Biology) are
two such lecturers and they provide examples of some applications of
feedback to a variety of scientific and engineering applications.
These lectures are used to emphasize how the concepts and tools are
applied to real examples, drawn from areas such as aerospace,
robotics, communications, physics, biology, and computer science.

The first iteration of the course, taught in 2001--02, succeeded in
developing a set of conceptual lectures (given as the first lecture in
the week) that introduced the main ideas of \cds{} with minimal
mathematical background.  Based on these lectures, students are able
to use the tools of \cds{} (e.g., ^{MATLAB} and ^{SIMULINK}) and understand
the results.  Two additional lecture hours per week are used to provide the
more 
traditional mathematical underpinnings of the subject and to derive
the various results that are presented in the conceptual lectures.

In the second iteration of the course, to be taught in 2002-03, we
intend to refine the lectures and put more effort into dividing the
class into sections based on research interests.  Individual lectures in the
sections will then be used to build the necessary background (for
example, providing a refresher on linear algebra and ODEs for
biologists and computer scientists) or to provide additional
perspectives (for example, linking transfer functions to Laplace
transforms in a more formal manner).  
}
 
\index{curriculum development}
\index{dynamics}
\index{linear systems}
\index{courses!undergraduate}
In addition to changes in specific courses on \cds, universities could also
integrate modules on dynamics and control into their undergraduate mathematics
and science curricula.  Any linear algebra course could be strengthened by the
addition of a short lesson on linear systems, eigenvalues, and their physical
interpretation and manipulation through feedback.  Freshman ^{physics} could be
enriched by extending lessons on mechanical oscillators to the problem of
balancing an inverted pendulum or the stability of person riding a bicycle.

\index{software!control}
\index{classical control}
\index{modern control}
The \cds{} community also must continue to implement its tools in software, so
that they are accessible to users of \cds{} technology.  While this has
already occurred in some areas of control (such as classical and modern linear
control theory), there are very few general purpose software packages
available for analysis and design of nonlinear, \index{nonlinear control}%
adaptive, \index{adaptive control}%
and hybrid \index{hybrid systems}%
systems---and many of these are not available on general purpose platforms
(such as ^{MATLAB}).  These tools can be used to allow non-experts to apply the
most advanced techniques in the field without requiring that they first obtain
a Ph.D.\ in \cds.  Coupled with modeling and ^{simulation} tools, such as
^{SIMULINK} 
and Modelica, these packages will be particularly useful in teaching the
principles of feedback by allowing exploration of relevant concepts in a
variety of domains.
\index{Modelica}

\index{accessibility, broadening|)}

\section{Broadening \CDS{} Education}

\index{accessibility, broadening}
\index{curriculum development}
\index{complex systems}
In addition to changes in the curriculum designed to broaden the accessibility
of \cds, it is important that \cds{} students also have a broader grasp of
engineering, science, and mathematics.  Modern \cds{} involves the development
and implementation of a wide variety of very complex engineering systems and
the \cds{} community has been a major source of training for people who
embrace a systems perspective.  The curriculum in \cds{} needs to reflect this
role and provide students with the opportunity to develop the skills necessary
for modern engineering and research activities.

\index{textbooks}
\index{compactification}
At the same time, the volume of work in \cds{} is enormous and so effort must
be placed on unifying the existing knowledge base into a more compact form.
There is a need for new books that systematically introduce a wide range of
control techniques in an effective manner.  This will be a major
undertaking, but 
is required if future students of \cds{} are to receive a concise but
thorough 
grounding in the fundamental principles underlying \cds, so that they can
continue to extend the research frontier beyond its current boundary.
\nocite{Lev99-ctrlhbook}
\index{control!principles}%


\index{systems perspective}
\index{complex systems}
Increasingly, \cds{} engineers are playing the role of ``systems integrator''
in large engineering projects.  This occurs in part because they bring systems
insight that is required for successful operation of a complex engineering
product, but also because \cds{} is often the glue that ties together the
components of the system (often in the form of embedded control software).
Unfortunately, most \cds{} curricula do not emphasize the types of leadership
and communications skills that are critical for success in these environments.

\index{teams!working in}
A related aspect of this is strengthening the skills required for working in
teams.  All modern systems design is done in interdisciplinary teams and it
requires certain skills to understand how to effectively interact with domain
experts from a wide variety of disciplines.  
\index{domain knowledge}
Project courses are an effective
mechanism for developing this type of insight and these should be more
aggressively incorporated into \cds{} curricula at both undergraduate and
graduate levels.  Another effective mechanism is participation in national
competitions where \cds{} tools are required, such as 
RoboCup\footnote{\tt http://www.robocup.org} and 
FIRST\footnote{\tt http://www.usfirst.org}. 
\index{RoboCup}
\index{FIRST}
\index{competitions, student}

\index{domain knowledge}
It is also important that \cds{} students be provided with a balance between
theory, applications, and computation.  In particularly, it is essential that
\cds{} students build a deep domain knowledge in one or more disciplines, so
that they understand how this knowledge interacts with the \cds{} methodology.
Independent of the specific domain chosen, this approach provides a context
for understanding other engineering domains and developing \cds{} practices and
tools that bridge application areas.

\index{experiments}
Experiments continue to form an important part of a \cds{} education and
projects should form an integral part of the curriculum for
both undergraduate and graduate students.  Shared laboratories within
individual colleges or universities as well as laboratories shared among
different universities could be used to implement this (with additional
benefits in building cross-disciplinary and cross-university interactions).
New experiments should be developed that explore the future frontiers of
\cds{}, including increased use of computing, communications and networking,
as well as exploration of control in novel application domains.
\index{new domains, control in}

\section{The Opportunities in K-12 Math and Science Education}
\index{K-12 education}
\index{math and science education}

\index{antilock brakes}
\index{experiments}
\index{microprocessors!use, in control}
Much as computer literacy has become commonplace in our K-12 curriculum, an
understanding of the requirements, limits, and capabilities of control should
become part of every scientifically literate citizen's knowledge.  Whether it
is understanding why you should not pump antilock brakes or why you need to
complete a regimen of antibiotics through the final pills even after symptoms
disappear, an understanding of dynamics and control is essential.  
\index{dynamics}
The
development of inexpensive microprocessors, high-level computer languages, and
graphical user interaces (GUIs) has made the development of test apparatus and
small laboratories for 
rudimentary control experiments and demonstrations available within the
budgets of all school districts.  The U.S.\ National Science Foundation
recognizes the importance of its funded programs impacting the general public
through its ``Criterion 2'' (Broader Impacts) in the evaluation of all
submitted proposals.  Because of the broad applications of \cds{} to the
public good and standards of living, it is important to develop a curriculum
for inclusion in pre-college (K-12) education.
\index{National Science Foundation (NSF)}
\index{curriculum development!K-12}

\index{teaching materials}
\index{discipline boundaries}
\index{Consortium for Mathematics and Its Applications (COMAP)}%
Currently, mathematics, science, and computer technology are taught in
separate departments in the vast majority of K-12 curricula.  Even sciences
are compartmentalized at many schools.  As at universities, the
multidisciplinary nature of \cds{} is very much antithetical to that
traditional thinking and structure in K-12 education.  However, there is some
evidence of advances toward application and integration of mathematics with
science. The Consortium for Mathematics and Its Applications
(COMAP)\footnote{\tt http://www.comap.com}, which develops curriculum
materials and teacher development programs in mathematics, is one
example. Indeed, the leveraging of 
efforts with COMAP could prove fruitful and the \cds{} community could work
with COMAP to enhance the current textbooks and curricula that have been
developed 
by that consortium over the past two decades.  Another resource is the
Eisenhower National Clearinghouse,\footnote{\url{http://www.enc.org}} which maintains a database of teaching modules and resources for
K-12 math and science education.
\index{Eisenhower National Clearinghouse}

\index{thermostat}
\index{governors}
In the \cds{} arena, simple experiments involving governors, thermostats, and
``see-saws'' can be performed as early as elementary school to illustrate the
basic 
concepts of \cds{}.  As mathematical sophistication increases through middle
school and high school, quantitative analysis can be added and experimentally
verified.  Some schools are beginning to teach calculus in the junior year and
so a post-calculus course in ^{applied mathematics} of differential equations
and 
dynamical systems could be created bridging chemistry, physics, biology, and
mathematics.
\index{differential equations}
\index{dynamical systems}
\index{biology}
\index{physics}
\index{chemistry}

\index{teaching materials}
\index{teachers, K-12}
Complementary to the development of educational materials and experiments, it
is also important to provide K-12 teachers with the opportunities to learn
more 
about \cds{}.  As an example of how this could be done, NASA Langley Research
Center sponsored a program for teachers under the auspices of the HPCCP (High
Performance Computing and Communications Program) several years ago.  In this
program teachers from six school districts spent 8 weeks learning the state of
the art in computer hardware and software for engineering and science.
Most days were spent with new material delivered in a lecture or laboratory
environment in the morning with a ``homework'' laboratory in the afternoons.
Teachers were paid a fellowship that approximated the per diem rate of
entry-level teachers. This type of residential environment allowed for a total
immersion in the material.  In addition to becoming familiar with
research-grade hardware and software and the Internet, the participants formed
partnerships with each other that promoted continued collaboration throughout
the coming academic years.

\index{curriculum development!K-12}
There are numerous curriculum development and general education meetings and
conferences throughout the country each year.  In particular, most states have
an active association of school boards and there is a National School Boards
Association.  A presentation at these meetings would communicate directly with
the policy and decision makers.  Such a presentation would have to be tailored
for the lay person but might produce a pull to match a push from one of the
ideas 
above.

\section{Other Opportunities and Trends}

In addition to the specific opportunities for education and outreach described
above, there are many other possible mechanisms to help expand the
understanding and use of \cds{} tools.

\subsection*{Popular Books and Articles}

\index{Scientific American}%
\index{cruise control}
In September 1952, {\it Scientific American} published an entire
issue dedicated to Automatic Control~\cite{sciam52}.  The issue highlighted
the role 
that \cds{} was playing in the new advancements of the time,
particularly in ^{manufacturing}.  The introduction of cruise control
(originally called Autopilot) a few years later provided direct
experience with the main principles of feedback.
\index{control!principles}%


\index{amplifiers}
\index{congestion control}
\index{feedback}
\index{control!successes of}
\index{communications!systems}
Since that time, \cds{} has become less and less visible to the
general public_{public awareness}, perhaps in part because of its success.
Individuals 
interact with control systems and feedback many times every day, from
the electronic amplifiers, tuners, and filters in television and
radio, to congestion control algorithms that enable smooth Internet
communications, to flight control systems for commercial aircraft.
Yet most people are unaware of \cds{} as a discipline.  Other fields,
such as artificial intelligence, robotics, and computer science have
often been given credit for ideas whose origins lie within the \cds{}
community.
\index{control!as hidden technology}
\index{artificial intelligence}
\index{robotics}
\index{computer science}

There is a great need to better educate the public on the successes
and opportunities for \cds{}.  This ^{public awareness} is increasingly
important in the face of decisions that will need to be made by
government funding agencies about support for specific areas of
research.

The use of any number of popular outlets for communication can reach
this group.  Many local newspapers now have a science page or
section on a weekly basis.  The development of a popular level series
of articles on dynamics and control could be prepared for these pages.
\index{dynamics}
The New York Times publishes a science section every Tuesday; a
series of articles could be developed for this section spanning
several weeks.  A number of science museums have been developed across
the nation in recent years and many of these museums are allied through
professional associations.  The development of interactive dynamics
and control displays for these museums would be beneficial to the
museum by giving them a new exhibit and the displays reach the entire age
range of 
the public from children through adults.

\index{books!popular}
\index{textbooks}
Books written for non-specialized audiences and chapters in high school
textbooks are another mechanism for increasing the understanding of \cds{}
principles in the general population.  
\index{dynamical systems}
The dynamical systems community has
been very successful in this regard, with many books available on chaos,
complexity theory, and related concepts.  Currently available books on control
include books on the history of control~\cite{Ben79,Ben86,May70} and a book
entitled ``Out of Control''~\cite{Kel94} that discusses many control concepts.
\index{control!history of}

\subsection*{Multimedia Tools}
\index{multimedia tools}

\index{teaching materials}
There is an increasing need for educational materials that can be used
in a variety of contexts for communicating the basic ideas behind \cds.
One possible mechanism is to develop a multimedia CDROM that would
include materials on the history and concepts of \cds, as well as
tutorial material on specific topics and public domain software tools
for control analysis and design.

The fluid mechanics community has recently developed such a multimedia CDROM
that can be used as a supplement to traditional courses in fluid
mechanics~\cite{Hom+00-fluidsCD}.  It contains historical accounts of fluid
mechanics, videos and animations of important concepts in fluids, and detailed
descriptions of fundamental phenomena.  It can be purchased through university
bookstores or online from Amazon.com.

\index{Rugh, W. J.}
One initial activity in developing such tools for \cds{} has been made by
Wilson J. Rugh at Johns Hopkins University, who has created a series of
interactive demonstrations of basic concepts of control that can be executed
over the web.\footnote{\tt http://www.jhu.edu/\~{}signals} Modules include
Fourier analysis, convolution, the sampling theorem, and elementary control
systems.  One of the most sophisticated tools demonstrates robust
stabilization, including the ability to specify an uncertainty weight by moving
poles and zeros of the weighting transfer function with the mouse.  A
controller can then be designed by dragging the compensator poles and zeros to
achieve robust, closed loop stability.

\subsection*{Software}  
\index{software!control}
\index{classical control}
\index{modern control}

\index{control!successes of}
One of the success stories of control is the wide availability of commercial
software for modeling, analyzing, designing, and implementing control
systems.  The Controls Toolbox in ^{MATLAB} provides the basic tools of
classical 
and modern control and many other toolboxes are available for more
implementing more specialized theory.  These toolboxes are used throughout
academia, government, and industry and give students, researchers, and
practitioners access to powerful tools that have been carefully designed and
tested.

Despite the impressive current state of the art, much of this software is
restricted to a very small class of the systems typically encountered in \cds{}
and there are many gaps that will need to be filled to enable the types of
applications described in the previous chapter.  One area where substantial
progress has been made recently is in modeling tools, where there are several
software packages available for modeling, ^{simulation}, and analysis of
large-scale, ^{complex systems}.  
\index{Modelica}%
\index{modeling}
One such is example is {\em Modelica,}\footnote{\tt http://www.modelica.org}
which provides an object oriented language for describing complex physical
systems.  Modelica is particularly noteworthy because it was designed to model
systems with algebraic constraints, allowing a much richer class of systems to
be represented.

\index{nonlinear control}
\index{information-based systems}
\index{hybrid systems}
Additional tools are needed for control-oriented modeling, analysis, and
synthesis of nonlinear and hybrid systems, particularly those that have a
strong interaction with information rich systems, where good scaling
properties are required.  As yet, there is not a
standard representational framework for such systems (beyond symbolic
representations) and hence software tools for nonlinear or hybrid analysis are
much less used than those for linear systems.  One of the main issues here is
to capture the relevant dynamics in a framework that is amenable to
computation.  
\index{dynamics}
Analysis and synthesis must be able to handle systems 
containing table lookups, logical elements, time delays, and models for
computation and communication elements.
\index{time delay}

\index{control!impact and payoffs}
The payoff for investing in the development of such tools is clear: it brings
the advanced theoretical techniques that are developed within the community to
the people who can most use those results.

\subsection*{Interaction with Industry and Government}  
\index{industry interaction}

Interaction with industry is an important component of any engineering
research or educational activity.  The \cds{} community has a strong history
of impact on many important problems and industry involvement will be critical
for the eventual success of the future directions described in this report.
This could occur through cooperative ^{Ph.D.\ programs} where industrial
researchers 
are supported half by companies and half by universities to pursue Ph.D.'s
(full-time), with the benefits of bringing more understanding of real-world
problems to the university and transferring the latest developments back to
industry.  In addition, industry leaders and executives from the \cds{}
community should continue to interact with the community and help communicate
the needs of their constituencies.

The NSF/CSS workshop also recognized the important role that industry plays
and recommended that educators and funding organizations
\begin{quote}
  encourage the development of WWW-based initiatives for technical information
  dissemination to industrial users of control systems and encourage the
  transfer of practical industrial experience to the
  classroom~\cite{nsfcss99}. 
\end{quote}
The further recommended that cooperative efforts between academia and
industry, especially in terms of educational matters, be significantly
expanded.

\index{International Federation of Automatic Control (IFAC)}
\index{IFAC Professional Briefs}%
The International Federation of Automatic Control (IFAC) is creating a
collection of IFAC Professional Briefs.  These Professional Briefs are aimed
at a readership of general professional control engineers (industrial and
academic), rather than specialist researchers.  The briefs provide an
introduction and overview of a ``hot topic,'' illustrative results, and a
sketch of the underlying theory, with special attention given to providing
information sources such as useful Internet sites, books, papers, etc.  Eight
titles have been selected to launch the Professional Briefs series: 
\begin{quote}
\begin{tabular}{l}
  Computer Controlled Systems \\
  PID Auto-Tuning \\
  Control of Biotechnological Processes \\
  Control Busses and Standards \\
  {Physical-Based Modeling of Mechatronic Systems} \\
  {Genetic Algorithms in Control Systems Engineering} \\
  {Low Cost Automation in Manufacturing} \\
  {Engineering Dependable Industrial Real-Time Software.}
\end{tabular}
\end{quote}
\index{biotechnology}
\index{mechatronics}
\index{real-time systems}

\index{government interaction|(}
\index{national laboratories}
\index{technology transfer}
Another avenue for interaction with industry is through the national
laboratories.  In the United States, many government laboratories have summer
faculty programs and student internships.  Extended visits serve not only to
transfer ideas and technology from research to application, but also provide a
mechanism for understanding problem areas of importance to the government and
the military.  The U.\ S.\ Air Force Research Laboratory has been particularly
active in bringing in visitors from universities and provides an example of
successful interchange of this kind
\index{Air Force Research Laboratory (AFRL)}

Finally, there are many opportunities for \cds{} researchers to participate in
government service.  This can range from serving on review committees and
advisory boards to serving as a program manager at a funding agency.  Active
participation by the \cds{} community is essential for building understanding
and support of the role of \cds{}.
\index{funding agencies, working for}
\index{government interaction|)}

\index{outreach|)}
\index{education|)}
%% \include{recomm}		% Recommendations
\clearpage
% Master File: cdspanel.tex
\cleardoublepage
\chapter{Recommendations}
\label{recomm.sec}

%%
%% Define a macro for the recommendations, so that everything is consistent
%%
\newcommand\recomm[1]{
  \begin{quote}
%   \fbox{\parbox{\linewidth}{\centering #1}}
    \bf #1
  \end{quote}
}

\index{recommendations|(}%
\index{CDS Panel!recommendations|(}%

\CDS{} continues to be a field rich in opportunities.  In order to realize
these opportunities, it is important that the next generation of \cds{}
researchers receive the support required to develop new tools and techniques,
explore new application areas, and reach out to new audiences.  Toward this
end, the Panel developed five major recommendations for accelerating
the impact of \cds{}.


\section{Integrated Control, Computation, Communications}
\index{communications!integration of control and}%
\index{computation!integration of control and}%
\index{computation!ubiquitous computing}
\index{ubiquitous computing}
Inexpensive and ubiquitous sensing, communications, and computation will be a
major 
enabler for new applications of control to large-scale, ^{complex systems}.
Research in control over networks, control of networks, and design of
^{safety}
critical, large-scale interconnected systems will generate many new research
issues and theoretical challenges.  A key feature of these systems is their
^{robust yet fragile behavior}, with cascade failures leading to large
disruptions in performance.  
\index{cascade failures}

\index{computer science!interaction of control and}%
\index{teams!working in}
\index{communications!integration of control and}%
A significant challenge will be to bring together the diverse
research communities in \cds{}, computer science, and communications in order
to build the unified theory required to make progress in this area.  Joint
research by these communities will be much more team-based and will likely
involve groups of domain experts working on common problems, 
in addition to individual investigator-based projects.

To realize the opportunities in this area, the Panel recommends that
government agencies and the \cds{} community
\recomm{
  Substantially increase research aimed at the {\em integration} of control,
  computer science, communications, and networking.
}
In the United States, the Department of Defense has already made substantial
investment in this area through the Multidisciplinary University Research
Initiative (MURI) program and this trend should be continued.  
\index{Multidisciplinary University Research Initiative (MURI)}
It will be
important to create larger, multidisciplinary centers that join \cds{},
computer science, and communications and to train engineers and researchers
who are knowledgeable 
in these areas.

\index{industry interaction}%
Industry involvement will be critical for the eventual success of this
integrated effort and universities should begin to seek partnerships with
relevant companies.  Examples include manufacturers of air traffic control
hardware and software, and manufacturers of networking equipment.

\index{wireless networks}
\index{electrical power}
\index{transportation systems}
\index{control!impact and payoffs}
The benefits of increased research in integrated control, communications, and
computing will be seen in our transportation systems (air, automotive, and
rail), our communications networks (wired, wireless, and cellular), and
enterprise-wide operations and supply networks (electrical power,
manufacturing, service and repair).
\index{enterprise level systems}
\index{supply chains}

\section{Control of Complex Decision Systems}
\index{decision making!systems}%
\index{complex systems}

\index{decision making!higher level}
\index{decision making!logic}
\index{artificial intelligence}
The role of logic and decision making in control systems is becoming an
increasingly large portion of modern control systems.  This decision making
includes not only traditional logical branching based on system conditions, but
higher levels of abstract reasoning using high level languages.  These
problems have traditionally been in the domain of the artificial intelligence
(AI) community, but the increasing role of dynamics, robustness, and
interconnection in many applications points to a clear need for participation
by the \cds{}
community as well.

\index{logistics}
\index{supply chains}
\index{decision making!logic}
\index{uncertainty!in resource allocation}
\index{enterprise level systems}
A parallel trend is the use of \cds{} in very large scale systems, such as
logistics and supply chains for entire enterprises.  These systems involve
decision making for very large, very heterogeneous systems where new protocols
are required for determining resource allocations in the face of an uncertain
future.  Although models will be central to analyzing and designing such
systems, these models (and the subsequent control mechanisms) must be
scalable to {\em very} large systems, with millions of elements that are
themselves as complicated as the systems we currently control on a routine
basis. 

\index{enterprise level systems}%
To tackle these problems, the Panel recommends that government agencies and the
\cds{} community
\recomm{
  Substantially increase research in \cds{} at higher levels of decision
  making, moving toward enterprise level systems.  
}
The extension of control beyond its traditional roots in differential
equations is an area that the \cds{} community has been involved in for many
years and it is clear that some new ideas are needed.  Effective frameworks
for analyzing and designing systems of this form have not yet been fully
developed and the \cds{} community must get involved in this class of
applications in order to understand how to formulate the problem.
\index{differential equations}

\index{testbeds}%
\index{experiments}
\index{adversarial environments}
\index{artificial intelligence}
A useful technique may be the development of experimental testbeds to explore
new ideas. 
In the military arena, these testbeds could consist of collections of unmanned
vehicles (air, land, sea and space), 
\index{unmanned vehicles}
operating in conjunction with human partners_{man-machine systems} and adversaries.  In the commercial sector, service robots_{robotics} and
personal assistants may be a fruitful area for exploration.  
\index{competitions, student}
And in a
university setting, the emergence of robotic competitions is an interesting
trend that \cds{} researchers should explore as a mechanism for developing
new paradigms and tools.  In all of these cases, stronger links with
the AI community should be explored, since that community is currently at the
forefront of many of these applications.

\index{control!impact and payoffs}
The benefits of research in this area include replacing {\it ad hoc} design
methods by systematic techniques to develop much more reliable and
maintainable decision systems.  It will also lead to more efficient and
autonomous 
enterprise-wide systems and, in the military domain, provide new alternatives
for defense that minimize the risk of human life.
\index{autonomous systems}

\section{High-Risk, Long-Range Applications of Control}

\index{new domains, control in}
The potential application areas for \cds{} are increasing rapidly as advances
in 
science and technology develop new understanding of the importance of feedback,
and new sensors and actuators allow manipulation of heretofore unimagined
detail.  To discover and exploit opportunities in these new domains, experts in \cds{} must
actively participate in new areas of research outside of their traditional
roots.  At the same time, mechanisms must be put in place to educate domain
experts about 
\cds{}, to allow a fuller dialog, and to accelerate the uses of \cds{} across
the enormous number of possible applications.

In addition, many applications will require new paradigms for thinking about
\cds{}.  For example, the traditional notions of signals that encode
information through amplitude and phase relationships may need to be extended
to allow the study of systems where pulse trains or biochemical ``signals''
are used to trace information.
\index{biochemical signals}

One of the opportunities in many of these domains is to export (and expand) the
framework for systems-oriented modeling that has been developed in \cds{}.
The tools that have been developed for aggregation and hierarchical modeling
can be important in many systems where complex phenomena must be understood.
The tools in \cds{} are among the most sophisticated available, particularly
with respect to uncertainty management.
\index{uncertainty management}

\index{quantum systems}%
\index{biology}%
\index{nanotechnology}%
\index{environmental science}%
To realize some of these opportunities, the Panel recommends that government
agencies and the \cds{} community
\recomm{
  Explore high-risk, long-range applications of \cds{} to new domains such as
  nanotechnology, quantum mechanics, electromagnetics, biology, and
  environmental science. 
}
\index{discipline boundaries}
A challenge in exploring new areas is that experts in two (or more)
fields must come together, which is often difficult under mainly
discipline-based funding constructs.  There are a variety of mechanisms that
might be 
used to do this, including dual investigator funding through \cds{} programs
that pay for biologists, physicists, and others to work on problems
side-by-side with \cds{} researchers.  Similarly, funding agencies should
broaden the funding of science and technology to include funding of the \cds{}
community 
through domain-specific programs.
\index{dual investigator funding}

Another need is to establish ``meeting places'' where \cds{} researchers can
join with new communities and each can develop an understanding of the
principles and tools of the other.  This could include focused workshops 
\index{focused workshops}%
of a
week or more to explore \cds{} applications in new domains or 4--6 week short
courses 
\index{short courses}%
on \cds{} that are tuned to a specific applications area, with
tutorials in that application area as well.

\index{dual appointments}%
\index{cross-disciplinary research centers}
\index{University of California, Santa Barbara}
\index{Caltech}
\index{discipline boundaries}%
At universities, new materials are needed to teach non-experts who want to
learn about \cds{}.  Universities should also consider dual appointments
between science and engineering departments that recognize the broad nature of
\cds{} and the need for \cds{} to not be confined to a single disciplinary
area.  Cross-disciplinary centers (such as the CCEC at UC Santa Barbara) and
programs in \cds{} (such as the CDS program at Caltech) are natural locations
for joint appointments and can act as a catalyst for getting into new areas of
\cds{} by attracting funding and students outside of traditional
disciplines.

\index{new domains, control in}%
\index{control!as enabling technology}
There are many areas ripe for the application of \cds{} and increased
activity in new domains will accelerate the use of \cds{} and enable new
advances and insights.
In many of these new application areas, the 
systems approach championed by the \cds{} community has yet to be applied,
but it will be required for eventual engineering applications.  Perhaps more
important, \cds{} has the opportunity to revolutionize other fields,
especially those where the systems are complicated and difficult to
understand.  Of course, these problems are extremely hard and many previous
attempts have not always been successful, but the opportunities are great and
we must continue to strive to move forward.

\section{Support for Theory and Interaction with Mathematics}

\index{theory!support for}%
\index{mathematics!interaction with}
A core strength of \cds{} has been its respect for and effective use of
theory, as well as contributions to mathematics driven by \cds{} problems.
\index{rigor}%
Rigor is a trademark of the community and one that has been key to 
many of its successes.  
\index{control!successes of}
Continued interaction with mathematics and support for
theory is even more important as the applications for \cds{} become more
complex and more diverse.

\index{abstraction}
\index{compactification}
An ongoing need is making the existing knowledge base more compact so that the
field can continue to grow.  Integrating previous results and providing a more
unified structure for understanding and applying those results is necessary in
any field and has happened many times in the history of \cds{}.  This process
must be continuously pursued and requires steady support for theoreticians
working on solidifying the foundations of \cds{}.  Control experts also need
to expand the applications base by having the appropriate level of abstraction
to identify new applications of existing theory.

To ensure the continued health of the field, the Panel recommends that the
community and funding agencies
\recomm{
  Maintain support for theory and interaction with mathematics, broadly
  interpreted.
}
Some possible areas of interaction include 
dynamical systems, \index{dynamical systems}%
graph theory, \index{graph theory}%
combinatorics,
complexity theory, \index{complex systems}%
queuing theory, \index{queuing theory}%
statistics, \index{statistics}%
etc.  Additional
perspectives on the interaction of control and mathematics can be found
in a recent survey article by Brockett~\cite{Bro00-mathunl}.

A key need is to identify and provide funding mechanisms for people to work on
core theory.  The proliferation of multi-disciplinary, multi-university
programs have supported many worthwhile projects, but they potentially
threaten the base of individual investigators who are working on the theory
that is required for future success.  It is important to leave room
for theorists on these applications-oriented projects and to better articulate
the successes of the past so that support for the theory is appreciated.
Program managers should support a balanced portfolio of applications,
computation, and theory, with clear articulation of the importance of long
term, theoretical results.

\index{national institutes, for control science}
The linkage of \cds{} with mathematics should also be increased, perhaps
through new centers and programs.  Funding agencies should consider funding
national institutes for control science that would engage the mathematics
community, and existing institutes in mathematics should be encouraged to
sponsor year-long programs on control, dynamics, and systems.

\index{control!impact and payoffs}
The benefits of this investment in theory will be a systematic design
methodology for building complex systems and rigorous training for the next
generation of researchers and engineers.

\section{New Approaches to Education and Outreach}

\index{education|(}%
\index{outreach|(}%
\index{accessibility, broadening}
\index{compactification}
As many of the recommendations above indicate, applications of \cds{} are
expanding and this is placing new demands on education.  The community must
continue to unify and compact the knowledge base by integrating materials and
frameworks from the past 40 years.  As important, material must be made more
accessible to a broad range of potential users, well beyond the traditional
base of engineering science students and practitioners.  This includes new
uses of \cds{} by computer scientists, biologists, physicists, and medical
researchers.  The technical background of these constituencies is often very
different than traditional engineering disciplines and will require new
approaches to education.

\index{textbooks}
\index{curriculum development}
\index{control!principles}%
The Panel believes that \cds{} principles are now a required part of any
educated scientist's or engineer's background and we recommend that the
community and funding agencies
\recomm{
  Invest in new approaches to education and outreach for the dissemination of
  control concepts and tools to non-traditional audiences. 
}
As a first step toward implementing this recommendation, new courses and
textbooks should be developed for both experts and non-experts.  
\CDS{}
should also be made a {\em required} part of engineering and science curricula
at major universities, including not only mechanical, electrical, chemical,
and aerospace engineering, but also computer science, ^{applied physics}, and
bioengineering.  It is also important that these courses emphasize the {\em
principles} of \cds{} rather than simply providing tools that can be used in a
given domain.
\index{control!principles}
\index{bioengineering|see{biological engineering}}
\index{biological engineering}

\index{software!control}%
\index{laboratories|see{experiments}}%
\index{curriculum development}
\index{experiments}%
An important element of education and outreach is the continued use of
experiments and the development of new laboratories and software tools.  These
are much easier to do than ever before and also more important. 
Laboratories and software tools should be integrated into the curriculum,
including moving beyond their current use in introductory \cds{} courses to
increased use in advanced (graduate) course work.  The importance of software
cannot be overemphasized, both in terms of design tools (e.~g., ^{MATLAB}
toolboxes) and implementation (real-time algorithms).

\index{industry interaction}%
\index{technology transfer}
Increased interaction with industry in education is another important step.
This could occur through cooperative ^{Ph.D.\ programs} where industrial
researchers 
are supported half by companies and half by universities to pursue Ph.D.'s
(full-time), with the benefits of bringing more understanding of real-world
problems to the university and transferring the latest developments back to
industry.  In addition, industry leaders and executives from the \cds{}
community should continue to interact with the community and help communicate
the needs of their constituencies.

\index{textbooks}
\index{teaching materials}
\index{public awareness}
\index{multimedia tools}
Additional steps to be taken include the development of new teaching materials
that can be used to broadly educate the public about \cds{}.  This might
include 
chapters on \cds{} in high school textbooks in biology, mathematics, and
physics or a multimedia CDROM
that describes the history, principles, successes,
and tools for \cds{}.  
\index{books!popular}
\index{control!principles}%
Popular books that explain the principles of feedback,
or perhaps a ``cartoon book'' on \cds{} should be considered.  The upcoming
IFAC Professional Briefs for use in industry are also an important avenue for
education.
\index{IFAC Professional Briefs}

\index{public awareness}%
\index{control!impact and payoffs}
The benefits of reaching out to broader communities will be an increased
awareness of the usefulness of \cds{}, and acceleration of the benefits of
\cds{} through broader use of its principles and tools.  The use of rigorous
design principles will result in safer systems, shorter development times, and
more transparent understanding of key systems issues.

\index{education|)}%
\index{outreach|)}%

\index{recommendations|)}%
\index{CDS Panel!recommendations|)}%

\section{Concluding Remarks}

The field of \cds{} has a rich history and a strong record of success and
impact in commercial, military, and scientific applications.  The tradition of
rigorous use of mathematics combined with strong interaction with
applications 
has produced a set of tools that are used in a wide variety of technologies.
The opportunities for future impact are even richer than those of the past,
and the field is well positioned to expand its tools to apply to new areas and
applications.

\index{information-based systems}
The pervasiveness of communications, computing and sensing will enable many
new applications of \cds{} but will also require a substantial expansion of
the current theory and tools.  The \cds{} community must embrace new,
information rich applications and generalize existing concepts to apply to
systems at higher levels of decisions making.  Combined with new, long-range
areas that are opening up to \cds{} techniques, the next decade promises to be
a fruitful one for the field.

\index{control!impact and payoffs}
\index{control!principles}%
\index{biology!increased understanding through use of control}
The payoffs for investment in \cds{} research are substantial.  They include
the successful development of systems that operate reliably, efficiently, and
robustly; new materials and devices that are made possible through advanced
control of manufacturing processes; and increased understanding of physical
and biological systems through the use of \cds{} principles.  Perhaps most
important is the continued development of individuals who embrace a systems
perspective and provide technical leadership in modeling, analysis, design and
testing of complex engineering systems.
\index{systems perspective}



\appendix
%% \include{nsfcss}
\clearpage
% Master File: cdspanel.tex

\chapter[NSF/CSS Workshop on Education]
{NSF/CSS Workshop on New Directions in Control Engineering 
Education}
\label{nsfcss.sec}
\index{education|(}

\index{National Science Foundation (NSF)}
\index{Control Systems Society (CSS), IEEE}
{\em The National Science Foundation (NSF) and the IEEE Control Systems Society
(CSS) held a workshop in October 1998 to identify the future needs in control
systems education~\cite{nsfcss99}.  The executive summary of the report is
presented here. 
The full report is available from the CDS Panel homepage.}

\section*{Executive Summary}

The field of control systems science and engineering is entering a
golden age of unprecedented growth and opportunity that will likely
dwarf the advancements stimulated by the space program of the
1960s. These opportunities for growth are being spurred by enormous
advances in computer technology, material science, sensor and actuator
technology, as well as in the theoretical foundations of dynamical
systems and control.  
\index{dynamical systems}
Many of the opportunities for future growth are
at the boundaries of traditional disciplines, particularly at the
boundary of computer science with other engineering
disciplines. Control systems technology is the cornerstone of the new
automation revolution occurring in such diverse areas as household
appliances, consumer electronics, automotive and aerospace systems,
manufacturing systems, chemical processes, civil and environmental
systems, transportation systems, and even biological, economic, and
medical systems.  
\index{consumer electronics}
\index{automotive systems}
\index{aerospace systems}
\index{manufacturing}
\index{medicine}
\index{discipline boundaries}
\index{process control}
\index{environmental science}
\index{transportation systems}
\index{biology}
\index{economic systems}

The needs of industry for well trained control
systems scientists and engineers are changing, due to marketplace
pressures and advances in technology. Future generations of
engineering students will have to be broadly educated to cope with
cross-disciplinary applications and rapidly changing technology. At
the same time, the backgrounds of students are changing. Many come
from nontraditional backgrounds; they often are less well prepared in
mathematics and the sciences while being better prepared to work with
modern computing technologies. The time is thus ripe for major
renovations in control and systems engineering education.

\index{University of Illinois, Urbana-Champaign}
To address these emerging challenges and opportunities, the IEEE
Control Systems Society initiated the idea of holding a workshop that
would bring together leading control systems researchers to identify
the future needs in control systems education. The workshop was held
on the campus of the University of Illinois at Urbana-Champaign,
October 2--3, 1998. It attracted sixty-eight participants.

This report summarizes the major conclusions and recommendations that
emerged from the workshop. A slightly modified version of the main
body of this report will also appear in the October, 1999, issue of
the IEEE Control Systems Magazine. These recommendations, which cover
a broad spectrum of educational issues, are addressed to several
constituencies, including the National Science Foundation, control
systems professional organizations, and control systems researchers
and educators.

\subsection*{1. General Recommendation}

\begin{itemize}\em
  \item[1] Enhance cooperation among various control organizations and
  control disciplines throughout the world to give attention to
  control systems education issues and to increase the general
  awareness of the importance of control systems technology in
  society.
\end{itemize}
Mechanisms to accomplish this include joint sponsorship of
conferences, workshops, conference sessions, and publications devoted
to control systems education as well as the development of books,
websites, videotapes, and so on, devoted to the promotion of control
systems technology.

\subsection*{2. Additional Recommendations}

\begin{itemize}\em
  \item[2] Provide practical experience in control systems engineering
  to freshmen to stimulate future interest and to introduce
  fundamental notions like feedback and the systems approach to
  engineering.
\end{itemize}
\index{feedback}
\index{control!principles}%
\index{courses!undergraduate}
This can be accomplished by incorporating modules and/or projects that
involve principles of control systems into freshmen courses that
already exist in many engineering schools and colleges.

\index{courses!undergraduate}
\begin{itemize}\em
  \item[3] Encourage the development of new courses and course
  materials that will significantly broaden the standard first
  introductory control systems course at the undergraduate level. 
\end{itemize}
\index{control!principles}%
Such new courses would be accessible to all third year engineering
students and would deal with fundamental principles of system
modeling, planning, design, optimization, hardware and software
implementation, computer aided control systems design and
^{simulation}, and systems performance evaluation. Equally important,
such courses would stress the fundamental applications and importance
of feedback control as well as the limits of feedback, and would
provide a bridge between control systems engineering and other
branches of engineering that benefit from systems engineering
concepts such as networks and communications, biomedical engineering,
computer science, economics. etc.
\index{systems engineering}
\index{feedback}
\index{medicine}
\index{computer science!interaction of control and}
\index{economics}
\index{communications!integration of control and}

\index{courses!undergraduate}
\begin{itemize}\em
  \item[4] Develop follow on courses at the undergraduate level that
  provide the necessary breadth and depth to prepare students both for
  industrial careers and for graduate studies in systems and control.
\end{itemize}
Advanced courses in both traditional control methodologies, like
digital control, and courses treating innovative control
applications should be available to undergraduate students in order to
convey the excitement of control systems engineering while still
providing the fundamentals needed in practice.

\begin{itemize}\em
  \item[5] Promote control systems laboratory development, especially
the concept of shared laboratories, and make 
experimental projects an integral part of control education for all students, including graduate students.
  \index{experiments}
\end{itemize}
Mechanisms to accomplish this include increased support for the
development of hands-on control systems laboratories, as well as the
development of benchmark control systems examples that are accessible
via the Internet. Shared laboratories within individual colleges or
universities as well as shared laboratories among different
universities makes more efficient use of resources, increases exposure
of students to the multidisciplinary nature of control, and promotes
the interaction of faculty and students across disciplines.
\index{discipline boundaries}

The promotion of laboratory development also includes mechanisms for
continued support. Too often, laboratories are developed and then
abandoned after a few years because faculty do not have time or funds
for continued support. It is equally important, therefore, to provide
continuity of support for periodic hardware and software upgrades,
maintenance, and the development of new experiments.

\index{short courses}
The National Science Foundation and IEEE Control Systems Society can
also help realize this goal by developing workshops and short courses
for laboratory development and instruction to promote interaction and
sharing of laboratory development experiences among faculty from
different universities.

\begin{itemize}\em
  \item[6] Emphasize the integration of control systems education and
research at all levels of instruction.
\end{itemize}
\index{competitions, student}
The National Science Foundation program, Research Experiences for
Undergraduates, exemplifies an excellent mechanism to accomplish
this at the undergraduate level and should be continued. Sponsorship
of student competitions in control is another such mechanism that
should be encouraged. At the graduate level control educators should
take advantage of National Science Foundation programs such as the
Integrative Graduate Education and Research Training Program (IGERT)
and the Course, Curriculum, and Laboratory Improvement Program (CCLI).
\index{National Science Foundation (NSF)}

\begin{itemize}\em
  \item[7] Improve information exchange by developing a centralized
  Internet repository for educational materials. 
\end{itemize}
These materials should include tutorials, exercises, case studies,
examples, and histories, as well as laboratory exercises, software,
manuals, etc.  The IEEE Control Systems Society can
play a leadership role in the development of such a repository by
coordinating the efforts among various public and private agencies.

\begin{itemize}\em 
  \item[8] Promote the development of a set of
  standards for Internet based control systems materials and identify
  pricing mechanisms to provide financial compensation to Internet
  laboratory providers and educational materials providers.
\end{itemize}
A mechanism to accomplish this could be a National Science Foundation
sponsored workshop devoted to Internet standards for control education
materials and pricing.

\index{journals, electronic}
\begin{itemize}\em
  \item[9] Develop WWW-based peer reviewed electronic journal on
  control education and laboratory development. 
\end{itemize}
Control systems professional organizations can play leadership roles,
perhaps working with the American Society of Engineering Education
(ASEE) to accomplish this goal.

\begin{itemize}\em
  \item[10] Encourage the development of initiatives for technical
  information dissemination to industrial users 
of control systems and encourage the transfer of practical industrial
  experience to the classroom.
\end{itemize}
Mechanisms to accomplish this include special issues of journals and
magazines devoted to industrial applications of control, programs to
bring speakers from industry to the classroom, and programs that allow
university faculty to spend extended periods of time in industry.

\index{education|)}

\backmatter
%% \include{references}
\clearpage
% Master File: cdspanel.tex
\cleardoublepage

\ifx\siam\undefined
\cleardoublepage
\addcontentsline{toc}{chapter}{References}
\fi

\bibliographystyle{plain}
\bibliography{cdspanel}

%\medskip
%\begin{verbatim}
%Additional references:
%  Fleming + previous IEEE report
%  Kumar, Brockett, etc (future directions)
%  EC report + other government reports; NSF education workshop
%  1952 Scientific American issues
%  Out of Control and other popular books (Wiener)
%  http://www.dfrc.nasa.gov/History/Publications/f8ctf/contents.html
%  http://www.theorem.net/theorem/lewis1.html
%\end{verbatim}
%% \include{index}
\clearpage
% Master File: cdspanel.tex

\cleardoublepage
\ifx\siam\undefined
\addcontentsline{toc}{section}{Index}
\footnotesize
\fi

%%
%% Additional indexing commands
%%
%% Define 

% gobble up argument (comma or page number)
\newcommand{\gobble}[1]{}	

% generate a "see also" entry
\newcommand{\alsoname}{see also}
\newcommand{\seealso}[2]{\emph{\alsoname} #1}
\newcommand{\indexalso}[2]{
  \index{#1!zzz@\emph{\alsoname} #2\protect\gobble|gobble}
}
\newcommand{\alsoinline}[2]{
  \index{#1|seealso{#2}}
}


%%
%% See also entries
%%
%% See also entries can either be generated inline or as subitems.  In either
%% case, they should occur at the end of the list of entries, so we place all
%% of the indexing commands here.
%%
%% To generate inline see also entries, use the |seealso construct.  Any other
%% related index entries must share the same heading.
%%
%% To generate subitemed see also entries, Use the \indexalso macro.  All
%% other index entries with the same head should be given as subitems
%% (heading!item).  The first argument gives the heading name, the second
%% gives the cross referenced heading name.
%%

\indexalso{books}{textbooks}
\indexalso{theory}{control, theory; mathematics}
\indexalso{communications}{networks; telephone system}
\indexalso{courses}{short courses}

%%
%% Inlined see also entries
%%
%% Use these entries when there are no subitems for a given heading.
%%

\alsoinline{autopilot}{cruise control}
\alsoinline{discrete mathematics}{formal methods; graph theory}
\alsoinline{asset management}{resource allocation}
\alsoinline{applied physics}{physics}
\alsoinline{resource allocation}{asset management; supply chains}
\alsoinline{robust yet fragile behavior}{cascade failures}
\alsoinline{C4ISR}{command and control}
\alsoinline{cellular phones}{wireless networks}
\alsoinline{component failures, robustness to}{reliability}
\alsoinline{cooperative control}{teams, control of}
\alsoinline{decentralized control}{distributed control}
\alsoinline{defense systems}{battlefield management; command and control}
\alsoinline{distributed control}{decentralized control}
\alsoinline{disturbances}{uncertainty}
\alsoinline{electrical power}{power grid}
\alsoinline{engine control}{gas turbine engines}
\alsoinline{estimation}{Kalman filter}
\alsoinline{fault tolerance}{reliability}
\alsoinline{geometric control}{feedback linearization}
\alsoinline{governors}{centrifugal governor}
\alsoinline{guidance}{mission management; trajectory generation}
\alsoinline{health status, system}{diagnostics}
\alsoinline{intracellular signaling}{bacteria; chemotaxis}
\alsoinline{outer loop}{guidance; trajectory generation}
\alsoinline{path planning}{motion control; trajectory generation}
\alsoinline{pilots}{flight control}
\alsoinline{simulation}{modeling}
\alsoinline{space systems}{aerospace systems}
\alsoinline{transportation systems}{air traffic control; automotive systems}
\alsoinline{vision guided systems}{active vision}
\alsoinline{courses!broader audiences}{accessibility, broadening}
\alsoinline{command and control}{battlefield management; mission management}
\alsoinline{airspace management}{air traffic control}
\alsoinline{applied mathematics}{mathematics}
\alsoinline{autocode}{automatic synthesis}

%%
%% See instead entries
%%
%% These entries generate see (instead) listings.  There should be no other
%% index entries with the same head.
%%

\index{accomplishments|see{control, successes of}}
\index{asset allocation|see{resource allocation}}
\index{semi-autonomous systems|see{autonomous systems}}
\index{UAVs|see{unmanned vehicles}}
\index{autonomous vehicles|see{unmanned vehicles}}
\index{cars|see{automobiles}}
\index{CDROM|see{multimedia tools}}
\index{cells|see{molecular biology}}
\index{chemical processing|see{processing, chemical}}
\index{undergraduate courses|see{courses}}
\index{higher level reasoning|see{reasoning; decision making}}
\index{automated highways|see{intelligent highways}}
\index{highways|see{intelligent highways}}
\index{delay|see{time delay}}
\index{ordinary differential equations (ODEs)|see{differential equations}}
\index{differential games|see{game theory}}
\index{surgery|see{image guided surgery; minimally invasive surgery}}
\index{electronics|see{amplifiers; consumer electronics; microelectronics}}
\index{embedded software|see{software}}
\index{finance|see{economic systems}}
\index{numerically controlled machines|see{computer numerically controlled
machines}}
\index{noise|see{disturbances}}
\index{panel|see{CDS Panel}}
\index{robustness|see{reliability; uncertainty management}}
\index{semiconductors|see{microelectronics}}
\index{successes|see{control, successes}}
\index{integration of control!with computation|see{computation}}
\index{integration of control!with communications|see{communications}}%

% Print out the index
\small
\printindex

\ifx\siam\undefined\normalsize\fi

\end{document}
