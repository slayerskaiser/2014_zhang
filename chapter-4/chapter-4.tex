\documentclass[../zhang_thesis.tex]{subfiles}
\begin{document}

\chapter{Discussion}

%%%%%%%%%%%%%%%%%%%%%%%%%%%%%%%%%%%%%%%%%%%%%%%%%%%%%%%%%%%%%%%

\section{Summary}

For the problem of estimating a battery's SOC using a high-accuracy model that describes the nonlinear and transient effects of a battery, an electrical-circuit battery model was chosen, and the accuracy and speed of various nonlinear filters were compared through simulation. It was shown that the accuracy of the prediction phases of the filters could generally be increased by using iterated numerical integration. Overall, the EKF is the fastest filter and is the most accurate for long sampling periods. Its speed comes mainly from its evaluation of only one state point, while the other filters apply the nonlinear function to multiple points. For the shorter sampling period of 30~seconds, the CKF5 and SLF are the best performing but not by much. The reason for the overall performance of the EKF is unknown. The general recommendation would be to use the EKF for estimation using the chosen electrical-circuit model, because its accuracy is either the best or very close to the best while being several times faster than the next fastest filter.

\section{Future Work}

Future work should focus on the update phases of the filters, since that is mainly where the various filters differ. This can be done by, for example, using the EKF prediction for all the filters, and using different updates. This would better isolate inaccuracies arising from the numerical integration so that the accuracy of the estimation of the nonlinearities of the different filters can be compared. Additionally, filtering at shorter sampling periods should be examined, since again, the inaccuracies from the numerical integration would be decreased. The lower error of the CKF5 and the SLF at a sampling period of 30~seconds could indicate that the nonlinearities are stronger for short sampling periods rather than for longer ones.

\end{document}
