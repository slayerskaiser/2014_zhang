\chapter{Discussion}

This chapter is organized as follows. Section 4.1 discusses the main raw materials. Section 4.2 determines the glazing method. Section 4.3 compares the results of this study to that of prior studies.

%%%%%%%%%%%%%%%%%%%%%%%%%%%%%%%%%%%%%%%%%%%%%%%%%%%%%%%%%%%%%%%

%\section{Production Method}

\section{Raw Materials}

\subsection{Quartz}

There are two main sources of quartz: crushed quartz pebbles and quartz sand. Crushed quartz pebbles are generally of a high purity, in particular low alumina, lime, and iron oxide contents, and have angular particles. Quartz sand can contain significant amounts of limestone and shell fragments, feldspars and iron-titanium oxides. Also, the sand particles tend to be rounded~\cite{tite07}.

The quartz found in the beads are mostly angular, but there are some rounded grains. It is generally of a high purity, but some contaminates of calcium oxide and some form of titanium oxide. These factors suggest that the quartz is mainly from crushed quartz pebbles with small additions of quartz sand (to account for the impurities). Additionally, the quartz is fine-grained with a diameter of roughly 20~\textmu m. The quartz would have required grinding to reach this small size.

\subsection{Alkali Flux}

Two two sources of alkali flux are natron and the ash of halophytic plants. Natron is mostly sodium carbonate and sodium bicarbonate and has very low potash, lime, and magnesia contents. In contrast, plant ash is a source for all of those~\cite{tite07}.

The beads contain high levels of both sodium and potassium, so the alkalis were probably added in the form of plant ash. It is interesting to note that bead 2 only has traces of calcium and magnesium. Tite \emph{et al.}\ suggested that this could be caused by a purification of the plant ashes by dissolution to leave insoluable calcium and magnesium carbonates~\cite{tite07}.

\subsection{Colorants}

Both copper and calcium oxides are present as colorants in the glaze. The presence of both oxides means that the blue color of the beads is primarily from Egyptian blue~\cite{kaczmarczyk83,lucas62}. Small amounts of tin oxide were found, which suggests that the source of the copper was almost certainly from the oxidation of bronze with 5--10\% tin~\cite{tite07}. Bronze was available in Egypt starting in the New Kingdom. The calcium oxide was probably due to the use of quartz sand, which contains such impurities as shown in \autoref{fig:is_4000x}b.

\section{Glazing Method}

The small size of the faience beads and their mass production mean that cementation and efflorescence were more appropriate methods than application---it would have been slow and tedious to apply glaze to numerous beads. The cementation and efflorescence methods would be better suited for the beads and would have caused the formation of interparticle glass to give strength to the small objects. Both methods required about the same amount of labor and similar lengths of time---cementation needed a longer firing time, but efflorescence required a drying period. The cementation method was probably more economical in terms of fuel usage, because many layers of beads could be placed in the glazing mixture and fired at the same time. With the discovery of the cementation method during the Middle Kingdom~\cite{nicholson00}, production of faience beads potentially switched to this cost-effective method. The beads in this thesis were made after the New Kingdom, so it is likely that the cementation method was used. However, Tite \emph{et al.}\ noted a conservatism in the production technology of Egyptian faience and stated that the glazing method did not change from the Middle Kingdom to the 22nd Dynasty (c.\ 2050--950 BC)~\cite{tite07}, so efflorescence is also likely. There is also the possibility that the cementation and efflorescence methods were combined to produce more interparticle glass~\cite{vandiver83}.

A more accurate prediction of the glazing method can be produced by taking into account the composition and microstructures of each bead. Tite and Bimson, Tite \emph{et al.}\, and Vandiver showed that the compositional and microstuctural data can be used to determine the glazing method~\cite{tite86,tite07,vandiver98}. Their criteria for determining the glazing method are detailed in Chapter 2. The remainder of this section applies their criteria to each bead.

\subsection{Bead 1}

It is difficult to determine the production method for bead 1, because only the outer surface of the glaze was analyzed. No drying or firing marks are on the visible portion of the sample, which suggests that glazing by application was not used, though these marks could be present on the hidden portion of the sample. Additionally, glazing by application is difficult to perform on a small object such as the bead. However, it is difficult to rule out the application method without examining the thicknesses of the glaze, interaction layer, and body as well as the extent of the interparticle glass in the body. Even so, using the criteria in Chapter 2, the glazing method was probably cementation, efflorescence, or a combination of the two.

\subsection{Bead 2}

Bead 2 has some interparticle glass, to strengthen the small bead, and no interaction layer. These observations suggest that the bead was glazed by efflorescence. However, the greater concentration of copper in the glaze versus the body (see \autorefs{fig:os_bulk_500x}b and \ref{fig:t_3_4000x}b versus \autoref{fig:is_1000x}b) rules out pure efflorescence glazing. The efflorescence method was probably used with one of the other methods in order to increase the strength of the bead. It is very difficult to determine which one of the other methods was used in conjunction, because the efflorescence characteristics are dominant~\cite{vandiver83,nicholson93,tite83}. By taking into account which methods would have been appropriate for the bead (cementation and efflorescence), this thesis believes that this bead was glazed by a combination of the cementation and efflorescence methods.

\subsection{Bead 3}

Bead 3 has extensive interparticle glass, a large interaction layer, and a thin wall, which suggest glazing by cementation or efflorescence. The rapid decrease in the concentration of copper from the glaze to the interaction layer (see \autorefs{fig:s2a_wall_xect_map}e and \ref{fig:s2a_wall_end_map}e) is characteristic of the cementation method. The diffuse boundaries between the glaze, interaction layer, and body (see \autoref{fig:s3b_composite}) are also characteristic of the cementation method. However, notice that \autoref{fig:s2a} shows that the interaction layer and glaze are thicker near the end of the bead versus the middle. This suggests efflorescence glazing--in which the greater air contact at the end of the bead would have caused faster drying, and thus, a thicker interaction layer---but these microstructures could also have been caused by the cementation method because of the greater contact with the glazing mixture at the end of the bead. Therefore, this bead was probably glazed by the cementation method, though a combination of the cementation and efflorescence methods may have been used.

%%%%%%%%%%%%%%%%%%%%%%%%%%%%%%%%%%%%%%%%%%%%%%%%%%%%%%%%%%%%%%%

%\section{Imperfections}

%%%%%%%%%%%%%%%%%%%%%%%%%%%%%%%%%%%%%%%%%%%%%%%%%%%%%%%%%%%%%%%

\section{Comparison with Previous Studies}

This section compares the compositional and microstructural data of this study to that of previous ones. Only beads 2 and 3 are compared as the composition and microstructures of bead 1 were not analyzed. Unfortunately, the EDS system used in this investigation was not calibrated, so the compositional data are not directly comparable to the results of prior studies. Instead, the compositional results are compared qualitatively. In contrast, the microstructural data is directly compared with that of previous results. The main comparison is with Tite \emph{et al.} They studied beads from Abydos with wall-thicknesses of 800-1200~\textmu m~\cite{tite07}, which are directly comparable to the beads analyzed in this study. Additionally, there are comparisons with studies on replicate faience by Tite \emph{et al.} and by Vandiver.

\subsection{Ancient Faience Comparisons}

The study by Tite \emph{et al.}\ encountered compositions and microstructures in their Abydos beads similar to those found in this study. In terms of composition, they encountered a pronounced drop in the copper oxide content from the glaze to the interaction layer. They also noted the presence of sodium, potassium, calcium, and magnesium with traces of tin in the beads from the 22nd Dynasty. The beads analyzed by this thesis had the same compositions qualitatively. In terms of microstructures, their beads colored by copper had barely distinguishable interaction layers and bodies of angular quartz particles in continuous glass matrices, while their black beads had defined interaction layers~\cite{tite07}. The microstructures of bead 2 show no interaction layer (similar to the beads colored by copper studied Tite \emph{et al.}), while those of bead 3 show a pronounced interaction (similar to the black beads studied by Tite \emph{et al.}). The rest of the microstructures are the same between the two studies.

Because of the similarity in the composition and microstructures of the beads, the study by Tite \emph{et al.}\ reached similar conclusions on the nature of the raw materials and the glazing method for their Abydos beads. In common with this thesis, they decided the quartz was crushed quartz pebbles, the alkali flux was plant ash, and the copper oxide was from oxidized bronze. However, while this thesis thought the glazing was a combination of the cementation and efflorescence method, they thought the glazing was by either efflorescence or cementation for the beads colored by copper and a hybrid of the two methods for the black beads~\cite{tite07}. Still, the conclusions on glazing methods are similar for the two studies.

Also in the study by Tite \emph{et al.}\ were faience beads from Esna and Amarna. They note many differences in microstructure among the beads from these different areas of Egypt, such as the thicknesses of the glaze and interaction layer, the amount of interparticle glass, the amount of pores on the surface and inside the body, and the definition of the boundaries between the layers; they also noted some differences in the concentration profiles~\cite{tite07}. Therefore, comparisons between beads of this study with beads from Esna or Amarna were not made, as the differences would have been too numerous. The differences in faience among these production sites could be of scholary interest.

\subsection{Replicate Faience Comparisons}

The comparisons with replicate faience will focus on the microstructures, because the composition of ancient faience may have changed due to weathering or through use and can vary widely depending on the choice of raw materials. In addition, only the efflorescence and cementation replicates are considered, because these are the likely glazing methods for the beads studied by this thesis.

Bead 2 has a thick glaze ($\sim$190~\textmu m) and no interaction layer, which is only seen in replicates with an applied glaze. However, it also has moderate amounts of interparticle glass, which is never found in applied replicates~\cite{tite86,vandiver98}. In addition, the concentration of copper oxide at the surface in the glaze suggests the cementation method~\cite{vandiver98}. These characteristics suggest that glazing was done with a combination of methods, most likely application and efflorescence. This is contrary to the conclusions of the current investigation and previous ones by Tite \emph{et al.}\ and Vandiver on ancient faience~\cite{tite07,vandiver98,tite86}---the application method is not appropriate for the glazing of small beads.

Bead 3 has a thin glaze ($\sim$50~\textmu m) and an interaction layer of 100--150~\textmu m. There is also extensive interparticle glass in the body. These characteristics are consistent with cementation replicates~\cite{vandiver98,tite83,tite86,tite07}. This agrees with the belief of this thesis that the bead was glazed by cementation.