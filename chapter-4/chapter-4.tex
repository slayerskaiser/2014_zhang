\documentclass[../zhang_thesis.tex]{subfiles}
\begin{document}

\chapter{Discussion}

%%%%%%%%%%%%%%%%%%%%%%%%%%%%%%%%%%%%%%%%%%%%%%%%%%%%%%%%%%%%%%%

\section{Summary}

For the problem of estimating a battery's SOC using a high-accuracy model that describes the nonlinear and transient effects of the battery, an electrical-circuit battery model was chosen, and the accuracy and speed of the SOC estimation were compared through simulation for the EKF, UKF, CKF3, CKF5, and SLF. The tested filters were all
able to accurately estimate the SOC at the tested sampling periods of 30, 150, and 300 seconds. Additionally, it was shown that the accuracy of the prediction phases of the filters could generally be increased by using iterated numerical integration rather than a single-step method.

Overall, the EKF was the fastest filter and was the most accurate for long sampling periods. Its speed comes mainly from its evaluation of only one state point, while the other filters apply the nonlinear function to multiple points. For the shorter sampling period of 30~seconds, the CKF5 and the SLF were the best performing but not by much, which could be a result of the short-term transient effects increasing the strength of the nonlinear effects for comparable length sampling periods. Thus, while
the higher-order filters can handle the additional nonlinear effect, the first-order EKF could not as well and performs worse. It would be interesting to see whether this outperformance of the EKF is seen at even shorter sampling periods. The reason for the overall performance of the EKF is unknown since it is the least complex filter. The general recommendation would be to use the EKF for estimation using the chosen electrical-circuit model, because its accuracy is either the best or very close to the best while being several
times faster than the next fastest filter.

Among the rest of the filters, the UKF had the most numerical problems, and its error improved less for increased numbers of integration steps than the CKFs and the SLF. However, there was a small range of integration steps for which the UKF exhibited lower error than them. This range was for small numbers of integration steps that were still large enough for divergence-free operation of the UKF. The CKF3 and the CKF5 were very close in performance for small numbers of integration steps,
but the CKF5 had slightly less error for large numbers of integration steps, as is expected due to its higher order. The SLF had approximately the same error as the CKF5 for all tested sampling periods and integration steps. Thus, the SLF, using the same third-order cubature rule as the CKF3 to calculate its expectations, is preferable to both CKFs, because it had similar error to the higher-order CKF5 while being faster than either CKF.

\section{Future Work}

Future work should focus on the update phases of the filters, since that is mainly where the various filters differ. This can be done, for example, by using the EKF prediction for all the filters and different update steps. This would better isolate inaccuracies arising from the numerical integration so that the accuracy of the estimation of the nonlinearities of the different filters can be compared. Additionally, filtering at shorter sampling periods should be examined,
since again, the inaccuracies from the numerical integration would be decreased because fewer integration steps would be necessary for divergence-free operation. The lower error of the CKF5 and the SLF at a sampling period of 30~seconds could indicate that the nonlinearities are stronger for short sampling periods rather than for longer ones. This could possibly be due to the short-term transient effects described by the battery model.

Furthermore, the chosen battery model is likely able to model battery chemistries other than lithium-ion. This is because the nonlinear part of the model uses information about the impedance of the battery. For a general chemical battery, the impedance can be determined using EIS or pulsed-discharge experiments. It would be interesting to test whether other battery types can be modeled using the same model and whether the performance advantage of the EKF over the other tested filters holds for over those battery types.

Moreover, the theoretical reasons why the filters performed the way they did should be studied. Particularly, the reasons why the changes to the UKF and CKF proposed by the author improved their performance should be studied. Knowledge of this could result in the creation of new, more accurate filters.

\end{document}
