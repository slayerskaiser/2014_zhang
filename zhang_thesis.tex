% Only one of the following lines should be used at a time.
%\documentclass[draft,phd,12pt]{psuthesis}
%\documentclass[draft,phd,inlinechaptertoc]{psuthesis}
%\documentclass[draft,ms]{psuthesis}
%\documentclass[draft,honorsdepthead,honors]{psuthesis}
%\documentclass[honors,12pt,draft]{psuthesis}
\documentclass[honors,12pt]{psuthesis}
%\documentclass[draft,secondthesissupervisor,honors]{psuthesis}
%\documentclass[draft,bs]{psuthesis}


%%%%%%%%%%%%%%%%%%%%%%%%%%%%
% Packages we like to use. %
%%%%%%%%%%%%%%%%%%%%%%%%%%%%
\usepackage{subfiles} % allows for better include support
%\usepackage{pslatex}
\usepackage[T1]{fontenc}
%\usepackage{amsmath}
\usepackage{mathtools}
\usepackage{amssymb}
\usepackage{amsthm}
%\usepackage{exscale}
%\usepackage[mathscr]{eucal}
%\usepackage{bm}
\usepackage{eqlist} % Makes for a nice list of symbols.
\usepackage[dvipsnames]{color}
%\usepackage{varioref}
\usepackage{booktabs}
\usepackage[version=3]{mhchem}
%\usepackage[caption=false]{subfig}
\usepackage{subcaption}
%\usepackage{url}
%\usepackage{psfrag}
%\usepackage[final]{graphicx}
%\DeclareGraphicsExtensions{.pdf, .jpg}
\graphicspath{{chapter-1/figures/}{chapter-2/figures/}{chapter-3/figures/}{chapter-4/figures/}{appendix-a/figures/}}
%\usepackage[first,timestamp]{draftcopy}
%\usepackage{rotating}
%\usepackage{ctable}
\usepackage[inline]{enumitem}
%\usepackage{lmodern}
%\usepackage{textcomp}
\usepackage{setspace}
\usepackage{rotating}
\usepackage[style=ieee,citestyle=numeric-comp,backend=biber,sorting=none,doi=false,isbn=false]{biblatex}
\addbibresource{Biblio-Database.bib}
%\usepackage{epstopdf}
%\epstopdfsetup{update} % only regenerate pdf files when eps file is newer
\usepackage[crop=pdfcrop,process=auto]{pstool}

%\usepackage[final,tracking=true,kerning=true,spacing=true,factor=1100,stretch=10,shrink=10]{microtype}
\usepackage[final]{microtype}
%\microtypecontext{spacing=nonfrench}
% activate={true,nocompatibility} - activate protrusion and expansion - (not needed; also works for dvi now)
% final - enable microtype; use ``draft'' to disable
% tracking=true, kerning=true, spacing=true - activate these techniques
% factor=1100 - add 10% to the protrusion amount (default is 1000)
% stretch=10, shrink=10 - reduce stretchability/shrinkability (default is 20/20)

% Define math scaling command
\newcommand\scalemath[2]{\scalebox{#1}{\mbox{\ensuremath{\displaystyle #2}}}}
% Define some math operators
\DeclareMathOperator{\sgn}{sgn}
% Define some math abbreviations
\newcommand{\vx}{\mathbf{x}}
\newcommand{\vu}{\mathbf{u}}
\newcommand{\vz}{\mathbf{z}}
\newcommand{\vw}{\mathbf{w}}
\newcommand{\vv}{\mathbf{v}}
\newcommand{\vf}{\mathbf{f}}
\newcommand{\vh}{\mathbf{h}}


\doublespacing


%%%%%%%%%%%%%%%%%%%%%%%%
% Setup caption format %
%%%%%%%%%%%%%%%%%%%%%%%%
%\captionsetup[subfloat]{listofformat=parens}
%\captionsetup{lofdepth=2}

%%%%%%%%%%%%%%%%%%%%%%%%
% Setting for fncychap %
%%%%%%%%%%%%%%%%%%%%%%%%
% Comment out or remove the next two lines and you will get
% the standard LaTeX chapter titles. We like these A LOT
% better.
\usepackage[Lenny]{fncychap}
%\ChTitleVar{\Huge\sffamily\bfseries}
\ChTitleVar{\Huge\rm\bfseries}

%%%%%%%%%%%%%%%%%%%%%%%%%%%
% Define title and author %
%%%%%%%%%%%%%%%%%%%%%%%%%%%
% Useful for matching between paper and PDF
\def\Title{Comparison of Nonlinear Filtering Methods for Battery State of Charge Estimation}
\def\Author{Klaus Zhang}

%%%%%%%%%%%%%%%%%%%%%%%%%%%%%%%
% Use of the hyperref package %
%%%%%%%%%%%%%%%%%%%%%%%%%%%%%%%
%
% This is optional and is included only for those students
% who want to use it.
%
% To the hyperref package, uncomment the following line:
\usepackage[linktocpage]{hyperref}
\usepackage[all]{hypcap}
\usepackage{breakurl}
%
% Note that you should also uncomment the following line:
\renewcommand{\theHchapter}{\thepart.\thechapter}
%
% to work around some a problem hyperref has with the fact
% the psuthesis class has unnumbered pages after which page
% counters are reset.
%\hypersetup{pdftitle={},
%	pdfauthor={Klaus Zhang},bookmarksnumbered,pdfstartview=FitH}
\hypersetup{pdftitle={\Title},pdfauthor={\Author},bookmarksnumbered,pdfstartview=FitH}
%
%\labelformat{equation}{\textup{(#1)}}
%\labelformat{enumi}{\textup{(#1)}}
%\renewcommand\thesubfigure{(\alph{subfigure})}
%
\let\orgautoref\autoref
\providecommand{\autorefs}
        {\def\equationautorefname{Eqs.}%
         \def\figureautorefname{Figures}%
         \def\subfigureautorefname{Figures}%
         \def\sectionautorefname{Sections}%
         \def\subsectionautorefname{Sections}%
         \def\subsubsectionautorefname{Sections}%
         \def\Itemautorefname{items}%
         \def\tableautorefname{Tables}%
         \orgautoref}
\renewcommand{\autoref}
        {\def\equationautorefname{Eq.}%
         \def\figureautorefname{Figure}%
         \def\subfigureautorefname{Figure}%
         \def\sectionautorefname{Section}%
         \def\subsectionautorefname{Section}%
         \def\subsubsectionautorefname{Section}%
         \def\Itemautorefname{Item}%
         \def\tableautorefname{Table}%
         \orgautoref}
%
\usepackage[capitalise,noabbrev]{cleveref}

%%%%%%%%%%%%%%%%%%%%%%%%%%%%%%%%%%%%
% SPECIAL SYMBOLS AND NEW COMMANDS %
%%%%%%%%%%%%%%%%%%%%%%%%%%%%%%%%%%%%
% Place user-defined commands below.



%%%%%%%%%%%%%%%%%%%%%%%%%%%%%%%%%%%%%%%%%
% Renewed Float Parameters              %
% (Makes floats fit better on the page) %
%%%%%%%%%%%%%%%%%%%%%%%%%%%%%%%%%%%%%%%%%
\renewcommand{\floatpagefraction}{0.85}
\renewcommand{\topfraction}      {0.85}
\renewcommand{\bottomfraction}   {0.85}
\renewcommand{\textfraction}     {0.15}

%%%%%%%%%%%%%%%%%%%%%%%%%%%%%%%%%%%%%%%%%%%%%%%%%%%%%%%%%%%%%
% Adjust the equation numbering style to be within sections %
%%%%%%%%%%%%%%%%%%%%%%%%%%%%%%%%%%%%%%%%%%%%%%%%%%%%%%%%%%%%%
\numberwithin{equation}{chapter}

%%%%%%%%%%%%%%%%%%%%%%%%%%%%%%%%%%%%%%%
% Define a narrow environment         %
% (as seen in epslatex documentation) %
%%%%%%%%%%%%%%%%%%%%%%%%%%%%%%%%%%%%%%%
\newenvironment{narrow}[2]{%
\begin{list}{}{%
\setlength{\topsep}{0pt}%
\setlength{\leftmargin}{#1}%
\setlength{\rightmargin}{#2}%
\setlength{\listparindent}{\parindent}%
\setlength{\itemindent}{\parindent}%
\setlength{\parsep}{\parskip}}%
\item[]}{\end{list}}

% ----------------------------------------------------------- %

%%%%%%%%%%%%%%%%
% FRONT-MATTER %
%%%%%%%%%%%%%%%%
% Title
%\title{Comparison of Nonlinear Filtering Methods for Battery State of Charge Estimation}
\title{\Title}

% Author and Department
%\author{Klaus Zhang}
\author{\Author}
\dept{Department of Electrical Engineering}
% the degree will be conferred on this date
\degreedate{May 2014}
% year of your copyright
\copyrightyear{2014}

% This command is used for students submitting a thesis to the
% Schreyer Honors College. The argument of this command should
% contain every after the word ``requirements'' that appears on
% the title page. This provides the needed flexibility for
% all the degree types.
\honorsdegreeinfo{Masters of Science \\ in \\ Engineering \\ Electrical Engineering}

% This is the document type. For example, this could also be:
%     Comprehensive Document
%     Thesis Proposal
\documenttype{Thesis}

% This will generally be The Graduate School, though you can
% put anything in here to suit your needs.
\submittedto{The Graduate Falculty}


%%%%%%%%%%%%%%%%%%
% Signatory Page %
%%%%%%%%%%%%%%%%%%
% You can have up to 7 committee members, i.e., one advisor
% and up to 6 readers.
%
% Begin by specifying the number of readers.
\numberofreaders{2}

% For baccalaureate honors degrees, enter the name of your
% honors adviser below.
%\honorsadviser{John D. Mitchell}

% For baccalaureate honors degrees, if you have a second
% Thesis Supervisor, enter his or her name below.
%\secondthesissupervisor{Jeffrey L. Schiano}

% For baccalaureate honors degrees, certain departments
% (e.g., Engineering Science and Mechanics) require the
% signature of the department head. In that case, enter the
% name of your department head below.
%\honorsdepthead{Department Q. Head}

% Input reader information below. The optional argument, which
% comes first, goes on the second line before the name.
%\advisor[Honors Advisor]
        %{Paul Howell}
        %{Professor of Metallurgy}

%\readerone[Honors Advisor]
          %{Elizabeth Walters}
          %{Associate Professor of Art History}

%\readertwo[Faculty Reader]
          %{Digby MacDonald}
          %{Distinguished Professor of Materials Science and Engineering}

%\readerthree[]
            %{Dana Kletchka}
            %{Affiliate Instructor of Art Education}

%\readerfour[Optional Title Here]
%           {Reader Name}
%           {Professor of SomeThing}
%
%\readerfive[Optional Title Here]
%           {Reader Name}
%           {Professor of SomeThing}

% Makes use of LaTeX's include facility. Add as many chapters
% and appendices as you like.
%\includeonly{%
%Chapter-1/Chapter-1,%
%Chapter-2/Chapter-2,%
%Chapter-3/Chapter-3,%
%Chapter-4/Chapter-4,%
%Chapter-5/Chapter-5,%
%Appendix-A/Appendix-A,%
%Appendix-B/Appendix-B%
%}

%%%%%%%%%%%%%%%%%
% THE BEGINNING %
%%%%%%%%%%%%%%%%%
\begin{document}
%%%%%%%%%%%%%%%%%%%%%%%%
% Preliminary Material %
%%%%%%%%%%%%%%%%%%%%%%%%
% This command is needed to properly set up the frontmatter.
\frontmatter

%%%%%%%%%%%%%%%%%%%%%%%%%%%%%%%%%%%%%%%%%%%%%%%%%%%%%%%%%%%%%%
% IMPORTANT
%
% The following commands allow you to include all the
% frontmatter in your thesis. If you don't need one or more of
% these items, you can comment it out. Most of these items are
% actually required by the Grad School -- see the Thesis Guide
% for details regarding what is and what is not required for
% your particular degree.
%%%%%%%%%%%%%%%%%%%%%%%%%%%%%%%%%%%%%%%%%%%%%%%%%%%%%%%%%%%%%%
% !!! DO NOT CHANGE THE SEQUENCE OF THESE ITEMS !!!
%%%%%%%%%%%%%%%%%%%%%%%%%%%%%%%%%%%%%%%%%%%%%%%%%%%%%%%%%%%%%%

% Generates the signature page. This is not bound with your
% thesis.
%\psusigpage

% Generates the title page based on info you have provided
% above.
\psutitlepage

% Generate the signature page
%\psusigpage

% Generates the committee page -- this is bound with your
% thesis. If this is an baccalaureate honors thesis, then
% comment out this line.
%\psucommitteepage

% Generates the abstract. The argument should point to the
% file containing your abstract. 
\setcounter{page}{1}
\thesisabstract{SupplementaryMaterial/Abstract}

% Generates the Table of Contents
\thesistableofcontents

% Generates the List of Figures
\thesislistoffigures

% Generates the List of Tables
\thesislistoftables

% Generates the List of Symbols. The argument should point to
% the file containing your List of Symbols. 
%\thesislistofsymbols{SupplementaryMaterial/ListOfSymbols}

% Generates the Acknowledgments. The argument should point to
% the file containing your Acknowledgments. 
\thesisacknowledgments{SupplementaryMaterial/Acknowledgments}

% Generates the Epigraph/Dedication. The first argument should
% point to the file containing your Epigraph/Dedication and
% the second argument should be the title of this page. 
\thesisdedication{SupplementaryMaterial/Dedication}{Dedication}



%%%%%%%%%%%%%%%%%%%%%%%%%%%%%%%%%%%%%%%%%%%%%%%%%%%%%%
% This command is needed to get the main part of the %
% document going.                                    %
%%%%%%%%%%%%%%%%%%%%%%%%%%%%%%%%%%%%%%%%%%%%%%%%%%%%%%
\thesismainmatter

%%%%%%%%%%%%%%%%%%%%%%%%%%%%%%%%%%%%%%%%%%%%%%%%%%
% This is an AMS-LaTeX command to allow breaking %
% of displayed equations across pages. Note the  %
% closing the "}" just before the bibliography.  %
%%%%%%%%%%%%%%%%%%%%%%%%%%%%%%%%%%%%%%%%%%%%%%%%%%
\allowdisplaybreaks{
%
%%%%%%%%%%%%%%%%%%%%%%
% THE ACTUAL CONTENT %
%%%%%%%%%%%%%%%%%%%%%%
% Chapters
\subfile{chapter-1/chapter-1}
\subfile{chapter-2/chapter-2}
\subfile{chapter-3/chapter-3}
\subfile{chapter-4/chapter-4}
%\subfile{chapter-5/chapter-5}
%\subfile{chapter-6/chapter-6}
%%%%%%%%%%%%%%%%%%%%%%%%%%%%%%%%%%%%%%%%%%%%%%%%%%%%%%%%%%%%%%%
% Appendices
%
% Because of a quirk in LaTeX (see p. 48 of The LaTeX
% Companion, 2e), you cannot use \include along with
% \addtocontents if you want things to appear the proper
% sequence. Since the PSU Grad School requires 
%%%%%%%%%%%%%%%%%%%%%%%%%%%%%%%%%%%%%%%%%%%%%%%%%%%%%%%%%%%%%%%
%\appendix
%%\Appendix{Vita}

\begin{singlespace}
%\setlength{\parskip}{\baselineskip}
%\newlength{\vitaskip}
%\setlength{\vitaskip}{-22pt}
%\newlength{\vitaitemsep}
%\setlength{\vitaitemsep}{-3.5pt}

The author was born in Nanjing, China. He obtained his Bachelor's of Science in electrical engineering from the Pennsylvania State University in 2011. He attended the University of New Orleans to pursue a Master's of Science in electrical engineering, and performed research with Dr. Huimin Chen.
\end{singlespace}
%%%%%%%%%%%%%%%%%%%%%%%%%%%%%%%%%%%%%%%%%%%
} % End of the \allowdisplaybreak command %
%%%%%%%%%%%%%%%%%%%%%%%%%%%%%%%%%%%%%%%%%%%

%%%%%%%%%%%%%%%%
% BIBLIOGRAPHY %
%%%%%%%%%%%%%%%%
% You can use BibTeX or other bibliography facility for your
% bibliography. LaTeX's standard stuff is shown below. If you
% bibtex, then this section should look something like:
\clearpage
\begin{singlespace}
    \phantomsection
    %\bibliographystyle{FG-bibstyle}
    \addcontentsline{toc}{chapter}{Bibliography}
    %\bibliography{Biblio-Database}
    \printbibliography
\end{singlespace}

%\allowdisplaybreaks{
%\appendix
%%\Appendix{Vita}

\begin{singlespace}
%\setlength{\parskip}{\baselineskip}
%\newlength{\vitaskip}
%\setlength{\vitaskip}{-22pt}
%\newlength{\vitaitemsep}
%\setlength{\vitaitemsep}{-3.5pt}

The author was born in Nanjing, China. He obtained his Bachelor's of Science in electrical engineering from the Pennsylvania State University in 2011. He attended the University of New Orleans to pursue a Master's of Science in electrical engineering, and performed research with Dr. Huimin Chen.
\end{singlespace}
%}

%\nocite{*}

\backmatter

% Vita
\phantomsection
\addcontentsline{toc}{chapter}{Vita}
\vita{SupplementaryMaterial/Vita}

\end{document}
